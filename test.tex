\documentclass{ecnuthesis}

\ecnuSetup {
  % 参数设置
  % 允许采用两种方式设置选项:
  %   1. style/... = ...
  %   2. style = { ... = ... }
  % 注意事项:
  %   1. 请勿在参数设置中出现空行
  %   2. "=" 两侧的空格将被忽略
  %   3. "/" 两侧的空格不会被忽略
  %   4. 请使用英文逗号 "," 分隔选项
  %
  % info 类用于输入论文信息
  info = {
    title = {讲义},
    % 中文标题
    %
    author = {},
    % 姓名
    %
  },
  % style 类用于简单设置论文格式
  style = {
    footnote  = plain,
    % 脚注编号样式
    % 可用选项:
    %   footnote = plain|circled
    % 说明:
    %   plain     脚注的编号仅为数字
    %   circled   脚注的编号为带圆圈数字 (仅限1-10)
    %   (默认选项为 plain )
    %
    numbering = arabic,
    % 章节编号样式
    % 可用选项:
    %   numbering = arabic|alpha|chinese
    % 说明:
    %   arabic    使用数字进行编号 (即理科要求)
    %   alpha     使用字母进行编号 (即外文要求)
    %   chinese   使用汉字进行编号 (即文科要求)
    %   (默认选项为 arabic )
    %
    fontCJK = mac,
    % 中文字体选择
    % 可用选项:
    %   fontCJK = fandol|windows|mac|default
    % 说明:
    %   fandol    使用 TeX 自带的 fandol 字体
    %   windows   使用 Windows 系统内的字体 (中易)
    %   mac       使用 MacOS 系统内的字体
    %   (默认选项为 fandol )
    %
    bibResource = {./source/thesis-ref.bib},
    % 参考文献数据源
    % 由于使用的是 biber + biblatex , 所以必须明确给出 .bib 后缀名
    %
    logoResource = {./source/inner-cover(contains_font).eps},
    % 封面插图数据源
    % 模版已自带, 位于 ./source/inner-cover(contains_font).eps
    % 默认值为空
  }
}

% 需要的宏包可以自行调用
\usepackage{mwe}
\usepackage{graphicx}
\usepackage{float}
\usepackage{subfig}
\usepackage{pifont}

\newcommand\px{\mathrel{/\mkern-5mu/}}  %平行
\newcommand\pxeq{\mathrel{\vcenter{     %平行且等于
\ialign{\hfil##\hfil\crcr
$\scriptstyle\px\!$\crcr
\noalign{\nointerlineskip\vskip1pt}$=$\crcr}}}}
\newcommand\backcong{\mathrel{\reflectbox{$\cong$}}}
\begin{document}

% 设置前置部分编号
\frontmatter

% 中文摘要环境


% 设置正文编号
\mainmatter

\chapter{二次根式}
\chapter{全等三角形}
\chapter{反比例函数}
\section{反比例函数与一次函数}
\section{反比例函数与四边形}
\chapter{一次函数}
\section{斜率与截距}
\begin{definition}
    形如$y=kx+b(k\ne 0)$的函数为一次函数。\\
    $b=0$时为正比例函数,正比例函数也是一次函数。
\end{definition}
\begin{knowledge}
    直线与$x$轴交点为$(-\frac{b}{k},0)$,与$y$轴交点为$(0,b)$
\end{knowledge}
\begin{knowledge}
    $k=\tan \alpha,\alpha$为直线与$x$轴正半轴的夹角。
\end{knowledge}
\begin{knowledge}
    $k=\frac{y_1-y_2}{x_1-x_2}$,$(x_1,y_1),(x_2,y_2)$为直线上两个不同的点。
\end{knowledge}
\begin{knowledge}
    $|k|$越大,直线越陡峭,越接近$y$轴;$|k|$越小,直线越平缓,越接近$x$轴。
\end{knowledge}
\begin{knowledge}
    $k>0$,$y$随着$x$增大而增大;$k<0$,$y$随着$x$增大而减小。
\end{knowledge}
\begin{knowledge}
    若$x$取值范围为$x_1<x<x_2$ \\
    $k>0$时,$f(x_1)$为最小值,$f(x_2)$为最大值。 \\
    $k<0$时,$f(x_1)$为最大值,$f(x_2)$为最小值。
\end{knowledge}
\begin{corollary}
    $k=1$时,直线与$x$轴正半轴夹角为$45^\circ$ \\
    $k=-1$时,直线与$x$轴正半轴夹角为$135^\circ$ \\
    $k=\frac{\sqrt3}{3}$时,直线与$x$轴正半轴夹角为$30^\circ$ \\
    $k=-\frac{\sqrt3}{3}$时,直线与$x$轴正半轴夹角为$150^\circ$ \\
    $k=\sqrt3$时,直线与$x$轴正半轴夹角为$60^\circ$ \\
    $k=-\sqrt3$时,直线与$x$轴正半轴夹角为$120^\circ$
\end{corollary}
\begin{corollary}
    $k=0\Leftrightarrow$直线垂直于$y$轴。$k\to \infty$ 不存在$\Leftrightarrow$直线垂直于$x$轴。
\end{corollary}
\begin{table}[ht]
\centering
\caption{一次函数过的象限}
\begin{tabular}{l|l|l}
 & $b>0$ & $b<0$ \\
\hline
$k>0$ & 一三二 & 一三四 \\
\hline
$k<0$ & 二四一 & 二四三 \\
\hline
\end{tabular}
\end{table}
\begin{problem}
    一次函数$y=kx+b$在定义域$0\le x \le 2$内的最大值为$4$,最小值为$−2$,求一次函数的解析式。
\end{problem}
\begin{problem}
    直线$y=kx+b$不经过第二象限,求$k$和$b$的取值范围。
\end{problem}
\clearpage
\section{平移与对称}
\begin{conclusion}
    将一次函数$y=kx+b$沿着某个方向平移$a(a>0)$个单位。 \\
    向上平移后的解析式为$y=kx+b+a$ \\
    向下平移后的解析式为$y=kx+b-a$ \\
    向左平移后的解析式为$y=k(x+\frac{b}{k}+a)$,即$y=kx+b+ak$ \\
    向右平移后的解析式为$y=k(x+\frac{b}{k}-a)$,即$y=kx+b-ak$ \\
    口诀:左加右减,上加下减。
\end{conclusion}
\begin{conclusion}
    将一次函数$y=kx+b$沿着某个方向作对称变换。\\
    沿着$x$轴对称后的解析式为$y=-kx-b$ \\
    沿着$y$轴对称后的解析式为$y=-kx+b$ \\
    关于原点对称后的解析式为$y=kx-b$
\end{conclusion}
\begin{problem}
    点$C$为直线$y=x$上与原点不重合的点,直线$y=2x+1$与$x$轴和$y$轴分别交于$A,B$两点,
    将直线沿着射线$OC$方向平移$3\sqrt2$个单位,求平移后直线的解析式。\\
    \\
\end{problem}
\begin{problem}
    将一次函数$y=-3x+4$向下平移$1$个单位相当于向\underline{\quad}(填左或右)平移\underline{\quad}个单位。
\end{problem}
\clearpage

\section{垂直直线的斜率关系}
\begin{conclusion}
    若直线$y=k_1x+b_1$与直线$y=k_2x+b_2$相垂直,则有$k_1·k_2=-1$
\end{conclusion}
\begin{example}
    已知坐标系内两点$A(x_1,y_1),B(x_2,y_2)$,求线段$AB$垂直平分线的解析式。\\
    设线段$AB$垂直平分线的解析式为$y=kx+b$,则有: \\
    $\begin{cases}k·\frac{y_1-y_2}{x_1-x_2}=-1 \\ \frac{y_1+y_2}{2}=k·\frac{x_1+x_2}{2}+b \\ \end{cases}$
\end{example}
\begin{corollary}
    \quad \\
    点$(a,b)$关于直线$y=x$的对称点为$(b,a)$ \\
    点$(a,b)$关于直线$y=-x$的对称点为$(-b,-a)$
\end{corollary}
\begin{problem}
    在平面直角坐标系中,定义点$(-b,-a)$为点$(a,b)$的关联点。若一个点和它的关联点在同一个象限,判断该点所在的象限。
\end{problem}
\begin{problem}
    在平面直角坐标系中,$A(0,3),B(4,0)$,点$C$在坐标轴上且满足$CA=CB$,求$C$点坐标。 \\
    \\
\end{problem}
\begin{problem}
    若直线$y_1=3x+2$与直线$y_2=kx+b$关于直线$y=x$对称,求直线$y_2$的解析式。 \\
    \\
\end{problem}

\clearpage
\section{直线交点与方程解的关系}
\begin{knowledge}
    已知直线$y=k_1x+b_1$和直线$y=k_2x+b_2$ \\
    若$k_1\ne k_2$,两条直线有唯一交点。 \\
    若$k_1 = k_2, b_1 \ne b_2$,两条直线平行,没有交点。 \\
    若$k_1 = k_2, b_1 = b_2$,两条直线重合,有无数个交点。
\end{knowledge}
\begin{knowledge}
    直线$y=k_1x+b_1$与直线$y=k_2x+b_2$的交点$(x_0,y_0)$是方程组 \\
    $\begin{cases} k_1x-y+b_1=0  \\ k_2x-y+b_2=0  \\ \end{cases}$的解。\\
    两条直线有唯一交点 $\Leftrightarrow$ 方程组有唯一解。 \\
    两条直线平行 $\Leftrightarrow$ 方程组无解。 \\
    两条直线重合 $\Leftrightarrow$ 方程组有无数解。
\end{knowledge}
\begin{problem}
    设$b>a$,在同一直角坐标系下画出一次函数$y=bx+a$和一次函数$y=ax+b$的图像。\\
\end{problem}
\begin{problem}
    直线$y=3x-1$与直线$y=x-k$的交点在第四象限,求$k$的取值范围。\\
\end{problem}
\clearpage
\section{一次函数与不等式}
\begin{knowledge}
    若一次函数$y=kx+b$与$x$轴交点为$(x_0,0)$,则有: \\
    方程$kx+b=0$的解为$x=x_0$ \\
    若$k>0$,不等式$kx+b>0$的解集为$x>x_0$,不等式$kx+b<0$的解集为$x<x_0$ \\
    若$k<0$,不等式$kx+b>0$的解集为$x<x_0$,不等式$kx+b<0$的解集为$x>x_0$
\end{knowledge}
\begin{knowledge}
    若一次函数$y=k_1x+b_1$与一次函数$y=k_2x+b_2$相交于$(x_0,y_0)$,且$k_1>k_2$,则有: \\
    $k_1x+b_1>k_2x+b_2$的解集为$x>x_0$ \\
    $k_1x+b_1<k_2x+b_2$的解集为$x<x_0$
\end{knowledge}
\begin{knowledge}
    如下图,若一次函数$y=k_1x+b_1$反比例函数$y=\frac{k_2}{x}$两支相交于$N(x_1,y_1),M(x_2,y_2)$两点,则有:\\
    $k_1x+b_1>\frac{k_2}{x} \Leftrightarrow x_1<x<0$或$x>x_2$\\
    $k_1x+b_1<\frac{k_2}{x} \Leftrightarrow x<x_1$或$0<x<x_2$
\end{knowledge}
\begin{figure}[ht]
\centering
\begin{minipage}[t]{0.48\textwidth}
\centering
\includegraphics[width=4cm]{picture/4_5_1.png}
\caption{一次函数与反比例函数}
\end{minipage}
\begin{minipage}[t]{0.48\textwidth}
\centering
\includegraphics[width=4cm]{picture/4_5_2.png}
\caption{习题4.5.1图}
\end{minipage}
\end{figure}
\begin{problem}
    如图,一次函数$y=kx+b$与$x$轴交于$(2,0)$,求不等式$kx+b>3k$的解集。 \\
\end{problem}
\begin{problem}
    一次函数$y=k_1x+b_1$和一次函数$y_2=k_2x+b_2$满足无论$x$取何值,$y_1>y_2$恒成立,判断$k_1$与$k_2$,$b_1$与$b_2$的关系。\\
\end{problem}
\begin{problem}
    一次函数$y_1=3x+2$和一次函数$y_2=\frac{1}{3}x-\frac{2}{3}$满足$x>m$时,$y_1>y_2$恒成立,求$m$的取值范围。\\
\end{problem}
\begin{problem}
    已知$f(x)=x-1,g(x)=\frac{2}{x}$,定义$h(x)=\max(f(x),g(x))$,
    其中$\max(a,b)=\begin{cases} a & \text{$a \ge b$} \\ b & \text {$a<b$}\end{cases}$,即取两者的最大值。
    求$h(x)$的最小值。
\end{problem}
\clearpage
\section{一次函数中的坐标轴问题}
\begin{problem}
    直线$y=kx+5$与坐标轴围成的三角形的周长为30,求$k$的值。\\
    \\
\end{problem}
\begin{problem}
    直线$y=kx-3$与坐标轴围成的三角形的面积为6,求$k$的值。\\
    \\
\end{problem}
\begin{problem}
    直线$y=kx+b$图像经过点$P(3,2)$且与$x$轴正半轴和$y$轴正半轴分别交于$A,B$两点,且$OA+OB=12$,求一次函数的解析式。\\
    \\
\end{problem}
\clearpage
\section{一次函数中的面积问题}
\begin{knowledge}
    如图,$\triangle ABC$面积可以表示为$\frac{1}{2}$水平宽·铅垂高。\\
    设$A(x_1,y_1),B(x_2,y_2),C(x_3,y_3)$ \\
    根据$A,B$坐标算出直线的解析式,求出$D$点坐标$(x_4,y_4)$ \\
    $S_{\triangle ABC}=\frac{1}{2}|x_2-x_1||y_3-y_4|$ \\
    若$C$点恰好在坐标原点,铅垂高即为一次函数的截距取绝对值。
\end{knowledge}
\begin{figure}[ht]
\centering
\begin{minipage}[t]{0.48\textwidth}
\centering
\includegraphics[width=4cm]{picture/4_7_1.png}
\caption{铅垂法}
\end{minipage}
\hfill
\begin{minipage}[t]{0.48\textwidth}
\centering
\includegraphics[width=4cm]{picture/4_7_2.png}
\caption{有一个顶点在原点的三角形}
\end{minipage}
\end{figure}
\begin{problem}
    //todo
\end{problem}
\clearpage
\section{一次函数过定点问题}
\begin{example}
    求一次函数$y=(2-k)x+3k$过的定点。 \\
    一次函数过定点,说明存在一组点$(x,y)$满足无论$k$取何值,等式$y=(2-k)x+3k$恒成立。 \\
    说明$k$的所有系数为零。 \\
    $(3-x)k+2x-y=0$ \\
    $\begin{cases} 3-x=0 \\ 2x-y=0 \\ \end{cases} \Rightarrow x=3,y=6$ \\
    所以一次函数过定点(3,6)
\end{example}
\begin{problem}
    若$(2k−1)x−(k−3)y−(k−4)=0$是$y$关于$x$的一次函数,求$k$的取值范围,并求出当$k$满足取值范围时,一次函数恒过的定点。\\
    \\
\end{problem}
\begin{problem}
    平面直角坐标系有两点$A(4,5),B(6,9)$,若一次函数$y=kx−2k+3$与线段$AB$有交点,求$k$的取值范围。\\
    \\
\end{problem}
\clearpage
\section{一次函数与将军饮马问题}
\begin{knowledge}
    如图,在直线$l$上找一点$P$,使得以下结论成立。 \\
    $PB+PA$最小,作$B$关于$l$的对称点$B'$,$P$为$AB'$与$l$的交点。最小值为$AB'$\\
    $|PA-PB|$最小,$P$为线段$AB$垂直平分线与$l$的交点。最小值为0 \\
    $|PA-PB|$最大,$P$为$AB$延长线与$l$的交点。最大值为$AB$ \\
\end{knowledge}
\begin{figure}[ht]
\includegraphics[width=4cm]{picture/4_9_1.png}
\caption{将军饮马}
\end{figure}
\begin{problem}
    已知平面直角坐标系两点$A(2,3),B(4,5)$,$C$在$x$轴上,求满足下列条件的$C$的坐标以及相应的最值。\\
    (1)$CA+CB$最小 \\
    \\
    (2)$|CA-CB|$最小 \\
    \\
    (3)$|CA-CB|$最大 \\
\end{problem}
\begin{problem}
    求根式$\sqrt{x^2-4x+13}+\sqrt{x^2-8x+41}$的最小值,并求出取最小值时$x$的取值。 \\
\end{problem}
\begin{problem}
    求根式$\sqrt{x^2-8x+41}-\sqrt{x^2-4x+13}$的最大值,并求出取最大值时$x$的取值。 \\
\end{problem}
\begin{problem}
    已知平面直角坐标系两点$A(4,6),B(6,4)$,$C$在$y$轴上,$D$在$x$轴上,求使得四边形$ABCD$周长最小的$D$点坐标。\\
\end{problem}

\clearpage
\section{一次函数夹角问题}
\clearpage
\section{一次函数几何综合}
\begin{problem}
    已知一次函数$y=-2x+4$的图像与$x$轴、$y$轴分别交于点$B,A$,以$AB$为边在第一象限内作等腰直角三角形$ABC$,且$\angle ABC=90^\circ$,
    $BA=BC$,作$OB$的垂直平分线$l$交直线$AB$于点$E$,交$x$轴于点$G$. \\
    (1)求点$C$的坐标。
    (2)在$OB$垂直平分线$l$上有一点$M$,且点$M$与点$C$位于直线$AB$的同侧,使得$2S_{\triangle ABM}=S_{\triangle ABC}$,求点$M$的坐标。\\
    (3)在(2)的条件下,联结$CE,CM$, 判断$\triangle CEM$的形状,并给予证明。
\end{problem}
\begin{figure}[H]
\centering
\includegraphics[width=4cm]{picture/401.png}
\caption{练习图}
\end{figure}
\clearpage
\section{一次函数图像分析}
\begin{problem}
    甲、乙两人在笔直的湖边公路上同起点、同终点、同方向匀速步行2400米,先到终点的人原地休息。
    已知甲先出发4分钟,在整个步行过程中,甲、乙两人间的距离(米)与甲出发的时间(分)之间的关系如图中折线
    $OA-AB-BC-CD$所示。\\
    (1)求线段$AB$的表达式,并写出自变量$x$的取值范围。 \\
    (2)求乙的步行速度。 \\
    (3)求点$C,D$的坐标。
\end{problem}
\begin{figure}[H]
\centering
\includegraphics[width=6cm]{picture/6122.png}
\caption{练习图}
\end{figure}
\clearpage
\section{一次函数与实际应用}
\begin{problem}
某自来水公司每月用水收费标准如下:每月固定水处理费$a$元,若当月用水量未超过$b$升,每升水价格为$c$元。
若超过$b$升,超过的部分每升水的单价为$1.5c$元。已知当月有三户人家用水量与计价如下表所示,求$a,b,c$的值。
\end{problem}
\begin{table}[ht]
\centering
\caption{用户用水量}
\begin{tabular}{|l|l|l|}
\hline
用户 & 用水量(升) & 价格(元) \\
\hline
A & 6 & 16 \\
\hline
B & 8 & 20 \\
\hline
C & 16 & 42 \\
\hline
\end{tabular}
\end{table}
\clearpage
\section{点到直线距离公式}
\clearpage
\chapter{代数方程}
\section{一元一次方程与不等式}
\clearpage
\section{一元二次方程与不等式}
\clearpage
\section{分式方程}
\clearpage
\section{无理方程}
\clearpage
\section{根与增根问题}
\begin{knowledge}
    分式方程的增根是去分母后所得整式方程的根,且使得分母为零。\\
    无理方程的增根可能使得根号为负,或者不满足无理方程本身。
\end{knowledge}
\begin{problem}
    分式方程$\frac{6}{(x-1)(x+1)}-\frac{3m}{x-1}-1=0$有增根$x=1$,求$m$的值。\\
\end{problem}
\begin{problem}
    分式方程$\frac{4x}{4-x^2}+1=\frac{k-k^2}{x-2}+\frac{1}{x+2}$不会出现增根,求$k$的取值范围。 \\
\end{problem}
\begin{problem}
    分式方程$\frac{2}{x}-\frac{x-m}{x^2-x}=1+\frac{1}{x-1}$在实数范围内无解,求$m$的值。\\
\end{problem}
\begin{problem}
    无理方程$\sqrt{2x-4}-\sqrt{x+a}=1$有一个增根为4,求$a$的值。\\
\end{problem}
\begin{problem}
    无理方程$\sqrt{4-2x}-kx+2=0$有实数根,求$k$的取值范围。\\
\end{problem}
\begin{problem}
    无理方程$\sqrt{x^2+2x+3}+k=0$有实数根,求$k$的取值范围。\\
\end{problem}
\begin{problem}
    无理方程$\sqrt{x+3}+2x+m=0$只有一个实数根,求$m$的取值范围。\\
\end{problem}
\clearpage
\section{高次方程组的解法}
\clearpage
\section{方程组}
\clearpage
\section{应用}
\begin{problem}
    有一项工程,甲单独做比甲乙合做完工的天数多5天,若甲乙先合做4天,再由乙单独做3天,则能完成全部工程的一半,求甲、乙单独完成此项工程各需几天。
\end{problem}
\clearpage
\chapter{四边形}
\section{多边形}
\begin{conclusion}
    对于$n$边形,有如下结论成立: \\
    内角和为$(n-2)·180^{\circ}$ \\
    外角和为$360^{\circ}$ \\
    最多有3个锐角 \\
    一共有$\frac{n(n-3)}{2}$条对角线 \\
    从一个顶点出发有$n-3$条对角线,将$n$边形分成$n-2$个三角形 \\
\end{conclusion}
\begin{problem}
    $n$边形每增加一条边,会增加多少条对角线? \\
\end{problem}
\begin{problem}
    $m$边形从一个顶点出发沿着对角线将$m$边形分成了7个三角形,$n$边形没有对角线,$k$边形有$k$条对角线,求$(m-k)^n$的值。\\
\end{problem}
\begin{problem}
    如图,正三角形、正方形、正五边形按如图所示位置摆放,求$\angle 1 +\angle 2 + \angle 3$的度数。
\end{problem}
\begin{figure}[H]
\centering
\includegraphics[width=3cm]{picture/6_1_2.png}
\caption{正三角形、正方形与正五边形}
\end{figure}
\begin{problem}
    如图,六边形六个内角都为$120^\circ$,求六边形的周长。
\end{problem}
\begin{figure}[H]
\centering
\includegraphics[width=3cm]{picture/647.png}
\caption{六边形}
\end{figure}
\begin{problem}
    定义有三个角相等的四边形为三等角四边形。\\
    (1)在三等角四边形$ABCD$中,$\angle A=\angle B=\angle C$,求$\angle A$的取值范围。 \\
    \\
    (2)如图1,折叠平行四边形$DEBF$使得点$E,F$分别落在边$BE,BF$上的点$A,C$处,折痕为$DG,DH$,求证:四边形$ABCD$为三等角四边形。\\
    \\
    (3)如图2,三等角四边形$ABCD$中,$\angle A=\angle B=\angle C$,$AB=5,AD=\sqrt{26},DC=7$,求$BC$的长度。 \\
    \\
\end{problem}
\begin{figure}[H]
\centering
\includegraphics[width=8cm]{picture/6_1_1.png}
\caption{三等角四边形}
\end{figure}
\begin{problem}
    定义至少有一组对边相等的四边形叫等对边四边形。\\
    (1)如图,在$\triangle ABC$中,点$D,E$分别在$AB,AC$上,$CD,BE$交于点$O$,
    若$\angle A=60^\circ,\angle DCB=\angle EBC=\frac{1}{2}\angle A$,写出图中一个与$\angle A$相等的角,并找出一个等对边四边形并证明。\\
    (2)若$\angle A$为不等于$60^\circ$的锐角,点$D,E$分别在$AB,AC$上,$\angle DCB=\angle EBC=\frac{1}{2}\angle A$,判断(1)中的等对边四边形是否依然成立,并证明。\\
\end{problem}
\begin{figure}[H]
\centering
\includegraphics[width=4cm]{picture/6125.png}
\caption{三等角四边形}
\end{figure}
\clearpage
\section{平行四边形基础}
\begin{table}[H]
\centering
\caption{平行四边形的定义、判定与性质}
\begin{tabular}{|l|l|}
\hline
    & 描述 \\
\hline
定义 & 两组对边互相平行的四边形是平行四边形 \\
\hline
判定 & 两组对边分别相等的四边形是平行四边形 \\
\hline
判定 & 一组对边平行且相等的四边形是平行四边形 \\
\hline
判定 & 两组对角分别相等的四边形是平行四边形 \\
\hline
判定 & 对角线互相平分的四边形是平行四边形 \\
\hline
性质 & 平行四边形对边相等 \\
\hline
性质 & 平行四边形对角相等 \\
\hline
性质 & 平行四边形的两条对角线互相平分 \\
\hline
性质 & 平行四边形是中心对称图形,对称中心是两条对角线的交点 \\
\hline
推论 & 夹在两条平行线间的平行线段相等 \\
\hline
\end{tabular}
\end{table}
\begin{problem}
    已知平行四边形两条对角线分别为$m,n$,求平行四边形边长的取值范围。
\end{problem}
\begin{problem}
    已知平行四边形边长和一条对角线分别为$m,n$,求平行四边形另一条对角线的取值范围。
\end{problem}
\begin{problem}
    如图,平行四边形$ABCD,AE\perp BC$于$E$,$AE,BD$相交于$G$,且$DG=2AB,\angle DBC=25^\circ$,求$\angle C$的度数。
\end{problem}
\begin{figure}[H]
\centering
\includegraphics[width=4cm]{picture/603.png}
\caption{练习图}
\end{figure}
\begin{problem}
    如图,在平行四边形$ABCD$中,以$BE$为折痕将$\triangle ABC$向上翻折使得$A$落在$CD$上。
    若$\triangle FDE$周长为8,$\triangle FCB$周长为22,求$FC$的长。
\end{problem}
\begin{problem}
    如图,在等边$\triangle ABC$中,$AB=8$,点$D$在$BC$上,$\triangle ADE$为等边三角形,点$E,D$在直线$AC$两侧。
    过$E$点作$EF\px\px BC$,$EF$与$AB,AC$分别交于$F,G$ \\
    (1)求证:$BF=CE$ \\
    (2)若$AD=7$,求$FG$的长。\\
\end{problem}
\begin{figure}[ht]
\begin{minipage}[t]{0.48\textwidth}
\centering
\includegraphics[width=4cm]{picture/6_2_1.png}
\caption{练习图}
\end{minipage}
\begin{minipage}[t]{0.48\textwidth}
\centering
\includegraphics[width=4cm]{picture/6_2_2.png}
\caption{练习图}
\end{minipage}
\end{figure}
\begin{problem}
    如图,$\triangle ABC$分别以$AB,AC,BC$为边在$BC$同侧作三个等边三角形,探究$\triangle ABC$满足什么条件时,以下结论成立:\\
    (1)四边形$DAEF$为矩形 \\
    (2)四边形$DAEF$为菱形 \\
    (3)以$D,A,E,F$为顶点的四边形不存在
\end{problem}
\begin{figure}[H]
\centering
\includegraphics[width=4cm]{picture/671.png}
\caption{练习图}
\end{figure}
\begin{problem}
    如图,菱形$ABCD$中,点$E,F$分别在边$BC,CD$上,$\angle BAF=\angle DAE$,$AE,BD$交于$G$. \\
    (1)求证:$EF\px \px BD$. \\
    (2)当$BE=GF$时,求证:四边形$BEFG$为平行四边形. \\
\end{problem}
\begin{figure}[H]
\centering
\includegraphics[width=4cm]{picture/6127.png}
\caption{练习图}
\end{figure}
\begin{problem}
    如图,平行四边形$ABCD,AB=5,BC=4,AC=\sqrt{17},\triangle ABC$的平分线交$CD$于$E$,交$AC$于$F$. \\
    (1)求平行四边形$ABCD$面积。\\
    (2)若$P$为$BC$上一动点(不与$B,C$重合),连$EP$,设$BP=x,S_{\triangle PFC}=y$,求$y$关于$x$的函数解析式,并写出定义域。\\
\end{problem}
\begin{figure}[H]
\centering
\includegraphics[width=4cm]{picture/6130.png}
\caption{练习图}
\end{figure}
\clearpage
\section{平行四边形的坐标}
\begin{conclusion}
    在平行四边形$ABCD$中,$A(x_1,y_1),B(x_2,y_2),C(x_3,y_3),D(x_4,y_4)$,则有如下结论成立: \\
    (1)对称中心$O$点坐标为$(\frac{x_1+x_3}{2},\frac{y_1+y_3}{2})$或$(\frac{x_2+x_4}{2},\frac{y_2+y_4}{2})$ \\
    (2)$x_1+x_3=x_2+x_4,y_1+y_3=y_2+y_4$ \\
\end{conclusion}
\begin{problem}
    如图,在菱形$ABCD$中,$A(0,4),C(0,-12),F(3,0)$,求$B$点坐标。
\end{problem}
\begin{figure}[H]
\centering
\includegraphics[width=4cm]{picture/646.png}
\caption{练习图}
\end{figure}
\begin{problem}
    如图,直线$l_1:y_1=-3x+3$与$x$轴交于$D$点,另一直线$l_2:y_2=kx+b$经过$A,B$两点,与$l_1$交于点$D$。
    直接写出$E$点坐标,使得以$E,A,C,D$为顶点的四边形为平行四边形。
\end{problem}
\begin{figure}[H]
\centering
\includegraphics[width=4cm]{picture/667.png}
\caption{练习图}
\end{figure}
\begin{problem}
    如图,四边形$ABCD$为平行四边形,$BC=2AB,A(-1,0),B(0,2),C,D$在反比例函数$y=\frac{k}{x}(k<0)$上,求$k$的值。
\end{problem}
\begin{figure}[H]
\centering
\includegraphics[width=4cm]{picture/670.png}
\caption{练习图}
\end{figure}
\begin{problem}
    如图,直线$y=mx+4$与反比例函数$y=\frac{k}{x}(k>0)$图像交于$A,B$两点,与$x$轴负半轴,$y$轴交于$D,C$两点,$CO:DO=2$,$D$,$A$点横坐标为1.
    点$M$在直线$x=-1$上,点$N$在反比例函数上,若以$A,B,M,N$为顶点的四边形是平行四边形,求点$N$的坐标。
\end{problem}
\begin{figure}[H]
\centering
\includegraphics[width=4cm]{picture/6123.png}
\caption{练习图}
\end{figure}
\clearpage
\section{平行四边形的对称中心}
\begin{model}
    如图,平行四边形$ABCD$中,$O$为两条对角线交点,$EF$过点$O$分别交$AD,BC$于$E,F$,有如下结论成立:\\
    (1)$EF$平分$ABCD$周长和面积 \\
    (2)$OF=OE$ \\
    (3)$S_{\triangle ABO}=S_{\triangle ADO}=S_{\triangle BCO}=S_{\triangle CDO}$ \\
    (4)若一条直线平分平行四边形面积,则该直线一定经过平行四边形对称中心
\end{model}
\begin{figure}[H]
\centering
\includegraphics[width=4cm]{picture/668.png}
\caption{练习图}
\end{figure}
\begin{problem}
    如图,$AB\px\px DC \px\px EF,AD\px\px CF\px\px BE$,请画出一条直线,平分该多边形的面积。
\end{problem}
\begin{figure}[H]
\centering
\includegraphics[width=4cm]{picture/6_2_3.png}
\caption{练习图}
\end{figure}
\begin{problem}
    如图,平行四边形$ABCO,A(5,0),C(1,4)$,过点$P(0,-2)$的直线交平行四边形于$M,N$两点,且将平行四边形分成面积相等的两部分,求$MN$的长度。
\end{problem}
\begin{figure}[H]
\centering
\includegraphics[width=4cm]{picture/669.png}
\caption{练习图}
\end{figure}
\clearpage
\section{面积问题}
\begin{model}
    与平行四边形等底等高的三角形面积是平行四边形的一半。\\
    如图,可以分析出$S_1=S_2$
\end{model}
\begin{figure}[H]
\centering
\includegraphics[width=3cm]{picture/680.png}
\caption{等面积模型}
\end{figure}
\begin{model}
    如图,平行四边形$ABCD$,$P$为平行四边形内一点,则有$S_{\triangle ABP}+S_{\triangle CDP}=S_{\triangle ADP}+S_{\triangle BCP}.$
\end{model}
\begin{figure}[H]
\centering
\includegraphics[width=4cm]{picture/6115.png}
\caption{等面积模型}
\end{figure}
\begin{problem}
    如图,在平行四边形$ABCD$中,$E$为$AD$上一点,$F$为$AB$上一点。$BE,DF$交于点$G$且$BE=DF$,连$GC$,求证:$GC$平分$\angle BGD$
\end{problem}
\begin{figure}[H]
\centering
\includegraphics[width=3cm]{picture/681.png}
\caption{练习图}
\end{figure}
\begin{problem}
    如图,在平行四边形$𝐴𝐵𝐶𝐷$中,$𝐸,F,𝑄,𝑃$分别为$𝐴𝐷,𝐷𝐶,𝐶𝐵,𝐵𝐴$的中点,若平行四边形$𝐴𝐵𝐶𝐷$面积为4,求$\triangle𝑃𝑄𝑇$的面积。
\end{problem}
\begin{figure}[H]
\centering
\includegraphics[width=3cm]{picture/682.png}
\caption{练习图}
\end{figure}
\begin{problem}
    如图,平行四边形$𝐴𝐵𝐶𝐷$的面积为64,$E,F$分别为$AB,AC$中点,求$\triangle CEF$的面积。
\end{problem}
\begin{figure}[H]
\centering
\includegraphics[width=3cm]{picture/683.png}
\caption{练习图}
\end{figure}
\begin{problem}
    如图,过平行四边形$𝐴𝐵𝐶𝐷$内一点$P$作边的平行线$EF,GH$,若阴影部分面积为$8$,则平行四边形$PHCF$面积比平行四边形$PGAE$大多少?
\end{problem}
\begin{figure}[H]
\centering
\includegraphics[width=4cm]{picture/684.png}
\caption{练习图}
\end{figure}
\begin{problem}
    如图,平行四边形$ABCD$的面积为120,$E$为$AB$上一点,且$AE=2BE$,$DE,CB$延长线交于$F$,求$\triangle AEF$与$\triangle BCE$的面积和。
\end{problem}
\begin{figure}[H]
\centering
\includegraphics[width=4cm]{picture/687.png}
\caption{练习图}
\end{figure}
\clearpage
\section{矩形基础}
\begin{knowledge}
    如图,矩形$ABCD$,对角线$AC,BD$交于点$O$,则有如下结论成立:\\
    (1)$OA=OB=OC=OD$ \\
    (2)若$\angle ACB=30^\circ$,则$\triangle ABO,\triangle CDO$为等边三角形 \\
\end{knowledge}
\begin{figure}[H]
\centering
\includegraphics[width=4cm]{picture/653.png}
\caption{矩形}
\end{figure}
\begin{table}[H]
\centering
\caption{矩形的定义、判定与性质}
\begin{tabular}{|l|l|}
\hline
    & 描述 \\
\hline
定义 & 有一个角是直角的平行四边形是矩形 \\
\hline
判定 & 有三个角是直角的四边形是矩形 \\
\hline
判定 & 对角线相等的平行四边形是矩形 \\
\hline
性质 & 矩形四个角都是直角 \\
\hline
性质 & 矩形对角线相等 \\
\hline
对称性 & 矩形是轴对称图形,2条对称轴,对称轴为对边中点连线 \\
\hline
\end{tabular}
\end{table}
\begin{problem}
    如图,矩形$ABCD$中,$AB=2AD$,$E$是$CD$的中点,且$AB=AF$,求$\angle EBF$.
\end{problem}
\begin{figure}[H]
\centering
\includegraphics[width=4cm]{picture/688.png}
\caption{练习图}
\end{figure}
\begin{problem}
    如图,点$E,F$分别是平行四边形$ABCD$边$AD,BC$的中点,$AD=2AB$,连$AC,BE$交于$G$,$DF,CE$交于$H$,求证:四边形$EGFH$为矩形。
\end{problem}
\begin{figure}[H]
\centering
\includegraphics[width=4cm]{picture/605.jpeg}
\caption{练习图}
\end{figure}
\begin{problem}
    如图,$P$为$\triangle ABC$边$BC$上一动点,$MN\px\px BC$交$\angle BCA$平分线于点$E$,交$\angle BCA$外角平分线于点$F$。\\
    (1)求证:$PE=PF$ \\
    (2)当$P$运动到何处时,四边形$AECF$为矩形? \\
    (3)若四边形$AECF$为正方形,且$\frac{AP}{BC}=\frac{\sqrt{3}}{2}$,求$\angle BAC$ \\
\end{problem}
\begin{figure}[H]
\centering
\includegraphics[width=4cm]{picture/654.png}
\caption{练习图}
\end{figure}
\begin{problem}
    如图,等腰梯形$ABCD$中,$AB\px \px CD,AD=BC$,对角线$AC,BD$交于$O$,$E,F$分别在$OA,OB$上,$OC=OE,OD=OF$,求证:四边形$DEFC$为矩形。
\end{problem}
\begin{figure}[H]
\centering
\includegraphics[width=4cm]{picture/6_3_2.png}
\caption{练习图}
\end{figure}
\begin{problem}
    如图,点$P$是平行四边形$ABCD$外一点。\\
    (1)若四边形$ABCD$为矩形,$PA\perp PC$,求证:$PB\perp PD$。 \\
    (2)若$PA \perp PC,PB \perp PD$,求证:四边形$ABCD$为矩形。 \\
\end{problem}
\begin{figure}[H]
\centering
\includegraphics[width=4cm]{picture/639.png}
\caption{练习图}
\end{figure}
\begin{problem}
    如图,$BE,BD$分别是$\angle ABC$的内外角平分线,$AD\perp BD,AE\perp BE$交$BC$延长线于$F$,求证:$DE=BF$。 \\
\end{problem}
\begin{figure}[H]
\centering
\includegraphics[width=4cm]{picture/641.png}
\caption{练习图}
\end{figure}
\clearpage
\section{腰双高模型}
\begin{problem}
    如图,$\triangle ABC,\angle C=90^\circ,PE\perp BD,PF\perp AD,BD=AD$,求证:$PE+PF=BC$
\end{problem}
\begin{figure}[H]
\centering
\includegraphics[width=3cm]{picture/655.png}
\caption{练习图}
\end{figure}
\begin{problem}
    如图,矩形$ABCD$中,将$\triangle BCD$沿着$BD$翻折,$AB=2,PN\perp BE,PM\perp AD$,求$PN+PM$的值。
\end{problem}
\begin{figure}[H]
\centering
\includegraphics[width=3cm]{picture/656.png}
\caption{练习图}
\end{figure}
\begin{problem}
    如图,矩形$ABCD$中,$AB=3,BC=4,PE\perp AC, PF\perp BD$,求$PE+PF$的值。
\end{problem}
\begin{figure}[H]
\centering
\includegraphics[width=3cm]{picture/657.png}
\caption{练习图}
\end{figure}
\begin{problem}
    如图,正方形$ABCD$边长为1,$BE=BC,PM\perp BE, PN\perp BC$,求$PM+PN$的值。
\end{problem}
\begin{figure}[H]
\centering
\includegraphics[width=3cm]{picture/658.png}
\caption{练习图}
\end{figure}
\clearpage
\section{翻折问题}
\begin{model}
    如图,将矩形$ABCD$沿着对角线$BD$翻折,$C$点翻折到$E$点,则四边形$ABDE$为等腰梯形。
\end{model}
\begin{figure}[H]
\centering
\includegraphics[width=4cm]{picture/672.png}
\caption{练习图}
\end{figure}
\begin{problem}
    如上图,若$AB=6,BC=8$. \\
    (1)求证:四边形$ABDE$为等腰梯形. \\
    (2)求四边形$BCDF$的周长. \\
    (3)求四边形$ABDE$的面积.
\end{problem}
\begin{model}
    如图,平行四边形$ABCD$,将$\triangle ABC$沿着$AC$翻折,$B$点翻折到$B'$,则$\triangle BB'D$为直角三角形。\\
\end{model}
\begin{figure}[H]
\centering
\includegraphics[width=4cm]{picture/6131.png}
\caption{模型图}
\end{figure}
\begin{problem}
    平行四边形$ABCD$,$AC,BD$交于点$O,\angle AOB=60^\circ,BD=4$,将$\triangle ABC$沿直线$AC$翻折,点$B$落在点$E$,求$S_{\triangle AED}$.
\end{problem}
\begin{problem}
    如图,矩形$ABCD$中,$E$为边$AB$的中点,将$\triangle EBC$沿着$EC$翻折,点$B$落在$P$处,连$BP$交$CE$于$Q$. \\
    (1)求证:四边形$AECF$为平行四边形。\\
    (2)若$PA=PE$,求证:$\triangle APB \backcong \triangle EPC$.
\end{problem}
\begin{figure}[H]
\centering
\includegraphics[width=4cm]{picture/6121.png}
\caption{练习图}
\end{figure}
\begin{model}
    如图,矩形$ABCD$沿着直线$EF$折叠,$B$的对称点$B'$恰好与$D$重合,则有四边形$BEDF$为菱形,四边形$AA'B'B$为等腰梯形。
\end{model}
\begin{figure}[H]
\centering
\includegraphics[width=4cm]{picture/6132.png}
\caption{模型图}
\end{figure}
\begin{problem}
    将矩形$ABCD$沿某条直线折叠,使得对角线两个端点$A,C$重合,折叠所在直线交射线$AB$于点$E$,若$AB=3,BE=1$,求$BC$的长。
\end{problem}
\begin{problem}
    矩形$ABCD$中,$AB=6,AD=8$,$E$为$BC$上的点,以$AE$为折痕折叠,$B$落在$F$。连接$FC$,若$\triangle EFC$为直角三角形,求$BE$的长度。
\end{problem}
\begin{problem}
    如图,矩形纸片$ABCD$,$AB=3,AD=5$,翻折纸片使得$A$落在$BC$上$E$处,折痕为$PQ$。$E$在边$BC$上移动时,折痕端点$P,Q$也随之移动,
    若限定$P,Q$分别在$AB,AD$上移动,求$E$在$BC$上可移动的最大距离。
\end{problem}
\begin{figure}[H]
\centering
\includegraphics[width=4cm]{picture/676.png}
\caption{练习图}
\end{figure}
\clearpage
\section{四边形与勾股定理}
\begin{conclusion}
    平行四边形$ABCD$满足$2AB^2+2BC^2=AC^2+BD^2$.
\end{conclusion}
\begin{corollary}
    菱形两条对角线的平方和等于菱形一边长平方和的4倍。
\end{corollary}
\begin{figure}[H]
\centering
\includegraphics[width=4cm]{picture/606.png}
\caption{菱形}
\end{figure}
\begin{model}
    如图,矩形$ABCD$,$P$为平面内一点,则有结论$PA^2+PC^2=PB^2+PD^2$成立。
\end{model}
\begin{figure}[H]
\centering
\includegraphics[width=8cm]{picture/660.png}
\caption{矩形勾股定理}
\end{figure}
\begin{model}
    如图,四边形$ABCD$对角线垂直,有如下结论成立: \\
    (1)$S_{ABCD}=\frac{1}{2}AC·BD$ \\
    (2)$AB^2+CD^2=AD^2+BC^2$ \\
    (3)中点四边形$EFGH$为矩形
\end{model}
\begin{figure}[H]
\centering
\includegraphics[width=4cm]{picture/665.png}
\caption{对角线垂直的四边形}
\end{figure}
\begin{problem}
    如图,矩形$ABCD$,$P$在对角线$AC$上,过点$P$作$EF\px\px AB$,分别交$AB,CD$于点$E,F$,连$PB,PD$。
    若$PB=2\sqrt5,PD=6$,阴影部分面积为9,求矩形$ABCD$的周长。
\end{problem}
\begin{figure}[H]
\centering
\includegraphics[width=4cm]{picture/661.png}
\caption{练习图}
\end{figure}
\begin{problem}
    如图1,我们把对角线互相垂直的四边形叫做垂美四边形。 \\
    (1)图2是一个特殊的四边形,满足$AD=AB,CD=CB$,证明四边形$ABCD$为垂美四边形。\\
    (2)证明:对于任意的垂美四边形,对边的平方和相同,即$AD^2+BC^2=AB^2+CD^2$。 \\
    (3)如图3,分别以直角三角形$ABC$的直角边$AC$和斜边$AB$为边外作正方形$ACFG$和正方形$ABDE$,连接$CE,BG,GE$。若$AC=4,AB=5$,求$GE$的长度。
\end{problem}
\begin{figure}[H]
\centering
\includegraphics[width=6cm]{picture/666.png}
\caption{练习图}
\end{figure}
\clearpage
\section{菱形基础}
\begin{table}[H]
\centering
\caption{菱形的定义、判定与性质}
\begin{tabular}{|l|l|}
\hline
    & 描述 \\
\hline
定义 & 有一组邻边相等的平行四边形是菱形 \\
\hline
判定 & 四条边都相等的四边形是菱形 \\
\hline
判定 & 对角线互相垂直的平行四边形是菱形 \\
\hline
性质 & 菱形四条边都相等 \\
\hline
性质 & 菱形对角线互相垂直,每条对角线平分一组对角 \\
\hline
对称性 & 菱形是轴对称图形,2条对称轴,对称轴为对角线所在直线 \\
\hline
\end{tabular}
\end{table}
\begin{problem}
    菱形$ABCD$的边长为$a$,两条对角线的和为$b$,求菱形的面积。
\end{problem}
\begin{problem}
    将两张宽度相等的矩形纸片叠放在一起得到如图5所示的四边形$ABCD$. \\
    (1)判断四边形$ABCD$的形状,并证明。 \\
    (2)若矩形的长为8,宽为2,求四边形$ABCD$周长的最小值和最大值。 \\
\end{problem}
\begin{figure}[H]
\centering
\includegraphics[width=4cm]{picture/607.png}
\caption{练习图}
\end{figure}
\begin{problem}
    如图,在$\triangle ABC$中,$BC$上有一点$P$,过$P$分别作$AB,AC$平行线,分别交$AB,AC$于点$E,D$。作出点$P$,使得四边形$AEPD$为菱形并证明。
\end{problem}
\begin{figure}[H]
\centering
\includegraphics[width=4cm]{picture/635.jpg}
\caption{练习图}
\end{figure}
\begin{problem}
    如图,在$\triangle ABC$中,点$D,E$分别是边$AB,BC$的中点,点$F,G$是边$AC$的三等分点,$DF,EG$的延长线相交于点$H$ \\
    (1)求证:四边形$FBGH$是平行四边形 \\
    (2)若$AC$平分$\angle BAH$,求证:四边形$ABCH$为菱形 \\
\end{problem}
\begin{figure}[H]
\centering
\includegraphics[width=4cm]{picture/678.png}
\caption{练习图}
\end{figure}
\begin{problem}
    如图,在$\triangle ABC$中,$\angle ACB=90^\circ$,$CD$为$AB$边上的高,$\angle BAC$的平分线$AE$交$CD$于$F$,$EG\perp AB$于$G$,
    求证:四边形$GECF$是菱形。
\end{problem}
\begin{figure}[H]
\centering
\includegraphics[width=4cm]{picture/679.png}
\caption{练习图}
\end{figure}
\clearpage
\section{内角为$60^\circ$的菱形}
\begin{model}
    如图,菱形$ABCD$中,$\angle B=60^\circ,E,F$为$BC,CD$上的动点,且满足$\angle EAF=60^\circ$.设$AB=a$,有如下结论成立:\\
    (1)$\triangle AEB \backcong \triangle AFC \sim \triangle AGF \sim \triangle EGC$ \\
    (2)$\triangle AEC \backcong \triangle AFD \sim \triangle AGE \sim \triangle FGC$ \\
    (3)$\triangle EAF$为等边三角形 \\
    (4)四边形$AECF$的面积为$\frac{\sqrt3}{4}a^2$ \\
    (5)菱形的面积为$\frac{\sqrt3}{2}a^2$ \\
    (6)$AC=a,BD=\sqrt3a$
\end{model}
\begin{figure}[H]
\centering
\includegraphics[width=4cm]{picture/608.png}
\caption{含60度角的菱形}
\end{figure}
\begin{problem}
    已知菱形有一个内角为$60^\circ$,一条对角线长为$4\sqrt3$,求菱形的面积。
\end{problem}
\begin{problem}
    如图,菱形$ABCD,AB=6,\angle A=60^\circ$,$E$为线段$AB$上不与$A,B$重合的一点. \\
    (1)作$EDF=60^\circ$交$BC$于$F$,求证:$\triangle DEF$为等边三角形\\
    (2)在(1)的基础上,探究四边形$DEBF$周长的最小值 \\
    (3)作$DEF=60^\circ$交$BC$于$F$,(1)中结论是否依然成立?若成立请证明,不成立请说明理由 \\
\end{problem}
\begin{figure}[H]
\centering
\includegraphics[width=4cm]{picture/609.png}
\caption{练习图}
\end{figure}
\begin{problem}
    如图,菱形$ABCD$中,$AC=2,\angle B=60^\circ$.$E$为线段$BC$上一点,且不与$B,C$重合,作$\angle EAF=60^\circ$交$CD$于$F$.设$CE=x,EG=y$.\\
    (1)证明:$\triangle AEF$为等边三角形。 \\
    (2)求$y$关于$x$的函数解析式,并写出定义域。 \\
    (3)设点$O$为线段$AC$中点,若$EG=EO$,求$x$的值。 \\
\end{problem}
\begin{figure}[H]
\centering
\includegraphics[width=4cm]{picture/609.png}
\caption{练习图}
\end{figure}
\begin{problem}
    如图,菱形$ABCD$边长为2,$\angle B=60\circ$,$M$为边$AB$中点,$N$为边$BC$上一动点(不与$B$重合),
    将$\triangle BMN$沿着直线$MN$折叠,点$B$落在点$E$处,连$DE,CE$. \\
    (1)当$N$为$BC$中点时,求$CE$的长。\\
    (2)当$\triangle CDE$为等腰三角形时,求$BN$的长。
\end{problem}
\begin{figure}[H]
\centering
\includegraphics[width=6cm]{picture/6124.png}
\caption{练习图}
\end{figure}
\begin{problem}
    如图,平行四边形$ABCD$,$P$为对角线$BD$上一个动点,且不与$BD$中点重合,$PA=OC$.\\
    (1)求证:四边形$ABCD$为菱形。\\
    (2)若$AB=6,\angle ABC=60^\circ$,设$BP=x,AP=y$,求$y$关于$x$的函数解析式,并写出定义域。\\
    (3)在(2)的条件下,延长$AP$交射线$BC$于点$E$,当$\triangle EPC$为直角三角形时,求$BP$的长。 \\
\end{problem}
\begin{figure}[H]
\centering
\includegraphics[width=6cm]{picture/6128.png}
\caption{练习图}
\end{figure}
\clearpage
\section{正方形基础}
\begin{table}[H]
\centering
\caption{正方形的定义、判定与性质}
\begin{tabular}{|l|l|}
\hline
    & 描述 \\
\hline
定义 & 一组邻边相等且垂直的平行四边形是四边形 \\
\hline
判定 & 一组邻边相等的矩形是正方形 \\
\hline
判定 & 有一个角为直角的菱形是正方形 \\
\hline
性质 & 正方形四条边相等,四个角都是直角 \\
\hline
性质 & 正方形对角线相等且互相垂直平分 \\
\hline
对称性 & 正方形是轴对称图形,有4条对称轴\\
\hline
\end{tabular}
\end{table}
\begin{problem}
    如图,平行四边形$ABCD$的四条内角平分线围成四边形$GEHF$. \\
    (1)判断四边形$GEHF$的形状,并证明。\\
    (2)若四边形$ABCD$为矩形,判断四边形$GEHF$的形状,并证明。\\
\end{problem}
\begin{figure}[H]
\centering
\includegraphics[width=4cm]{picture/6120.png}
\caption{练习图}
\end{figure}
\begin{problem}
    如图,边长为6的正方形$ABCD$中有两个小正方形,求这两个小正方形的面积之和。
\end{problem}
\begin{figure}[H]
\centering
\includegraphics[width=4cm]{picture/644.png}
\caption{练习图}
\end{figure}
\begin{problem}
    如图,直角三角形$ACB,\angle C=90^\circ$,$\angle A,\angle B$的平分线交于点$D,DE\perp BC$于$E,DF\perp AC$于$F$,
    求证:四边形$CEDF$为正方形.
\end{problem}
\begin{figure}[H]
\centering
\includegraphics[width=4cm]{picture/6118.png}
\caption{练习图}
\end{figure}
\begin{problem}
    如图,点$A',B',C',D'$是正方形$ABCD$四边上的一点,满足$AA'=BB'=CC'=DD'$。\\
    (1)求证:四边形$A'B'C'D'$为正方形。 \\
    (2)探究当$A',B',C',D'$位于什么位置时,正方形$A'B'C'D'$为正方形$ABCD$面积的$\frac{5}{9}$. \\
\end{problem}
\begin{figure}[H]
\centering
\includegraphics[width=4cm]{picture/643.png}
\caption{练习图}
\end{figure}
\clearpage
\section{旋转模型}
\begin{model}
    如图,以下四个图形中,$CD=CB,\angle DAB=\alpha,\angle DCB=\beta$.\\
    (1)$\alpha=90^\circ,\beta=90^\circ\Rightarrow AD+AB=\sqrt2 AC$ \\
    (2)$\alpha=90^\circ,\beta=90^\circ\Rightarrow AB-AD=\sqrt2 AC$ \\
    (3)$\alpha=60^\circ,\beta=120^\circ\Rightarrow AD+AB=\sqrt3 AC$ \\
    (4)$\alpha=60^\circ,\beta=120^\circ\Rightarrow AB-AD=\sqrt3 AC$ \\
\end{model}
\begin{figure}[H]
\centering
\includegraphics[width=12cm]{picture/6108.png}
\caption{旋转模型}
\end{figure}
\begin{model}
    如图,正方形$ABCD$中,$AC,BD$交于$O$,$E$为$AB$上一点,$F$为$BC$上一点且$OE\perp OF$,有如下结论成立:\\
    (1)$\triangle BEO \backcong \triangle CFO$ \\
    (2)$OE=OF,\triangle EOF$为等腰直角三角形 \\
    (3)四边形$BEOF$面积为定值,且为正方形$ABCD$面积的$\frac{1}{4}$ \\
    (4)$BE+BF=\sqrt2 OB$ \\
    (5)$CF^2+BF^2=2OF^2$ \\
\end{model}
\begin{figure}[H]
\centering
\includegraphics[width=3cm]{picture/6109.png}
\caption{旋转模型}
\end{figure}
\begin{problem}
    如图,$n$个边长为1的正方形按照图示摆放,$A_1,A_2,…,A_n$分别是正方形的中心,求$n$个正方形重叠后阴影部分的面积。
\end{problem}
\begin{figure}[H]
\centering
\includegraphics[width=4cm]{picture/6112.png}
\caption{练习图}
\end{figure}
\begin{problem}
    如图,在正方形$ABCD$中,$F$为对角线$AC$上任意一点,$EF\perp BF$交$AD$于$E$,联结$BE$,求$\angle EBF$的度数。\\
\end{problem}
\begin{figure}[H]
\centering
\includegraphics[width=3cm]{picture/645.png}
\caption{练习图}
\end{figure}
\begin{problem}
    如图,正方形$ABCD$边长为6,$AC,BD$交于点$O$,$E$为$CD$上一点,$DE=2CE,CF\perp BE$交$BE$于$F$,求$OF$的长。
\end{problem}
\begin{figure}[H]
\centering
\includegraphics[width=3cm]{picture/6110.png}
\caption{练习图}
\end{figure}
\begin{problem}
    如图,在正方形$ABCD$中,$E$为$AC$上一点,$F$为$CD$上一点,$ED=EF$. \\
    (1)求证:$DF=\sqrt2 AE$ \\
    (2)求证:$BF=\sqrt2 EF$ \\
    (3)求证:$CB+CF=\sqrt2 CE$ \\
\end{problem}
\begin{figure}[H]
\centering
\includegraphics[width=3cm]{picture/6111.png}
\caption{练习图}
\end{figure}
\clearpage
\section{十字架模型}
\begin{model}
    在正方形$ABCD$中 \\
    (1)如左图,$AF\perp BE \Leftrightarrow AF=BE$ \\
    (2)如右图,$MP\perp NQ \Rightarrow MP=NQ$
\end{model}
\begin{figure}[H]
\centering
\includegraphics[width=6cm]{picture/6100.png}
\caption{十字架模型图}
\end{figure}
\begin{problem}
    如图,在正方形$ABCD$中,$E,F$分别为$CD,AD$中点,设正方形边长为$a$. \\
    (1)求证:$BE\perp CF$ \\
    (2)用含$a$的式子表示$AP,CP$的长度
\end{problem}
\begin{figure}[H]
\centering
\includegraphics[width=3cm]{picture/6101.png}
\caption{练习图}
\end{figure}
\begin{problem}
    如图,在正方形$ABCD$中,$E$为$AD$中点,$F$为$CD$上一点,$AF\perp BE$,$M$为$AD$上一点,且满足$BM=DM+CD$. \\
    (1)求证:$CF=FD$ \\
    (2)求证:$\angle MBC=2\angle ABE$
\end{problem}
\begin{figure}[H]
\centering
\includegraphics[width=4cm]{picture/6102.png}
\caption{练习图}
\end{figure}
\begin{problem}
    如图,正方形$ABCD$边长为3,$E$为$CD$上一点,$\angle DAE=30^\circ$,$M$为$AE$中点,过$M$作直线分别与$AD,BC$交于点$P,Q$,
    若$PQ=AE$,求$AP$的长。
\end{problem}
\begin{figure}[H]
\centering
\includegraphics[width=3cm]{picture/6103.png}
\caption{练习图}
\end{figure}
\begin{problem}
    如图,在正方形$ABCD$中,$E$在$AB$上(不与$A,B$)重合。过$E$作$FG\perp DE$与边$BC$交于$F$,与边$DA$延长线交于$G$。 \\
    (1)求证:$AG+BF=AE$ \\
    (2)联结$DF$,若正方形边长为2,$AE=x$,$\triangle DFG$面积为$y$,求$y$与$x$的函数解析式。\\
    (3)若正方形边长为2,$FG=\frac{5}{2}$,求点$C$到直线$DE$的距离。\\
\end{problem}
\begin{figure}[H]
\centering
\includegraphics[width=5cm]{picture/6_3_1.png}
\caption{练习图}
\end{figure}
\begin{problem}
    如图(1),在正方形$ABCD$中,点$E$是$BC$上一点,点$F$是$AE$上一点。过点$F$作$GH\perp AF$交直线$AB$于$G$,交直线$CD$于点$H$。 \\
    (1)求证:$BG=CH-BE$ \\
    (2)如图(2),若点$F$是$AE$延长线上一点,其余条件不变,试探究$BG,BE,CH$之间的数量关系,并证明。\\
\end{problem}
\begin{figure}[H]
\centering
\includegraphics[width=6cm]{picture/648.png}
\caption{练习图}
\end{figure}
\clearpage
\section{外角平分线模型}
\begin{example}
    如图,菱形$ABCD$中,$\angle B=60^\circ$,$E$为射线$BC$上一点,作$\angle AEF$交$CD$于$F$. \\
    (1)如左图,$E$在$BC$上,求证:$AE=EF$ \\
    \\
    (2)如右图,$E$在$BC$延长线上,求证:$AE=EF$ \\
    \\
\end{example}
\begin{figure}[H]
\begin{minipage}{0.48\linewidth}
\includegraphics[width=4cm]{picture/616.png}
\end{minipage}
\begin{minipage}{0.48\linewidth}
\includegraphics[width=4cm]{picture/617.png}
\end{minipage}
\end{figure}
\begin{example}
    如图,正方形$ABCD$中,点$E$是射线$BC$上一点,$AE\perp EF$交$\angle DCB$外角平分线于$F$. \\
    (1)如左图,若$E$在线段$BC$上,求证:$AE=EF$ \\
    \\
    (2)如右图,若$E$在$BC$延长线上,求证:$AE=EF$ \\
    \\
\end{example}
\begin{figure}[H]
\begin{minipage}{0.48\linewidth}
\includegraphics[width=4cm]{picture/651.png}
\end{minipage}
\begin{minipage}{0.48\linewidth}
\includegraphics[width=4cm]{picture/618.png}
\end{minipage}
\end{figure}

\clearpage
\section{半角模型}
\begin{model}
    如图,正方形$ABCD$边长为$a$,$E,F$为$BC,CD$上两动点,且满足$\angle EAF=45^\circ$,连$BD$交$AE,AF$于$M,N$,有如下结论成立:\\
    (1)$BE+DF=EF$ \\
    (2)$BM^2+ND^2=MN^2$ \\
    (3)$AE$平分$\angle BEF$,$AF$平分$\angle DFE$ \\
    (4)$A$到$EF$距离为定值$a$ \\
    (5)$\triangle ECF$周长为定值$2a$ \\
    (6)$\triangle ANE,\triangle AMF$为等腰直角三角形 \\
    (7)$\triangle ANM \sim \triangle AEF \sim \triangle DNF \sim \triangle BEM \sim \triangle DAM \sim \triangle BNA$ \\
    $\Rightarrow \frac{S_{\triangle AMN}}{S_{\triangle AEF}}=\frac{1}{2}$ \\
    $\Rightarrow AM^2=MD·MN$ \\
    $\Rightarrow AN^2=NB·NM$ \\
    (8)$BE=DF$时,$\triangle CEF$面积最大
\end{model}
\begin{figure}[H]
\centering
\includegraphics[width=4cm]{picture/6104.png}
\caption{半角模型}
\end{figure}
\begin{problem}
    如左图,已知正方形$ABCD$,将$\triangle DAE$与$\triangle DCF$分别沿$DE,DF$向内折叠得到右图,此时$DA$与$DC$重合($A,C$均落在$G$点),
    若$GF=4,EG=6$,求$DG$的长。
\end{problem}
\begin{figure}[H]
\centering
\includegraphics[width=6cm]{picture/6105.png}
\caption{练习图}
\end{figure}
\begin{problem}
    如图,正方形$ABCD$,$M$为$CB$延长线一点,作$\angle MAN$交$DC$于$N$,作$AH\perp MN$交于$H$.\\
    (1)求证:$MN=DN-BM$ \\
    (2)求证:$AH=AB$
\end{problem}
\begin{figure}[H]
\centering
\includegraphics[width=2.5cm]{picture/6106.png}
\caption{练习图}
\end{figure}
\begin{problem}
    (1)如图1,在正方形$ABCD$中,$E,F$为$BC,CD$上点,且$\angle EAF=45^\circ$,直接写出$BE,DF,EF$三条线段的数量关系。 \\
    (2)如图2,将正方形$ABCD$改为四边形$ABCD$,$AB=AD,\angle B+\angle D=180^\circ$,$E,F$分别为边$BC,CD$上的点,
    $\angle EAF=\frac{1}{2}\angle BAD$,判断(1)中结论是否依然成立。 \\
    (3)在(2)基础上将$\triangle AEF$绕$A$逆时针旋转,$E,F$运动到$BC,CD$延长线上,如图3,判断$BE,DF,EF$三条线段的数量关系并证明。 \\
\end{problem}
\begin{figure}[H]
\centering
\includegraphics[width=6cm]{picture/6107.jpg}
\caption{练习图}
\end{figure}
\begin{problem}
    在$\triangle ABC$中,$\angle BAC=45^\circ,AD\perp BC$于点$D$.
    将$\triangle ABD$沿$AB$所在直线折叠,使点$D$落在$E$处,将$\triangle ACD$沿$AC$所在直线折叠,使点$D$落在$F$处,延长$EB,FC$交于点$M$.
    (1)判断四边形$AEMF$的形状,并证明。\\
    (2)若$BD=1,CD=\frac{3}{2}$,求四边形$AEMF$的面积。 \\
\end{problem}
\begin{figure}[H]
\centering
\includegraphics[width=2.5cm]{picture/6119.png}
\caption{练习图}
\end{figure}
\clearpage
\section{对称性}
\begin{model}
    如图,正方形$ABCD$,$P$为对角线$BD$一点,有如下结论成立。\\
    (1)$\triangle ADP \backcong \triangle CDP$ \\
    (2)$\triangle ABP \backcong \triangle CBP$ \\
    (3)$AP=CP$ \\
    (4)过$P$作$PE\perp AB$交$AB$于$E$,作$PF\perp BC$交$BC$于$F$,则四边形$EPFB$为正方形 \\
\end{model}
\begin{figure}[H]
\centering
\includegraphics[width=4cm]{picture/652.png}
\caption{对称模型}
\end{figure}
\begin{problem}
    如上图,矩形$ABCD$,$P$为对角线$BD$上一点,且有$AP=CP,\angle ABP=\angle CBP$,求证:四边形$ABCD$为正方形。
\end{problem}
\begin{example}
    如图,菱形$ABCD$中,$AB=8,\angle ABC=60^\circ,M$为对角线$BD$上任意一点(不含$B$点). \\
    (1)求$AM+MC$的最小值 \\
    (2)求$AM+\frac{1}{2}BM$的最小值 \\
\end{example}
\begin{figure}[H]
\centering
\includegraphics[width=4cm]{picture/611.png}
\caption{例题图}
\end{figure}
\begin{example}
    如图,正方形$ABCD$以$AB$为边向外作等边三角形$ABE$,$M$为对角线$BD$上一点,将$BM$绕$B$逆时针旋转$60^\circ$到$BN$,若正方形边长为2,求
    $AM+BM+CM$的最小值。
\end{example}
\begin{figure}[H]
\centering
\includegraphics[width=4cm]{picture/612.png}
\caption{例题图}
\end{figure}
\begin{problem}
    如图,菱形$ABCD$的面积为120,正方形$AECF$的面积为50,求菱形的边长。\\
\end{problem}
\begin{figure}[H]
\centering
\includegraphics[width=6cm]{picture/650.png}
\caption{练习图}
\end{figure}
\begin{problem}
    如图,正方形$ABCD$中,$E$为$CD$中点。$BE,AC$交于$F$,连$DF$.\\
    (1)求证:$AE\perp DF$ \\
    (2)延长$DF$交$BC$于$M$,求证:$MB=MC$
\end{problem}
\begin{figure}[H]
\centering
\includegraphics[width=3cm]{picture/613.png}
\caption{练习图}
\end{figure}
\begin{problem}
    如图(1),正方形$ABCD$中,$P$为对角线$AC$上一点,$E$在$BC$延长线上,且$PE=PB$,证明:$DP\perp PE$。\\
    如图(2),菱形$ABCD$中,求证:$\angle DPE=\angle ABC$。\\
\end{problem}
\begin{figure}[H]
\centering
\includegraphics[width=6cm]{picture/640.png}
\caption{练习图}
\begin{problem}
    如图,$P$是正方形$ABCD$对角线$BD$上一动点,$PE\perp BC,PF\perp DC$,判断以下结论哪些是正确的:\\
    (1)$AP=EF$ \\
    (2)$AP\perp EF$ \\
    (3)$\triangle APD$为等腰三角形 \\
    (4)$\angle PFE=\angle BAP$ \\
    (5)$PD=\sqrt2 EC$ \\
\end{problem}
\begin{figure}[H]
\centering
\includegraphics[width=3cm]{picture/677.png}
\caption{练习图}
\end{figure}
\begin{problem}
    如图,$P$是正方形$ABCD$对角线$BD$上一动点,$PE\perp DC,PF\perp BC$.\\
    (1)探究$GE,GF,AB$的关系\\
    (2)探究$GE,GF,AG$的关系\\
    (3)探究$AG,EF$的关系\\
    (4)求证:$AB^2=AG^2+2GE·GF$\\
    (5)若$AB=6,AG=\sqrt{26}$,求$BG$的长\\
\end{problem}
\begin{figure}[H]
\centering
\includegraphics[width=4cm]{picture/614.png}
\caption{练习图}
\end{figure}
\end{figure}
\clearpage
\section{手拉手模型}
\begin{example}
    如图,在锐角三角形$ABC$中,$AH$是$BC$边上的高,分别以$AB,AC$为一边向外作正方形$ABDE$和正方形$ACFG$,连$CE,BG,EG$,$EG,HA$的延长线交于$M$,求证:\\
    (1)$BG=CE$ \\
    (2)$BG\perp CE$ \\
    (3)$AM$为$\triangle AEG$中线 \\
\end{example}
\begin{figure}[H]
\centering
\includegraphics[width=4cm]{picture/675.png}
\caption{练习图}
\end{figure}
\begin{problem}
    如图,正方形$ABCD$和正方形$DEFG$面积分别为9和13,$G$在线段$AB$上,求阴影部分的面积。
\end{problem}
\begin{figure}[H]
\centering
\includegraphics[width=4cm]{picture/615.png}
\caption{练习图}
\end{figure}
\clearpage
\section{三垂直模型}
\begin{model}
    如图,在正方形$ABCD$中,直线$l$经过点$D$,作$AM\perp l,BR\perp l,CN\perp l$,有如下结论成立:\\
    (1)$\triangle AMD \backcong \triangle CND$ \\
    (2)$AM+CN=MN=BR$ \\
\end{model}
\begin{figure}[H]
\centering
\includegraphics[width=4cm]{picture/624.png}
\caption{三垂直模型}
\end{figure}
\begin{problem}
    如图,正方形$ABCD,A(0,4)$,$E$是$AD$与$x$轴交点,且有$E(-2,0),AD=AE$,求$CE$的长度。
\end{problem}
\begin{figure}[H]
\centering
\includegraphics[width=3cm]{picture/625.png}
\caption{练习图}
\end{figure}
\begin{problem}
    如图,正方形$ABCD,A(-3,4)$,$D$在$y$轴上。\\
    (1)求$B$点坐标 \\
    (2)求$D$点坐标 \\
\end{problem}
\begin{figure}[H]
\centering
\includegraphics[width=3cm]{picture/626.png}
\caption{练习图}
\end{figure}
\begin{problem}
    如图,以$\triangle ABC$两边向外作正方形$ACDEA$和正方形$BCGF$,$P$为$EF$中点,$AB=a$,求$P$到$AB$的距离。\\
\end{problem}
\begin{figure}[H]
\centering
\includegraphics[width=4cm]{picture/627.png}
\caption{练习图}
\end{figure}
\begin{problem}
    如图,正方形$ABCD$中,$\angle 1=\angle 2,CE\perp AF$,垂足点为$E$,求证:$CE=\frac{1}{2}AF$.
\end{problem}
\begin{figure}[H]
\centering
\includegraphics[width=4cm]{picture/6116.png}
\caption{练习图}
\end{figure}
\begin{problem}
    已知直线$l$经过正方形$ABCD$的顶点,过点$B$和点$D$分别作直线$l$的垂线$BM$和$DN$,垂足分别为$M,N$。若$BM=5,DN=3$,求$MN$的长度。
\end{problem}
\clearpage
\section{最值问题}
\begin{problem}
    如图,直角三角形$ABC$中,$AC=3,BC=4,PE\perp AC,PF\perp BC$,求$EF$的最小值。
\end{problem}
\begin{figure}[H]
\centering
\includegraphics[width=3cm]{picture/659.png}
\caption{练习图}
\end{figure}
\begin{problem}
    如图,$E,F$是正方形$ABCD$的边$AD$上两个动点,满足$AE=DF$。连接$CF$交$BD$于$G$,连接$BE$交$AG$于$H$。
    若正方形边长为2,求线段$DH$长度的最小值。
\end{problem}
\begin{figure}[H]
\centering
\includegraphics[width=3cm]{picture/649.png}
\caption{练习图}
\end{figure}
\begin{problem}
    如图,四边形$ABCD,\angle A=90^\circ,AB=3\sqrt3,AD=3$.点$M,N$分别为线段$BC,AB$上的动点(含端点,但$M$不与$B$重合).
    点$E,F$分别为$DM,MN$的中点,求$EF$的最大值.
\end{problem}
\begin{figure}[H]
\centering
\includegraphics[width=3cm]{picture/631.png}
\caption{练习图}
\end{figure}
\clearpage
\section{梯形基础}
\begin{model}
    如图,梯形$ABCD$,$BE$平分$\angle ABC$,$CE$平分$\angle DCB$,$F$为$BC$中点,有如下结论成立:\\
    (1)$E$为$AD$中点 \\
    (2)$EF\px \px AB \px \px CD$ \\
    (3)$\angle BEC=90^\circ$ \\
    (4)$AB+CD=BC$ \\
    (5)$EF=\frac{1}{2}BC=\frac{1}{2}(AB+CD)$ \\
    (6)$S_{\triangle BEC}=\frac{1}{2}S_{ABCD}$
\end{model}
\begin{figure}[H]
\centering
\includegraphics[width=4cm]{picture/619.png}
\caption{模型图}
\end{figure}
\begin{problem}
    如图,反比例函数$y=\frac{6}{x}(x>0)$经过四边形$OABC$中$A,C$两点,$AB$平行$x$轴,$OC$平分$OA$与$x$轴正半轴夹角,且$\angle B=90^\circ$,
    求四边形$OABC$的面积。
\end{problem}
\begin{figure}[H]
\centering
\includegraphics[width=4cm]{picture/620.png}
\caption{模型图}
\end{figure}
\begin{problem}
    在平面直角坐标系中有$A,B$两点,$A(-4,0),B(0,2)$。将直线$AB$向下平移$b+4(b>0)$个单位后得到直线$l$. \\
    (1)用$b$表示直线$l$的函数解析式 \\
    (2)若$x$轴上有一点$C(b,0)$,过$C$作$CD\perp x$轴与直线$l$交于$D$,延长$BC$交$l$于点$E$. \\
    \textcircled{\small{1}}当四边形$ACED$面积为9时,求直线$l$的函数解析式 \\
    \textcircled{\small{2}}求证:$AD-DE$为定值,并求出该定值
\end{problem}
\begin{model}
    梯形的常见辅助线方式如图所示。
\end{model}
\begin{figure}[H]
\centering
\includegraphics[width=8cm]{picture/673.png}
\end{figure}
\begin{figure}[H]
\centering
\includegraphics[width=8cm]{picture/674.png}
\caption{梯形常见辅助线}
\end{figure}
\begin{table}[H]
\centering
\caption{梯形辅助线思路}
\begin{tabular}{|l|l|}
\hline
    & 思路 \\
\hline
作平行 & 构造平行四边形 \\
\hline
作双高 & 求解腰长,常见于等腰梯形中 \\
\hline
平移对角线 & 常见于对角线垂直或相等 \\
\hline
延长两腰相交 & 常见于等腰梯形中 \\
\hline
倍长中线 & 腰上有中点 \\
\hline
\end{tabular}
\end{table}
\begin{problem}
    梯形$ABCD,AD\px \px BC,AD=2,BC=7,AB=4$,求$CD$的取值范围。
\end{problem}
\begin{problem}
    梯形$ABCD,AD\px \px BC,AC\perp BD,AC=3$,若梯形中位线长为2.5,求梯形的面积。
\end{problem}
\begin{problem}
    梯形$ABCD$中,$AD\px \px BC,\angle B=30^\circ,\angle C=75^\circ,AD=2,BC=7$,求$AB$的长。
\end{problem}
\begin{problem}
    如图,梯形$ABCD,AD\px \px BC,\angle B=52^\circ,\angle C=38^\circ,AD=6,BC=10$,$E,F$分别为$AD,BC$中点,求$EF$的长。
\end{problem}
\begin{figure}[H]
\centering
\includegraphics[width=4cm]{picture/628.png}
\caption{练习图}
\end{figure}
\begin{problem}
    等腰梯形$ABCD,AD\px \px BC,AB=CD,AC\perp BD$,若梯形的中位线长为$a$,求梯形$ABCD$的面积。
\end{problem}
\begin{problem}
    如图,$E$为梯形$ABCD$的中点,$AD\px \px BC,EF\perp AB,AB=a,EF=b$,求梯形的面积。
\end{problem}
\begin{figure}[H]
\centering
\includegraphics[width=4cm]{picture/629.png}
\caption{练习图}
\end{figure}
\begin{problem}
    如图,梯形$ABCD$中,$AD\px\px BC,\angle BAC=90^\circ,AB=AC,BD=BC$,求$\angle BOC$的度数。
\end{problem}
\begin{figure}[H]
\centering
\includegraphics[width=4cm]{picture/636.png}
\caption{练习图}
\end{figure}
\begin{problem}
    如图,梯形$ABCD$中,$AD\px\px BC,\angle B=90^\circ,AD=AB=4,BC=7$,点$E$在$BC$边上,
    将$\triangle CDE$沿$DE$折叠,$C$恰好落在$AB$边上的$C'$处。 \\
    (1)求$\triangle C'DE$的面积 \\
    (2)求$\angle CDC'$的度数 \\
\end{problem}
\begin{figure}[H]
\centering
\includegraphics[width=4cm]{picture/637.png}
\caption{练习图}
\end{figure}

\begin{problem}
    如图,在平行四边形$ABCD$中,将$\triangle ABE$沿着$AE$翻折得到$\triangle AFE$,延长$AF$交$CD$于$G$,证明:$FG=CG$。 \\
\end{problem}
\begin{figure}[H]
\centering
\includegraphics[width=4cm]{picture/638.png}
\caption{练习图}
\end{figure}

\begin{problem}
    如图,在梯形$ABCD$中,$AD\px\px BC,BC=DC,CF$平分$\angle BCD,DF\px\px AB$,$BF$延长线交$DC$于$E$,求证:$AD=DE$\\
\end{problem}
\begin{figure}[H]
\centering
\includegraphics[width=4cm]{picture/642.png}
\caption{练习图}
\end{figure}
\begin{problem}
    直线$y=\frac{1}{3}x+1$与$x$轴,$y$轴分别交于点$A,B$,$C$点坐标为(2,0)。若$D$在$y$轴上,且以$A,B,C,D$四点组成梯形,求所有满足条件的$D$点坐标。
\end{problem}
\clearpage

\section{直角梯形}
\begin{problem}
    在直角梯形$ABCD$中,$AD\px\px BC,\angle A=90^\circ,CD=5,AB=\frac{5\sqrt3}{2}$,求$\angle D$的度数。\\
\end{problem}
\begin{problem}
    一次函数$y=\frac{\sqrt3}{3}x+b$与$x$轴交于$A(5\sqrt3,0)$,与$y$轴交于$B$。若$C(0,3)$,且$A,B,C,D$四点构成直角梯形,求$D$点坐标。\\
\end{problem}
\begin{problem}
    如图,平面直角坐标系中,$\angle ACO=90^\circ,A(5,4)$,点$P$为$AC$上一点,将$\triangle OCP$沿直线$OP$翻折,点$C$落在$C'$处,
    连$AC'$,若$AC'\px \px OC$,求$P$点坐标。
\end{problem}
\begin{problem}
\begin{figure}[H]
\centering
\includegraphics[width=4cm]{picture/6126.png}
\caption{练习图}
\end{figure}
    如图,直角梯形$ABCD$中,$AD\px\px BC,\angle B=90^\circ,AB=3,BC=4,AD=2,DE\perp AC$,求$DE$的长。
\end{problem}
\begin{figure}[H]
\centering
\includegraphics[width=3cm]{picture/689.png}
\caption{练习图}
\end{figure}
\begin{problem}
    如图,直角梯形$ABCD$中,$AD\px\px BC,\angle ABC=90^\circ,AB=BC=8$。点$E$在边$AB$上,
    $DE\perp CE$,$DE$的延长线与$CB$延长线交于点$F$. \\
    (1)求证:$DF=CE$ \\
    (2)当点$E$为$AB$中点时,求$CD$的长 \\
    (3)设$CE=x,AD=y$,试用$x$的代数式表示$y$ \\
\end{problem}
\begin{figure}[H]
\centering
\includegraphics[width=4cm]{picture/685.png}
\caption{练习图}
\end{figure}
\clearpage

\section{等腰梯形}
\begin{knowledge}
    等腰梯形$ABDC$中,$AC=BD,AB\px\px CD$,有如下结论成立: \\
    (1)$\angle ACD=\angle BDC $  证明方式:过$B$作$BE \px \px AC$交$CD$于E \\
    (2)$\angle CAB=\angle DBA $  证明方式:两直线平行,同旁内角互补\\
    (3)对角互补 \\
    (4)$AD=BC$ 证明方式:$\triangle ACD \backcong \triangle BDC$ \\
    (5)$AO=BO$ 证明方式:$\triangle ACB \backcong \triangle BDA$ \\
    (6)$CO=DO$ 证明方式:$\triangle ACD\backcong \triangle BDC$
\end{knowledge}
\begin{table}[H]
\centering
\caption{等腰梯形的定义、性质和判定}
\begin{tabular}{|l|l|}
\hline
    & 描述 \\
\hline
定义 & 两腰相等的梯形是等腰梯形 \\
\hline
判定 & 同一底边上两个底角相等的梯形是等腰梯形 \\
\hline
判定 & 对角线相等的梯形是等腰梯形 \\
\hline
性质 & 等腰梯形两腰相等,同一底边上两底角相等 \\
\hline
性质 & 等腰梯形对角线相等 \\
\hline
\end{tabular}
\end{table}
\begin{figure}[H]
\centering
\includegraphics[width=4cm]{picture/633.jpeg}
\caption{等腰梯形}
\end{figure}
\begin{problem}
    如上图,四边形$ABDC$,$AD,BC$相交于点$O$。\\
    (1)若$OC=OD,OA=OB,OA\ne OD$,求证:四边形$ABDC$为等腰梯形。 \\
    \\
    (2)延长$AC,BD$交于点$E$,若$EA=EB,AC=BD$,求证:四边形$ABDC$为等腰梯形。\\
\end{problem}
\begin{model}
    如图,等腰梯形$ABCD$,$CD\px \px AB,\angle A=60^\circ,AD=DC=CB=a$,则有如下结论成立:\\
    (1)$AB=2a$ \\
    (2)$BD$平分$\angle ABC$ \\
    (3)$BD\perp AD$ \\
    (4)梯形$ABCD$面积为$\frac{3\sqrt{3}}{4}a^2$
\end{model}
\begin{figure}[H]
\centering
\includegraphics[width=4cm]{picture/634.png}
\caption{有60度角的等腰梯形}
\end{figure}
\begin{problem}
    有一梯形四边长为6、6、6、12,求梯形的面积。\\
\end{problem}
\begin{problem}
    如图,直角三角形$ABC,\angle ACB=90^\circ,AB=4$,将一个$30^\circ$的角的顶点$P$放在边$AB$边上滑动,保持$30^\circ$角的一边平行$BC$,
    交$AC$边于点$E$,另一边交射线$BC$于点$D$,连$ED$.  \\
    (1)若四边形$PBDE$为等腰梯形,求$AP$的长。\\
    (2)若四边形$PBDE$为平行四边形,求$AP$的长。\\
\end{problem}
\begin{figure}[H]
\centering
\includegraphics[width=4cm]{picture/6129.png}
\caption{练习图}
\end{figure}
\begin{problem}
    等腰梯形有三边长为3、4、11,求等腰梯形的周长。\\
\end{problem}
\begin{problem}
    如图,在$\triangle ABC$中,$D,E$分别是$AC,BC$上的点,$AE,BD$交于点$O$,$CD=CE,\angle 1=\angle 2$,求证:四边形$ABED$为等腰梯形。\\
\end{problem}
\begin{figure}[H]
\centering
\includegraphics[width=4cm]{picture/662.png}
\caption{练习图}
\end{figure}
\begin{problem}
    如图,等腰梯形$ABCD$中位线$EF$长为6,对角线$BD$平分$\angle ADC$,下底$BC$比等腰梯形周长小$20$,求上底$AD$的长。
\end{problem}
\begin{figure}[H]
\centering
\includegraphics[width=4cm]{picture/6117.png}
\caption{练习图}
\end{figure}
\begin{problem}
    如图,在梯形$ABCD$中,$AD\px\px BC$,$E$是梯形内一点,$ED\perp AD$交$BC$于$F$,$\angle EBC=\angle EDC,\angle ECB=45^\circ$ \\
    (1)求证:$BE=CD$.\\
    (2)若梯形$ABCD$为等腰梯形,求证:$AD=DE$. \\
\end{problem}
\begin{figure}[H]
\centering
\includegraphics[width=4cm]{picture/663.png}
\caption{练习图}
\end{figure}
\begin{problem}
    如图,在等腰梯形$ABCD$中,$AD\px\px BC,AB=CD=AD=6,BC=12,P$为$BC$上一动点,$PE\perp AB,PF\perp CD$。求$PE+PF$的值。
\end{problem}
\begin{figure}[H]
\centering
\includegraphics[width=4cm]{picture/664.png}
\caption{练习图}
\end{figure}
\clearpage
\section{蝴蝶定理}
\begin{model}
    如图,梯形$ABCD$中,$AD\px\px BC,AD=a,BC=b$,有如下结论成立:\\
    (1)$S_3=S_4$ \\
    (2)$S_3·S_4=S_1·S_2$ \\
    (3)$\frac{S_1}{S_2}=\frac{a^2}{b^2}$ \\
    (4)$S_1:S_2:S_3:S_4:S_{ABCD}=a^2:b^2:ab:ab:(a+b)^2$
\end{model}
\begin{figure}[H]
\centering
\includegraphics[width=4cm]{picture/690.png}
\caption{蝴蝶定理模型图}
\end{figure}
\begin{problem}
    如图,$M$为正方形$ABCD$边$AD$的中点,若正方形面积为3,求阴影部分的面积。
\end{problem}
\begin{figure}[H]
\centering
\includegraphics[width=4cm]{picture/691.png}
\caption{练习图}
\end{figure}
\begin{problem}
    如图,正方形$ABCD$和正方形$EBGF$并排在一起,若正方形$ABCD$边长为2,求阴影部分的面积。
\end{problem}
\begin{figure}[H]
\centering
\includegraphics[width=4cm]{picture/692.png}
\caption{练习图}
\end{figure}
\clearpage
\section{中位线}
\begin{knowledge}
    如图,$\triangle ABC$,$D,E,F$分别为$AB,BC,AC$的中点,则有如下结论成立:\\
    (1)$DF\px\px BC,DF=\frac{1}{2}BC$ \\
    (2)四边形$ADEF,BEFE,CEDF$均为平行四边形 \\
    (3)已知平面内不共线三点$D,E,F$,在平面内再找一个点,使得这四个点构成平行四边形,这样的点有且仅有3个,即图中的$A,B,C$ \\
    (4)$S_{\triangle ADF}=\frac{1}{4}S_{\triangle ABC},C_{\triangle ADF}=\frac{1}{2}C_{\triangle ABC}$ \\
\end{knowledge}
\begin{corollary}
    如图,$\triangle ABC$,$D$为$AB$中点,$DF \px \px BC$交$AC$于$F$,则$F$也为$AC$中点。\\
    证明:在$AC$上取中点$F'$,则$DF'\px \px BC$,又$DF\px \px BC$。所以$F'$与$F$重合(过直线外一点有且只有一条直线与已知直线平行)。
\end{corollary}
\begin{figure}[H]
\centering
\includegraphics[width=4cm]{picture/693.png}
\caption{三角形中位线}
\end{figure}
\begin{knowledge}
    如图,梯形$ABCD,AD\px \px BC$,$E$为$AB$中点,$F$为$CD$中点,则有如下结论成立:\\
    (1) $AD \px \px EF \px \px BC$ \\
    (2) $EF=\frac{1}{2}(AD+BC)$ \\
    (3)若$h$为梯形的高,则梯形面积为$EF·h$ \\
\end{knowledge}
\begin{figure}[H]
\centering
\includegraphics[width=4cm]{picture/694.png}
\caption{梯形中位线}
\end{figure}
\begin{model}
    如图,梯子上有若干条直线$A_iB_i$互相平行,且满足$A_1A_2=A_2A_3=...=A_iA_{i+1},B_1B_2=B_2B_3=...=B_iB_{i+1}.$
    若$A_1B_1=a,A_2B_2=b$,则$A_iB_i=a-(i-1)(a-b).$
\end{model}
\begin{figure}[H]
\centering
\includegraphics[width=4cm]{picture/695.png}
\caption{等间距平行线}
\end{figure}
\begin{problem}
    在$\triangle ABC$中,$AB=12$,$AC$上有$A_1,A_2,A_3,A_4,A_5,A_6$将$AC$七等分,$BC$上有$B_1,B_2,B_3,B_4,B_5,B_6$将$BC$七等分,
    求$A_1B_1+A_2B_2+A_3B_3+A_4B_4+A_5B_5+A_6B_6$的值。
\end{problem}
\begin{model}
    如图,梯形$ABCD$中,$BC>AD,AD\px \px BC$,$EF$为梯形中位线,交$AC,BD$于$M,N$两点,有如下结论成立:\\
    (1)$M$为$BD$中点,$N$为$AC$中点 \\
    (2)$MN=\frac{1}{2}(BC-AD)$
\end{model}
\begin{figure}[H]
\centering
\includegraphics[width=4cm]{picture/696.png}
\caption{模型图}
\end{figure}
\begin{problem}
    如图,等腰梯形$ABCD$,$AD\px BC,AB=CD$,$E,F$为$BD,AC$中点,$EF=a$.若$\angle ABC=60^\circ,BD\perp CD$,求梯形$ABCD$的面积。
\end{problem}
\begin{figure}[H]
\centering
\includegraphics[width=4cm]{picture/623.png}
\caption{练习图}
\end{figure}
\begin{problem}
    如图,$\triangle ABC,\angle ABE=\angle EBC,AE\perp BE$,$F$为$AC$中点,求证:$EF=\frac{1}{2}(BC-AB)$
\end{problem}
\begin{figure}[H]
\centering
\includegraphics[width=4cm]{picture/604.png}
\caption{练习图}
\end{figure}
\begin{problem}
    如图,已知梯形$ABCD,AB\px\px CD$,以$AC,AD$为边作平行四边形$ACED,DC$延长线交$BE$于$F$. \\
    (1)求证:$EF=FB$ \\
    (2)若$S_{\triangle BCE}=\frac{1}{3}S_{ABCD}$,求$\frac{AB}{CD}$
\end{problem}
\begin{figure}[H]
\centering
\includegraphics[width=4cm]{picture/697.jpg}
\caption{练习图}
\end{figure}
\begin{problem}
    如图,在$\triangle ABC$中,$\angle ABC,\angle ACB$的角平分线$BE,CF$相交于点$O,AG\perp BE$于$G,AH\perp CF$于$H,C_{\triangle ABC}=41,BC=18$,求$HG$的长度。
\end{problem}
\begin{figure}[H]
\centering
\includegraphics[width=4cm]{picture/698.png}
\caption{练习图}
\end{figure}
\begin{problem}
    如图,点$D$为$\triangle ABC$边$AB$的中点,$E$为$AC$上一点,且$AE=2CE.BE,CD$相交于$O$.若$BE=12$,求$OE$的长度。
\end{problem}
\begin{figure}[H]
\centering
\includegraphics[width=4cm]{picture/699.png}
\caption{练习图}
\end{figure}
\begin{problem}
    如图,矩形$ABCD$中,$DE\perp AC$交$BC$于$E$,点$F$在$CD$上,$BF,DE$交于$G$,且$BG=GF=FD.$若$AC=2\sqrt6$,求$BC$的长度。
\end{problem}
\begin{figure}[H]
\centering
\includegraphics[width=4cm]{picture/600.png}
\caption{练习图}
\end{figure}
\begin{problem}
    如图,在平行四边形$ABCD$中,$E$为$CD$的中点,$F$为$AE$的中点,$FC$与$BE$交于点$G$,求证:$GF=GC$.
\end{problem}
\begin{figure}[H]
\centering
\includegraphics[width=4cm]{picture/686.png}
\caption{练习图}
\end{figure}
\begin{problem}
    如图,正方形$ABCD$和正方形$GBEF$面积分别为$m,n$,$A,B,E$三点共线,$H$为$DF$中点,求$BH$的长。
\end{problem}
\begin{figure}[H]
\centering
\includegraphics[width=4cm]{picture/621.jpg}
\caption{练习图}
\end{figure}
\begin{problem}
    如图,边长为1的正方形$EFGH$在边长为3的正方形$ABCD$所在的平面上移动,且始终满足$EF\px \px AB$,若$M,N$分别为线段$CF,DH$的中点,求线段$MN$的长度。
\end{problem}
\begin{figure}[H]
\centering
\includegraphics[width=3cm]{picture/622.png}
\caption{练习图}
\end{figure}
\begin{problem}
    (1)如图a,四边形$ABCD,AB=CD$,$E,F$分别是$BC,AD$的中点,$EF$与$BA,CD$延长线交于$M,N$,求证:$\angle BME=\angle CNE$. \\
    (2)如图b,四边形$ADBC,AB=CD$,$E,F$分别是$BC,AD$的中点,连$EF$交$DC,AB$于$M,N$,判断$\triangle OMN$的形状,并证明. \\
    (3)如图c,$\triangle ABC,AC>AB$,$D$为$AC$上一点,且$AB=CD$,$E,F$分别是$BC,AD$的中点,连$EF$与$BA$延长线交于$G$.若$\angle EFC=60^\circ$,连$GD$,
    判断$\triangle AGD$的形状并证明.
\end{problem}
\begin{figure}[H]
\centering
\includegraphics[width=12cm]{picture/630.png}
\caption{练习图}
\begin{problem}
    如图,$C,D$为线段$AB$上两点,$P$是线段$CD$上的动点,分别以$AP,BP$为边在$AB$同侧作两个等边三角形$APE$和$BPF$,$M$是$EF$的中点,
    已知$AB=20,AC=BD=2$,当$P$从$C$运动到$D$时(无重复运动),求$M$点的运动路径长。
\end{problem}
\end{figure}
\begin{figure}[H]
\centering
\includegraphics[width=3cm]{picture/632.png}
\caption{练习图}
\end{figure}
\clearpage
\section{中点四边形}
\begin{knowledge}
    顺次连接四边形各边中点组成的四边形叫做中点四边形,有如下结论成立:\\
    (1)任意四边形的中点四边形为平行四边形 \\
    (2)对角线互相垂直的四边形的中点四边形为矩形 \\
    (3)对角线相等的四边形的中点四边形为菱形 \\
    (4)对角线垂直且相等的四边形的中点四边形为正方形 \\
    (5)中点四边形面积为原四边形面积的一半  \\
    (6)中点四边形周长为原四边形两条对角线的长,一定小于原四边形的周长 \\
\end{knowledge}
\begin{figure}[H]
\centering
\includegraphics[width=10cm]{picture/601.png}
\caption{练习图}
\end{figure}
\begin{problem}
    判断:若一个四边形的中点四边形为菱形,则原四边形为矩形或等腰梯形。
\end{problem}
\begin{problem}
    如图,四边形$ABCD$中,点$E,F,G$分别为$AB,BC,CD$中点,若$\triangle EFG$面积为4,求四边形$ABCD$的面积。
\end{problem}
\begin{figure}[H]
\centering
\includegraphics[width=4cm]{picture/602.png}
\caption{练习图}
\end{figure}
\clearpage
\section{平面向量}
\begin{knowledge}
    $\quad$ \\
    (1)既有大小又有方向的量叫做向量 \\
    (2)平行向量:方向相同或相反的向量 \\
    (3)相反向量:方向相反,长度相同的向量 \\
    (4)相等向量:长度,方向均相同的向量 \\
    (5)向量的模:向量的长度,记作$\vec{x}$ \\
    (6)零向量:长度为零的向量,记作$\vec{0}$ \\
    (7)规定零向量与任意向量平行 \\
    (8)$\vec{a}+\vec{b}=\vec{b}+\vec{a}$ \\
    (9)$(\vec{a}+\vec{b})+\vec{c}=\vec{a}+(\vec{b}+\vec{c})$ \\
    (10)$\vec{a}+(-\vec{a})=\vec{0}$
\end{knowledge}
\begin{model}
    如图,平行四边形$ABCD$,$AC,BD$交于点$O$. \\
    (1)与$\overrightarrow{AO}$平行的向量有:$\overrightarrow{OA},\overrightarrow{AC},\overrightarrow{OC},\overrightarrow{CO},\overrightarrow{CA}$ \\
    (2)$|\overrightarrow{OB}|=|\overrightarrow{OD}|=|\overrightarrow{BO}|=|\overrightarrow{DO}|$ \\
    (3)$\overrightarrow{AB}+\overrightarrow{BC}=\overrightarrow{AC}$ \\
    (4)$\overrightarrow{BA}+\overrightarrow{BC}=\overrightarrow{BD}$ \\
    (5)$\overrightarrow{BA}-\overrightarrow{BC}=\overrightarrow{CA}$ \\
    (6)$\overrightarrow{BA}-\overrightarrow{DA}=\overrightarrow{BD}$ \\
\end{model}
\begin{figure}[H]
\centering
\includegraphics[width=3cm]{picture/6113.png}
\caption{平行四边形与向量}
\begin{problem}
    如图,梯形$ABCD$中,$DC\px \px AB,CE\perp AB$,且$\overrightarrow{BA}-\overrightarrow{EA}-\overrightarrow{BC}+\overrightarrow{AD}=\vec{0}$,
    求证:梯形$ABCD$为直角梯形。
\end{problem}
\begin{figure}[H]
\centering
\includegraphics[width=4cm]{picture/6114.png}
\caption{练习图}
\end{figure}
\begin{problem}
     如图,$\triangle ABC$,$D,E,F$分别为$AB,BC,AC$的中点。\\
     (1)$\overrightarrow{AB}+\overrightarrow{AC}$ \\
     (2)
\end{problem}
\end{figure}
\begin{figure}[H]
\centering
\includegraphics[width=4cm]{picture/693.png}
\caption{练习图}
\end{figure}
\clearpage
\chapter{概率}
\section{列表法与树形图法}
\clearpage
\section{概率加法原理和乘法原理}
\clearpage
\section{有放回与无放回}
\clearpage
\section{抽奖问题}
\clearpage
\section{几何概型}
\clearpage
\section{简单的排列组合}
\clearpage
\chapter{相似三角形}
\section{比例}
\begin{table}[H]
\centering
\caption{比例的性质}
\begin{tabular}{l|p{12cm}}
\hline
\hline
性质 & 描述 \\
\hline
基本性质 & $ \frac{a}{b}=\frac{c}{d}\Leftrightarrow ad=bc(bd \ne 0) $ \\
\hline
反比定理 & $ \frac{a}{b}=\frac{c}{d}\Leftrightarrow \frac{b}{a}=\frac{d}{c}(abcd \ne 0) $ \\
\hline
更比定理 & $ \frac{a}{b}=\frac{c}{d}\Leftrightarrow \frac{a}{c}=\frac{b}{d}(bcd \ne 0) $ \\
\hline
合比定理 & $ \frac{a}{b}=\frac{c}{d}\Leftrightarrow \frac{a+b}{b}=\frac{c+d}{d}(bd \ne 0) $ \\
\hline
分比定理 & $ \frac{a}{b}=\frac{c}{d}\Leftrightarrow \frac{a-b}{b}=\frac{c-d}{d}(bd \ne 0) $ \\
\hline
合分比定理 & $ \frac{a}{b}=\frac{c}{d}\Leftrightarrow \frac{a+b}{a-b}=\frac{c+d}{c-d}(bd \ne 0,a\ne b,c\ne d) $ \\
\hline
等比定理 & $ \frac{a_1}{b_1}=\frac{a_2}{b_2}=...=\frac{a_n}{b_n}=k(b_1+b_2+...+b_n\ne 0)\Rightarrow \frac{a_1+a_2+...+a_n}{b_1+b_2+...+b_n}=k $ \\
\hline
其他 & $\frac{a}{b}=\frac{c}{d}\Leftrightarrow \frac{a}{b+a}=\frac{c}{d+c}(bd \ne 0,b+a\ne 0,d+c \ne 0)$ \\
\hline
\hline
\end{tabular}
\end{table}
\begin{problem}
    若$\frac{x}{10}=\frac{y}{8}=\frac{z}{9}$,求$\frac{x+y+z}{y+z}$的值。\\
\end{problem}
\begin{problem}
    若$\frac{a}{b}=\frac{2}{3}(a\ne 2,b\ne 3)$,求$\frac{a-b+1}{a+b-5}$的值。\\
\end{problem}
\begin{problem}
    已知$k=\frac{a+b}{c}=\frac{b+c}{a}=\frac{c+a}{b}$,求$k$的值。\\
\end{problem}
\begin{problem}
    若实数$m\ne n$,且$\frac{8m+n}{8n+m}=\frac{m+1}{n+1}$,求$m+n$的值。 \\
\end{problem}
\clearpage
\section{比例线段与黄金比例}
\begin{knowledge}
    比例式$a:b=c:d$中,$a,d$称为比例外项,$b,c$称为比例内项,$d$称为第四比例项。\\
    若$b=c$,即$a:b=b:d$,则$b$为$ad$的比例中项,满足$b^2=ad$. \\
    若有四条线段长度为$a,b,c,d$,且满足$a:b=c:d$,则这四条线段为成比例线段。
\end{knowledge}
\begin{knowledge}
    在线段$AB$上找一点$P$,把线段$AB$分成两条线段$AP,BP(AP>BP)$,且满足$AP^2=AB·BP$,则$P$为线段$AB$黄金分割点。
    设$AB=l,AP=x$,则$x^2=l(l-x)$,$x^2+lx-l^2=0$,解得$x=\frac{\sqrt5-1}{2}l$.\\
    (1)$AP=\frac{\sqrt5 - 1}{2}AB \approx 0.618AB$ \\
    (2)$BP=\frac{3 - \sqrt5}{2}AB \approx 0.382AB$ \\
    (3)黄金分割点有2个,另一个满足$AP<BP,BP^2=AB·AP.$ \\
    (4)$\frac{\sqrt5-1}{2}+\frac{3-\sqrt5}{2}=1;\frac{\sqrt5-1}{2}·\frac{\sqrt5+1}{2}=1$.
\end{knowledge}
\begin{model}
    黄金三角形是指顶角或底角为$36^\circ$的等腰三角形。\\
    如左图,顶角为$36^\circ$的等腰三角形,底边与腰的比$\frac{BC}{AB}=\frac{\sqrt5-1}{2}$. \\
    如右图,底角为$36^\circ$的等腰三角形,腰与底边的比$\frac{AB}{BC}=\frac{\sqrt5-1}{2}$.
\end{model}
\begin{figure}[H]
\centering
\includegraphics[width=6cm]{picture/801.png}
\caption{黄金三角形}
\end{figure}
\begin{model}
    如图,矩形$ADFB$中,$\frac{AD}{AB}=\frac{\sqrt5-1}{2}$,则该矩形为黄金矩形。黄金矩形可以再次切分为一个正方形和一个黄金矩形。
\end{model}
\begin{figure}[H]
\centering
\includegraphics[width=4cm]{picture/859.png}
\caption{黄金矩形}
\end{figure}
\begin{problem}
    已知点$P$是线段$AB$的黄金分割点,且$AP=\sqrt5-1$,求$AB$的长。
\end{problem}
\begin{problem}
    已知线段$AB=2$,$C$为$AB$的黄金分割点$(AC<BC)$,点$D$在$AB$上,且满足$AD^2=BD·AB$,求$\frac{CD}{AC}$的值。
\end{problem}
\clearpage
\section{面积法}
\begin{knowledge}
    对于等高或同高的三角形,三角形面积比等于底边长度的比。\\
    (探究1)如图(蝴蝶定理),四边形$ABCD$中,$S_{\triangle AOB}=S_{\triangle COD}$,求证:$\frac{OA}{OC}=\frac{OD}{OB}.$ \\
    证明:$\frac{S_{\triangle AOB}}{S_{\triangle COB}}=\frac{OA}{OC}=\frac{S_{\triangle COD}}{S_{\triangle COB}}=\frac{OD}{OB}$. \\
    (探究2)$\triangle ABC$中,$D$为$AB$中点,$DE\px \px BC$交$AC$于$E$,求证:$E$为$AC$中点。\\
    证明:$S_{\triangle ADE}=S_{\triangle BDE}=S_{\triangle CDE}.$ \\
    (探究3)如图(燕尾定理),$\triangle ABC$中,$AD,BE,CF$交于点$O$,求证:$\frac{S_{\triangle AOB}}{S_{\triangle AOC}}=\frac{BD}{CD}$. \\
    证明:$\frac{S_{\triangle ABD}}{S_{\triangle ACD}}=\frac{BD}{CD}=\frac{S_{\triangle BOD}}{S_{\triangle COD}}=\frac{S_{\triangle ABD}-S_{\triangle BOD}}{S_{\triangle ACD}-S_{\triangle COD}}=\frac{S_{\triangle AOB}}{S_{\triangle AOC}}.$
\end{knowledge}
\begin{figure}[H]
\centering
\includegraphics[width=4cm]{picture/690.png}
\caption{蝴蝶定理}
\end{figure}
\begin{figure}[H]
\centering
\includegraphics[width=4cm]{picture/860.png}
\caption{燕尾定理}
\end{figure}
\begin{model}
    如图(蝴蝶定理),梯形$ABCD$中,$AD\px\px BC,AD=a,BC=b$,有如下结论成立:\\
    (1)$S_3=S_4$ \\
    (2)$S_3·S_4=S_1·S_2$ \\
    (3)$\frac{S_1}{S_2}=\frac{a^2}{b^2}$ \\
    (4)$S_1:S_2:S_3:S_4:S_{ABCD}=a^2:b^2:ab:ab:(a+b)^2$
\end{model}
\begin{knowledge}
    在本章其他小节,利用面积法还可以证明如下结论:\\
    (1)平行线分线段成比例 \\
    (2)三角形重心性质 \\
    (3)角平分线定理 \\
    (4)塞瓦定理
\end{knowledge}
\clearpage
\section{平行线分线段成比例}
\begin{knowledge}
    平行线分线段成比例定理:两条直线被三条平行的直线所截,截得的对应线段成比例。\\
    如图,即$a\px\px b\px\px c \Rightarrow \frac{AB}{BC}=\frac{DE}{EF}.$ \\
    证明:$\frac{AB}{BC}=\frac{S_{\triangle ABE}}{S_{\triangle CBE}}=\frac{S_{\triangle DBE}}{S_{\triangle FBE}}=\frac{DE}{EF}.$ \\
    根据比例的性质还能推出如下结论:\\
    (1)$\frac{AB}{AC}=\frac{DE}{DF}$(上:总=上:总) \\
    (2)$\frac{BC}{AC}=\frac{EF}{DF}$(下:总=下:总) \\
    (3)$\frac{AB}{DE}=\frac{BC}{EF}=\frac{AC}{DF}$(上:上=下:下=总:总) \\
    平行线等分线段定理:两条直线被三条平行的直线所截,如果在一条直线上截得的线段相等,那么在另一条直线上截得的线段也相等。\\
    如图,即$a\px\px b\px\px c,AB=BC \Rightarrow DE=EF$. \\
    注意:$\frac{AB}{BC}=\frac{DE}{EF} \nRightarrow a \px \px b \px \px c$
\end{knowledge}
\begin{figure}[H]
\centering
\includegraphics[width=6cm]{picture/802.png}
\caption{平行线分线段成比例}
\end{figure}
\begin{problem}
    如图,三条平行线与两条直线相交,写出成比例的线段。
\end{problem}
\begin{figure}[H]
\centering
\includegraphics[width=10cm]{picture/803.png}
\caption{练习图}
\end{figure}
\clearpage
\section{三角形一边的平行线}
\begin{knowledge}
    三角形一边的平行线性质定理:平行于三角形一边的直线截其他两边所在的直线,截得的对应线段成比例。\\
    如图,即$DE \px \px BC \Rightarrow \frac{AD}{AB}=\frac{AE}{AC}$. \\
    三角形一边的平行线性质定理推论:平行与三角形的一边的直线截其他两边所在的直线,截得的三角形的三边与原三角形三边对应成比例。\\
    如图,即$DE\px\px BC \Rightarrow \frac{AD}{AB}=\frac{AE}{AC}=\frac{DE}{BC}.$ \\
    证明:过$E$作$EF\px\px AB$交直线$BC$于$F$,易证四边形$BDEF$为平行四边形,$DE=BF$.\\
    $DE \px \px BC \Rightarrow \frac{AD}{AB}=\frac{AE}{AC}$ \\
    $EF \px \px AB \Rightarrow \frac{AE}{AC}=\frac{BF}{BC}=\frac{DE}{BC}$
\end{knowledge}
\begin{figure}[H]
\centering
\includegraphics[width=6cm]{picture/804.png}
\caption{三角形一边的平行线}
\end{figure}
\begin{knowledge}
    三角形一边的平行线判定定理:如果一条直线截三角形的两边所得的对应线段成比例,那么这条直线平行于三角形的第三边。\\
    如左图,即$\frac{AD}{DB}=\frac{AE}{EC}\Rightarrow DE \px \px BC$. \\
    证明:过$C$作$CF \px \px AB$交直线$DE$于$F$,$\frac{AD}{CF}=\frac{AE}{EC}$,又$\frac{AD}{BD}=\frac{AE}{EC}\Rightarrow BD=CF$,
    故四边形$BDFC$为平行四边形,$DE\px \px BC$. \\
    注意:$\frac{AD}{AB}=\frac{DE}{BC} \nRightarrow DE\px \px BC.$ \\
    三角形一边的平行线判定定理推论:如果一条直线截三角形两边的延长线(这两边的延长线在第三边的同侧)所得的对应线段成比例,那么这条直线平行于三角形的第三边。\\
    如右图,即$\frac{AD}{AB}=\frac{AE}{AC}\Rightarrow DE \px \px BC.$
\end{knowledge}
\begin{problem}
    如图,梯形$ABCD,AD\px \px BC,BC=2AD,DE=EC$,求$\frac{AF}{CF}$和$\frac{EF}{BF}$的值。
\end{problem}
\begin{figure}[H]
\centering
\includegraphics[width=4cm]{picture/853.png}
\caption{练习图}
\end{figure}
\begin{problem}
    如图,$\frac{AM}{BM}=\frac{3}{2},\frac{CN}{BN}=\frac{4}{5},AD=CD$,求$\frac{DO}{BO}$的值。\\
\end{problem}
\begin{figure}[H]
\centering
\includegraphics[width=4cm]{picture/852.png}
\caption{练习图}
\end{figure}
\clearpage
\section{梯形中的比例线段}
\begin{model}
    如图,梯形$ACFD$中,$AD\px \px BE\px \px CF$,有如下结论成立:\\
    (1)$\frac{AB}{AC}=\frac{BE-AD}{CF-AD}=\frac{DE}{DF}$ \\
    (2)$\frac{AB}{BC}=\frac{BE-AD}{CF-BE}=\frac{DE}{EF}$ \\
    证明:过$D$作$AC$平行线交$BE$于$M$,交$CF$于$N$. \\
    (1)$\frac{AB}{AC}=\frac{DE}{DF}=\frac{DM}{DN}=\frac{ME}{NF}=\frac{BE-AD}{CF-AD}$ \\
    (2)$\frac{AB}{BC}=\frac{DE}{EF}=\frac{AB}{AC-AB}=\frac{BE-AD}{CF-AD-(BE-AD)}=\frac{BE-AD}{CF-BE}$
\end{model}
\begin{figure}[H]
\centering
\includegraphics[width=4cm]{picture/808.png}
\caption{梯形中的比例线段}
\end{figure}
\begin{problem}
    如上图,用3种方法证明结论(1).\\
    提示一:作$AM\perp BE,DN\perp BE$,作$BP \perp CF,EQ \perp CF$. \\
    提示二:延长$AE$交$CF$于$G$. \\
    提示三:延长$CA,FD$交于$G$.
\end{problem}
\begin{problem}
    如图,梯形$ABCD,AD\px \px EG \px \px BC$,设$EF=x,FG=y$,求$y$与$x$的函数关系式,并写出定义域。\\
\end{problem}
\begin{figure}[H]
\centering
\includegraphics[width=4cm]{picture/851.png}
\caption{练习图}
\end{figure}
\clearpage
\section{三平行模型}
\begin{model}
    如图,$AB\px \px EF\px \px CD$,有结论$\frac{1}{EF}=\frac{1}{AB}+\frac{1}{CD}$成立。\\
    证明:$\frac{EF}{AB}=\frac{FC}{BC},\frac{EF}{CD}=\frac{BF}{BC}\rightarrow \frac{EF}{AB}+\frac{EF}{CD}=\frac{BF+FC}{BC}=1$ \\
    推论:$\frac{1}{S_{\triangle EBC}}=\frac{1}{S_{\triangle ABC}}+\frac{1}{S_{\triangle DBC}} \\$
\end{model}
\begin{figure}[H]
\centering
\includegraphics[width=4cm]{picture/805.png}
\caption{三平行模型}
\end{figure}
\begin{problem}
    如图,在$\triangle ABC$中,$\angle BAC=120^\circ,AD$平分$\angle BAC$交$BC$于点$D$,求证:$\frac{1}{AD}=\frac{1}{AB}+\frac{1}{AC}.$ \\
\end{problem}
\begin{figure}[H]
\centering
\includegraphics[width=4cm]{picture/806.png}
\caption{练习图}
\end{figure}
\begin{problem}
    如图,梯形$ABCD$中,$AD\px \px BC$,$AC,BD$交于点$O$,过$O$作$AD$的平行线交$AB$于$E$,交$CD$于$F$,求证:$EF=\frac{2·AD·BC}{AD+BC}.$\\
\end{problem}
\begin{figure}[H]
\centering
\includegraphics[width=4cm]{picture/807.jpg}
\caption{练习图}
\end{figure}
\clearpage
\section{三角形内接矩形}
\begin{model}
    如图,四边形$DEFG$为$\triangle ABC$的内接矩形,$AM\perp BC$交$BC$于$M$,交$DG$于$N$,探究如下问题:\\
    (1)设$DG=x,DE=y$,探究$y$与$x$的函数关系式。\\
    $\frac{DE}{AM}=\frac{BD}{BA},\frac{DG}{BC}=\frac{AD}{AB}$ \\
    $\frac{DE}{AH}+\frac{DG}{BC}=\frac{BD+AD}{AB}=1$ \\
    $\frac{y}{AM}+\frac{x}{BC}=1$ \\
    (2)探究矩形$DEFG$的面积最大值。\\
    $S_{DEFG}=x·y=x·(-\frac{AM}{BC}·x+AM)=-\frac{AM}{BC}(x^2-BC·x+\frac{BC^2}{4})+\frac{AM·BC}{4}$ \\
    $x=\frac{BC}{2}$时,即$DG$为$\triangle ABC$中位线时取得最大值,此时面积为$\frac{AM·BC}{4}$,为三角形面积的一半。\\
    (3)若四边形$DEFG$为正方形,则边长为$\frac{BC·AM}{BC+AM}.$ \\
\end{model}
\begin{figure}[H]
\centering
\includegraphics[width=4cm]{picture/825.png}
\caption{三角形内接矩形}
\end{figure}
\begin{problem}
    如上图,$\triangle ABC$中,$AM=8,BC=12$,若内接矩形$DEFG$的面积为$\triangle ABC$面积的$\frac{3}{8}$,求矩形$DEFG$的周长。\\
\end{problem}
\begin{problem}
    如图,$\triangle ABC,\angle B<45^\circ,BC=6,AN\perp BC,AN=4$,直角梯形$DEFG$的底$EF$在边$BC$上,$EF=4$,
    $D,G$分别在$AB,AC$上,且$DG\px \px EF,GF\perp EF$,设$GF$长为$x$,直角梯形$DEFG$面积为$y$,求$y$关于$x$的函数关系式,并写出定义域。\\
\end{problem}
\begin{figure}[H]
\centering
\includegraphics[width=4cm]{picture/829.png}
\caption{三角形内接矩形}
\end{figure}
\clearpage
\section{线束定理}
\begin{model}
    线束定理:过一点的三条直线截两条平行线,截得的线段对应成比例。\\
    如图,$DE\px \px BC$,则有$\frac{DG}{BF}=\frac{GE}{FC},\frac{DG}{DE}=\frac{BF}{BC},\frac{DG}{GE}=\frac{BF}{FC}.$
\end{model}
\begin{figure}[H]
\centering
\includegraphics[width=4cm]{picture/817.png}
\caption{线束定理}
\end{figure}
\begin{problem}
    如图,已知正方形$ABCD$,点$E$在$CB$延长线上,联结$AE,DE$,$DE$与边$AB$交于点$F$,$EG\px \px BE$且与$AE$交于点$G.$ \\
    (1)求证:$GF=BF$; \\
    (2)在边$BC$上取一点$M$,使得$BM=BE$,联结$AM$交$DE$于点$O$,求证:$FO·ED=OD·EF.$ \\
\end{problem}
\begin{figure}[H]
\centering
\includegraphics[width=4cm]{picture/818.png}
\caption{练习图}
\end{figure}
\clearpage
\section{三角形的重心}
\begin{knowledge}
    三角形三条中线交于一点,一定在三角形内部,称此交点为重心。\\
    如图,$\triangle ABC$的三条中线$AD,BE,CF$交于点$G$,有如下结论成立:\\
    (1)重心到顶点的距离等于重心到对边中点距离的2倍,即$AG=2GD,BG=2GE,CG=2GF$. \\
    (2)$S_{\triangle AFG}=S_{\triangle BFG}=S_{\triangle BDG}=S_{\triangle CDG}=S_{\triangle CEG}=S_{\triangle AEG}$ \\
    现证明三条中线交于一点和结论(1): \\
    先令中线$BE,CF$交于点$G$,延长$AG$交$BC$于点$D$,设$EF$与$AD$交于$H$\\
    $EF\px \px BC,EF=\frac{1}{2}BC \rightarrow BG=2GE,CG=2GF.$ \\
    $HE\px \px BD,BG=2GE \rightarrow BD=2HE.$ \\
    $HE\px \px CD,AC=2AE \rightarrow CD=2HE.$ \\
    $BD=2HE=CD$,故$D$也为$BC$中点,命题得证。\\
\end{knowledge}
\begin{figure}[H]
\centering
\includegraphics[width=4cm]{picture/809.png}
\caption{三角形的重心}
\end{figure}
\begin{problem}
    下列结论中,正确的有:\\
    如图(A),若$G$为重心,$DE\px \px BC$,则$DE=\frac{2}{3}BC$. \\
    如图(B),若$G$为重心,则$S_{\triangle GBC}:S_{\triangle ABC}=1:9$. \\
    如图(C),若$G$为重心,$GE\perp BC,AF\perp BC$,则$GE:AF=1:3$. \\
    如图(D),若$G$为重心,$\triangle ABC$为等腰直角三角形,$GD\px \px AB$,则$DG:AB=1:3$. \\
\end{problem}
\begin{figure}[H]
\centering
\includegraphics[width=12cm]{picture/822.png}
\caption{练习图}
\end{figure}
\begin{problem}
    如图,$\triangle ABC$中,$\angle C=90^\circ,BC=6,AC=9$,平移$\triangle ABC$使其顶点$C$位于$\triangle ABC$的重心$G$,
    求平移后所得三角形与原三角形重叠部分的面积。
\end{problem}
\begin{figure}[H]
\centering
\includegraphics[width=4cm]{picture/824.png}
\caption{练习图}
\end{figure}
\begin{problem}
    新定义:我们把两条中线互相垂直的三角形称为"中垂三角形"。如图,$\triangle ABC$为中垂三角形,$AF,BE$为中线,若$\angle ABE=30^\circ,AB=4$,求$AC$的长。
\end{problem}
\begin{figure}[H]
\centering
\includegraphics[width=4cm]{picture/823.png}
\caption{练习图}
\end{figure}
\clearpage
\section{角平分线定理}
\begin{model}
    如图,$AM$为$\triangle BAC$的内角平分线,则有$\frac{AB}{AC}=\frac{BM}{MC}.$ \\
    证明一:$\frac{AB}{AC}=\frac{S_{\triangle ABM}}{S_{\triangle ACM}}=\frac{BM}{MC}$. \\
    证明二:作$CN\px \px AB$交$AM$于$N$,易得$AC=NC$; \\
    $\frac{AB}{AC}=\frac{AB}{NC}=\frac{BM}{MC}.$
\end{model}
\begin{figure}[H]
\centering
\includegraphics[width=4cm]{picture/854.png}
\caption{角平分线定理}
\end{figure}
\begin{problem}
    如图,$BD$为$\triangle ABC$的角平分线,点$E$位于边$BC$上,且$BD$是$BA$与$BE$的比例中项。求证:\\
    (1)$\angle CDE=\frac{1}{2}\angle ABC$ \\
    (2)$AD·CD=AB·CE$
\end{problem}
\begin{figure}[H]
\centering
\includegraphics[width=4cm]{picture/811.png}
\caption{练习图}
\end{figure}
\clearpage
\section{*梅涅劳斯定理}
\begin{model}
    如图,直线交$\triangle ABC$于$F,E,D$三点,则有$\frac{AF}{FB}·\frac{BD}{DC}·\frac{CE}{EA}=1$. \\
    证明一:作$CP\px \px FE$交$AB$于$P$. \\
    $\frac{BD}{DC}=\frac{FB}{PF},\frac{CE}{EA}=\frac{PF}{AF}$ \\
    证明二:连$FC,AD$. \\
    $\frac{AF}{BF}=\frac{S_{\triangle AFD}}{S_{\triangle BFD}}$ \\
    $\frac{BD}{DC}=\frac{S_{\triangle BFD}}{S_{\triangle CFD}}$ \\
    $\frac{CE}{EA}=\frac{S_{\triangle CED}+S_{\triangle CEF}}{S_{\triangle AED}+S_{\triangle AEF}}=\frac{S_{\triangle CFD}}{S_{\triangle AFD}}$ \\
    证明三:过$A,B,C$分别向直线$FED$作垂线$AA',BB',CC'$. \\
    $\frac{AF}{FB}·\frac{BD}{DC}·\frac{CE}{EA}=\frac{AA'}{BB'}·\frac{BB'}{CC'}·\frac{CC'}{AA'}=1$ \\
    选择$\triangle BFD$为梅式三角形,则$AEC$为梅式线,有$\frac{BA}{AF}·\frac{FE}{ED}·\frac{DC}{CB}=1$. \\
    梅涅劳斯定理的逆定理:$F,D,E$分别是$\triangle ABC$三边$AB,BC,CA$或其延长线的三点,若$\frac{AF}{FB}·\frac{BD}{DC}·\frac{CE}{EA}=1$,则$F,D,E$三点共线。\\
\end{model}
\begin{figure}[H]
\centering
\includegraphics[width=4cm]{picture/841.png}
\end{figure}
\begin{figure}[H]
\centering
\includegraphics[width=4cm]{picture/842.png}
\end{figure}
\begin{figure}[H]
\centering
\includegraphics[width=4cm]{picture/843.png}
\caption{梅涅劳斯定理}
\end{figure}
\clearpage
\section{*塞瓦定理}
\begin{model}
    如图,$\triangle ABC$的三个顶点与一点$O$的连线$AO,BO,CO$交对边或延长线于点$D,E,F$,则有$\frac{BD}{DC}·\frac{CE}{EA}·\frac{AF}{FB}=1$. \\
    证明一:直线$FOC,EOB$分别是$\triangle ABD,\triangle ACD$的梅式线,由梅式定理: \\
    $\frac{BC}{CD}·\frac{DO}{OA}·\frac{AF}{FB}=1,\frac{DB}{BC}·\frac{CE}{EA}·\frac{AO}{OD}=1$ \\
    两式相乘即证;\\
    证明二:根据燕尾定理:\\
    $\frac{AF}{FB}=\frac{S_{\triangle ACO}}{S_{\triangle BCO}}$ \\
    $\frac{BD}{DC}=\frac{S_{\triangle ABO}}{S_{\triangle ACO}}$ \\
    $\frac{CE}{EA}=\frac{S_{\triangle BCO}}{S_{\triangle BAO}}$ \\
    塞瓦定理的逆定理:点$D,E,F$分别在$\triangle ABC$的边$BC,CA,AB$或延长线上,且$\frac{BD}{DC}·\frac{CE}{EA}·\frac{AF}{FB}=1$,则$AD,BE,CF$交于一点(或平行). \\
\end{model}
\begin{figure}[H]
\centering
\includegraphics[width=4cm]{picture/844.png}
\caption{塞瓦定理}
\end{figure}
\clearpage
\section{相似三角形的判定}
\begin{knowledge}
    具有相同形状的图形叫做相似形。相似图形对应角相等,对应边成比例。\\
    根据定义可以判断,所有的等边三角形相似、所有的等腰直角三角形相似、所有的正方形相似、所有的圆相似。\\
    两个相似图形对应边的比叫做相似比,全等形也是相似形,且相似比为1:1. \\
    对应角相等,对应边成比例的三角形叫做相似三角形。
\end{knowledge}
\begin{table}[H]
\centering
\caption{相似三角形的判定}
\begin{tabular}{l|p{12cm}}
\hline
\hline
判定 & 描述 \\
\hline
预备定理 & 平行于三角形一边的直线和其他两边(或两边的延长线)相交,所截得的三角形与原三角形相似\\
\hline
两角相等 & 两角对应相等的两个三角形相似 \\
\hline
两边与夹角 & 两边对应成比例,且夹角相等的两个三角形相似 \\
\hline
三边 & 三边对应成比例的两个三角形相似 \\
\hline
直角三角形 & 一对锐角相等/两条直角边对应成比例/一条直角边与斜边对应成比例 \\
\hline
等腰三角形 & 一对顶角相等/一对底角相等/腰和底对应成比例 \\
\hline
\hline
\end{tabular}
\end{table}
\begin{problem}
    如图,在平行四边形$ABCD$中,$G$为$BC$延长线上一点,$AG$与$BD$交于点$E$,与$DC$交于点$F$,写出图中所有的相似三角形对。
\end{problem}
\begin{figure}[H]
\centering
\includegraphics[width=4cm]{picture/821.jpg}
\caption{练习图}
\end{figure}
\begin{problem}
    如图,$\angle AOD=90^\circ,OA=OB=BC=CD$,在图中找出一对相似三角形。
\end{problem}
\begin{figure}[H]
\centering
\includegraphics[width=4cm]{picture/847.png}
\caption{练习图}
\end{figure}
\begin{problem}
    如图,矩形$ABCD,CE:EB=1:2,BC:AB=3:4,AE\perp AF$,求$CO:OA$的值。
\end{problem}
\begin{figure}[H]
\centering
\includegraphics[width=4cm]{picture/848.png}
\caption{练习图}
\end{figure}
\begin{problem}
    如图,$OC$平分$\angle MON,\angle MON=50^\circ,OA·OB=OP^2$,求$\angle APB$的度数。
\end{problem}
\begin{figure}[H]
\centering
\includegraphics[width=4cm]{picture/849.png}
\caption{练习图}
\end{figure}
\begin{problem}
    如图,梯形$ABCD$中,$DC\px \px AB,AD=BD,AD\perp DB,\angle EBC=45^\circ$,求证:$\triangle ECB$为等腰直角三角形。
\end{problem}
\begin{figure}[H]
\centering
\includegraphics[width=4cm]{picture/835.png}
\caption{练习图}
\end{figure}
\begin{problem}
    如图,直角梯形$ABCD,\angle D=\angle C=90^\circ,AD=3,BC=12$,在$CD$上有点$P$,使得$\triangle PAD$和$\triangle PBC$相似,探究如下问题:\\
    (1)若$CD=12$,求所有可能的$PD$的长; \\
    (2)若$CD=13$,求所有可能的$PD$的长; \\
    (3)若$CD=10$,求所有可能的$PD$的长; \\
    (4)探究$P$点个数和$CD,AD,BC$长度的关系。
\end{problem}
\begin{figure}[H]
\centering
\includegraphics[width=4cm]{picture/840.png}
\caption{练习图}
\end{figure}
\begin{problem}
    如图,$\triangle ACB$中,$\angle C=90^\circ,AC=3,BC=4$,设$P,Q$分别为$AB,BC$上的点,$P$从$A$沿$AB$方向向$B$匀速运动,
    $Q$从$B$沿$BC$方向向$C$匀速运动,移动速度均为1每秒。当$Q$到达$C$时,$P$点也停止运动。设运动时间为$t$秒:\\
    (1)求$S_{\triangle PBQ}$与时间$t$的函数表达式,并写出定义域;\\
    (2)求$t$为何值时$\triangle PBQ$为等腰三角形;\\
    (3)求$t$为何值时$\triangle PBQ$与$\triangle ABC$相似。
\end{problem}
\begin{figure}[H]
\centering
\includegraphics[width=4cm]{picture/826.png}
\caption{练习图}
\end{figure}
\begin{problem}
    如图,矩形$ABCD,AB=3,BC=4$,$E$是射线$CB$上的动点,$F$是射线$CD$上一点,且$AF\perp AE$,射线$EF$与对角线$BD$交于$G$,与射线$AD$交于$M$. \\
    (1)当$E$在线段$BC$上时,求证:$\triangle AEF \sim \triangle ABD$; \\
    (2)在(1)的条件下,联结$AG$,设$BE=x,\tan \angle MAG=y$,求$y$关于$x$的函数解析式,并写出定义域;\\
    (3)当$\triangle AGM $与$\triangle ADF $相似时,求$BE$的长。
\end{problem}
\begin{figure}[H]
\centering
\includegraphics[width=8cm]{picture/846.png}
\caption{练习图}
\end{figure}
\clearpage
\section{相似三角形的性质}
\begin{table}[H]
\centering
\caption{相似三角形的性质}
\begin{tabular}{l|p{8cm}}
\hline
\hline
形状 & 相同 \\
\hline
大小 & 不一定相同 \\
\hline
对应边 & 成比例 \\
\hline
对应角 & 相等 \\
\hline
对应中线、高线、角平分线 & 成比例 \\
\hline
周长 & 周长比等于相似比 \\
\hline
面积 & 面积比等于相似比的平方 \\
\hline
\hline
\end{tabular}
\end{table}
\begin{problem}
    如图,边长为1的25个小正方形组成的网格中有一个与$\triangle ABC$相似且面积最大的$\triangle A_1B_1C_1$,
    且它的三个顶点都落在小正方形的顶点上,求$\triangle A_1B_1C_1$的面积。
\end{problem}
\begin{figure}[H]
\centering
\includegraphics[width=4cm]{picture/827.png}
\caption{练习图}
\end{figure}
\begin{problem}
    如图,$\frac{BD}{AD}=\frac{AE}{CE}=3,\angle AED=\angle B$,求$\frac{S_{\triangle AED}}{S_{\triangle ABC}}$的值。\\
\end{problem}
\begin{figure}[H]
\centering
\includegraphics[width=4cm]{picture/833.png}
\caption{练习图}
\end{figure}
\begin{problem}
    边长为2、3、5的正方形如图排列,求阴影部分的面积。
\end{problem}
\begin{figure}[H]
\centering
\includegraphics[width=4cm]{picture/834.png}
\caption{练习图}
\end{figure}
\begin{problem}
    如图,$BD,CE$是$\triangle ABC$的两条高,$AM$是$\angle BAC$的角平分线,交$BC$于点$M$,交$DE$于点$N$,求证:\\
    (1)$\frac{AM}{AN}=\frac{BC}{DE}$ \\
    (2)$\angle EDB=\angle ECB$ \\
\end{problem}
\begin{figure}[H]
\centering
\includegraphics[width=4cm]{picture/816.png}
\caption{练习图}
\end{figure}
\begin{problem}
    如图,四边形$ABCD$中,$AD\px \px BC,\frac{FG}{DG}=\frac{AD}{CE}.$ \\
    (1)求证:$AB \px \px CB$; \\
    (2)若$AD^2=DG·DE$,求证:$\frac{EG^2}{CE^2}=\frac{AG}{AC}.$ \\
\end{problem}
\begin{figure}[H]
\centering
\includegraphics[width=4cm]{picture/837.png}
\caption{练习图}
\end{figure}
\clearpage
\section{隐圆与相似}
%\begin{knowledge}
%    四点共圆的判定方法:\\
%    (1)若四个点到一定点的距离相等,则这四个点共圆。\\
%    (2)若一个四边形的一组对角的和等于180度,则这个四边形四个顶点共圆。\\
%    (3)若一个四边形的一个外角等于它的内对角,则这个四边形四个顶点共圆。\\
%    (4)若两个点在一条线段的同旁,并且和这条线段的两端连线所夹的角相等,那么这两点和这条线的两个端点共圆。
%\end{knowledge}
%\begin{figure}[H]
%\centering
%\includegraphics[width=4cm]{picture/862.png}
%\caption{四点共圆}
%\end{figure}
%\begin{knowledge}
%    圆周角定理:一条弧所对的圆周角等于它所对的圆心角的一半,同弧或等弧所对的圆周角相等。如图,有$\angle BAC=\frac{1}{2}\angle BOC$.
%\end{knowledge}
%\begin{figure}[H]
%\centering
%\includegraphics[width=6cm]{picture/864.png}
%\caption{圆周角定理}
%\end{figure}
%\begin{knowledge}
%    圆幂定理:交点为$P$的两条相交直线与圆$O$相交于$A,B$与$C,D$,则有$PA·PB=PC·PD$,证明方法:$\triangle PAD \sim \triangle PCB$
%    或$\triangle PAC \sim \triangle PDB$. \\
%    (1)相交弦定理:圆内的两条弦$AB,CD$交于$P$.\\
%    (2)割线定理:从圆外一点$P$引圆的两条割线$AB,CD$.\\
%    (3)切割线定理:从圆外一点$P$引圆的一条切线$PA(B)$和一条割线$CD$. \\
%    (4)切线长定理:从圆外一点$P$引圆的两条切线$PA(B),PC(D)$,此时$PA(B)=PC(D)$.
%\end{knowledge}
%\begin{figure}[H]
%\centering
%\includegraphics[width=8cm]{picture/863.png}
%\caption{圆幂定理}
%\end{figure}
\begin{model}
    如图,在$\triangle ABC$中,$\angle ADE=\angle C$,则$D,E,C,B$四点共圆,$\triangle ADE \sim \triangle ACB.$ \\
    根据相似,有结论$AD·AB=AE·AC$成立(割线定理)。 \\
    特别的,若点$E$和点$C$重合,则$AC$为$\triangle BDC$外接圆的切线,有结论$AD·AB=AC^2$成立(切割线定理)。 \\
\end{model}
\begin{figure}[H]
\centering
\includegraphics[width=3cm]{picture/819.png}
\end{figure}
\begin{figure}[H]
\centering
\includegraphics[width=3cm]{picture/820.png}
\caption{反A字型}
\end{figure}
\begin{problem}
    点$P$是$\triangle ABC$中$AB$边上一点,过点$P$作直线(不与直线$AB$重合)截$\triangle ABC$,使得到的三角形与原三角形相似,满足条件的直线最多有多少条?
\end{problem}
\begin{problem}
    如图,$\triangle ABC$中,$AB=AC$,$\angle ADE=\angle B,\angle ADF=\angle C,EF$交$AD$于$G$. \\
    (1)求证:$AE=AF$; \\
    (2)若$\frac{DF}{DE}=\frac{CF}{AE}$,求证:四边形$EBDF$为平行四边形。 \\
\end{problem}
\begin{figure}[H]
\centering
\includegraphics[width=4cm]{picture/830.png}
\caption{练习图}
\end{figure}
\begin{problem}
    如图(1),$\triangle ABC$中,$CA^2=CD·CB.$ \\
    (1)求证:$\frac{AD}{AB}=\frac{AC}{BC}$; \\
    (2)如图(2),延长$AC$,取$AE=AB$,联结$BE$,延长$AD$交$BE$于点$F$,求证:$\frac{EF}{BF}=\frac{AD}{BD}.$ \\
\end{problem}
\begin{figure}[H]
\centering
\includegraphics[width=8cm]{picture/850.png}
\caption{练习图}
\end{figure}
\begin{model}
    如图,$\angle ABD=\angle ACD$,则$A,B,C,D$四点共圆,图中有若干对相似三角形,如下:\\
    (1)$\triangle AOB \sim \triangle DOC (\angle ABO=\angle DCO, \angle AOB=\angle DOC)$ \\
    (2)$\triangle AOD \sim \triangle BOC (\triangle AOB \sim \triangle DOC \Rightarrow \frac{AO}{BO}=\frac{DO}{CO}, \angle AOB=\angle DOC)$ \\
    (3)$\triangle PDB \sim \triangle PAC (\angle DPB=\angle APC, \angle PBD=\angle PCA)$ \\
    (4)$\triangle PAD \sim \triangle PCB (\triangle PDB \sim \triangle PAC \Rightarrow \frac{PA}{PC}=\frac{PD}{PB}, \angle APD=\angle CPB)$ \\
    根据相似,可以推出如下结论: \\
    (1)$PA·PB=PD·PC$(割线定理) \\
    (2)$OA·OC=OB·OD$(相交弦定理) \\
    根据同弧所对的圆周角相等,还可以推出如下结论:\\
    (1)$\angle ADB=\angle ACB$ \\
    (2)$\angle DAC=\angle DBC$ \\
    (3)$\angle BAC=\angle BDC$ \\
\end{model}
\begin{figure}[H]
\centering
\includegraphics[width=4cm]{picture/828.png}
\caption{四点共圆图}
\end{figure}
\begin{example}
    如图,正方形$ABCD$边长为$a$,$E,F$为$BC,CD$上两动点,且满足$\angle EAF=45^\circ$,连$BD$交$AE,AF$于$M,N$. \\
    (1)写出所有与$\triangle AMN$相似的三角形 \\
    (2)求证:$S_{\triangle AMN}=\frac{1}{2}S_{\triangle AFE}$ \\
    (3)求证:$AE$平分$\angle BEF$,$AF$平分$\angle DFE$\\
    解 \\
    (1)$\triangle AMN \sim \triangle DMA \sim \triangle BAN \sim \triangle BME \sim \triangle DFN \sim \triangle AFE$ \\
    反A字型:$\triangle AMN \sim \triangle DMA \sim \triangle BAN(\angle MAN=\angle MDA=\angle ABN,\angle AMN=\angle DMA,\angle ANM=\angle BNA)$ \\
    蝴蝶型:$\triangle AMN \sim \triangle BME \sim \triangle DFN(\angle MAN=\angle MBE=\angle FDN,\angle AMN=\angle BME,\angle ANM=\angle DNF)$ \\
    证明$\triangle AMN \sim \triangle AFE$ \\
    $\triangle AMN\sim\triangle BME\Rightarrow\frac{AM}{BM}=\frac{MN}{ME}\Rightarrow\frac{AM}{NM}=\frac{MB}{ME},\angle AMB=\angle NME\Rightarrow\triangle AMB\sim\triangle NME\Rightarrow\angle MEN=\angle MBA=45^\circ$ \\
    $\triangle AMN\sim\triangle DFN\Rightarrow\frac{AN}{DN}=\frac{MN}{FN}\Rightarrow\frac{AN}{MN}=\frac{ND}{NF},\angle AND=\angle MNF\Rightarrow\triangle AND\sim\triangle MNF\Rightarrow\angle NFM=\angle NDA=45^\circ$ \\
    $\triangle ANE,\triangle AMF$均为等腰直角三角形,$\triangle ANE \sim \triangle AMF \Rightarrow \frac{AN}{AM}=\frac{AE}{AF} \Rightarrow \frac{AN}{AE}=\frac{AM}{AF},\angle MAN=\angle FAE\Rightarrow \triangle AMN \sim \triangle AFE$ \\
    证明(2) \\
    $\frac{S_{\triangle AMN}}{S_{\triangle AFE}}=\frac{AN^2}{AE^2}=\frac{1}{2}$ \\
    证明(3) \\
    $\triangle AFE \sim \triangle AMN \sim \triangle DFN \Rightarrow \angle AFE=\angle DFN$ \\
    $\triangle AFE \sim \triangle AMN \sim \triangle BME \Rightarrow \angle AEF=\angle BEM$ \\
    \\
\end{example}
\begin{figure}[H]
\centering
\includegraphics[width=4cm]{picture/6104.png}
\caption{半角模型}
\end{figure}
\begin{problem}
    如图,$\triangle ABC$中,$DE,BC$的延长线交于$F$,且$EF·DF=BF·CF$. \\
    (1)求证:$AD·AB=AE·AC$; \\
    (2)$AB=12,AC=9,AE=8$,求$BD$的长和$\frac{S_{\triangle ADE}}{S_{\triangle ECF}}$的值。\\
\end{problem}
\begin{figure}[H]
\centering
\includegraphics[width=4cm]{picture/836.png}
\caption{练习图}
\end{figure}
\begin{problem}
    如图,$\angle BAC=\angle BDC=\angle DAE$,求证:\\
    (1)$\triangle ABE \sim \triangle ACD$ \\
    (2)$BC·AD=DE·AC$ \\
\end{problem}
\begin{figure}[H]
\centering
\includegraphics[width=4cm]{picture/838.png}
\caption{练习图}
\end{figure}
\begin{problem}
    如图,$PA=PB,\angle APB=2\angle ACB,PB=4,PD=3$,求$AD·DC$的值。
\end{problem}
\begin{figure}[H]
\centering
\includegraphics[width=4cm]{picture/857.png}
\caption{练习图}
\end{figure}
\clearpage
\section{三角形的垂心}
\begin{model}
    如图,$\triangle BAC$中,$AD\perp BC,BE\perp AC,CF\perp AB$,则$AD,BE,CF$三线交于一点$O$,该点被称作三角形的垂心。\\
    证明:在$\triangle BAC$中,先作$CF\perp AB$交$AB$于$F$,再作$BE\perp AC$于$E$,$CF,BE$交于$O$,延长$AO$交$BC$于$D$. \\
    易证$\triangle AFC\sim \triangle OFB$,则有$\frac{AF}{OF}=\frac{FC}{FB}.$ \\
    $\frac{AF}{FC}=\frac{OF}{FB},\angle AFG=\angle CFB\rightarrow \triangle AFO \sim \triangle  CFB.$ \\
    $\angle FAO=\angle FCB, \angle FBC+\angle FCB=90^\circ \rightarrow \angle FBC+\angle FAO=90^\circ \rightarrow AD\perp BC.$ \\
    图中有若干组相似三角形,总结如下: \\
    (1)$\triangle OEC \sim \triangle AFC \sim \triangle OFB \sim \triangle AEB $ \\
    (2)$\triangle ODC \sim \triangle BFC \sim \triangle OFA \sim \triangle BDA $ \\
    (3)$\triangle ODB \sim \triangle CEB \sim \triangle OEA \sim \triangle CDA $ \\
\end{model}
\begin{figure}[H]
\centering
\includegraphics[width=4cm]{picture/814.png}
\caption{三角形的垂心}
\end{figure}
\begin{problem}
    如图,锐角$\triangle ABC$中,$CE\perp AB$于点$E$,点$D$在边$AC$上,联结$BD$交$CE$于$F$,且$EF·FC=FB·DF$. \\
    (1)求证:$BD\perp AC$; \\
    (2)联结$AC$,求证:$AF·BE=BC·EF$; \\
    (3)求证:$\angle AED=\angle ACB$. \\
\end{problem}
\begin{figure}[H]
\centering
\includegraphics[width=4cm]{picture/815.png}
\caption{练习图}
\end{figure}
\clearpage
\section{一线三等角}
\begin{model}
    如图,$\angle ABC=\angle CAE=\angle ADE$,有如下结论成立:\\
    (1)$\triangle ABC\sim \triangle EDA$ \\
    (2)$\frac{AB}{ED}=\frac{BC}{DA}=\frac{AC}{EA}$ \\
    (3)$BA=AD \Leftrightarrow \triangle ABC \sim \triangle EAC \sim \triangle EDA$ \\
    证明:$\frac{BC}{DA}=\frac{AC}{EA}\Rightarrow \frac{BC}{AC}=\frac{DA}{EA}=\frac{BA}{EA}$
\end{model}
\begin{figure}[H]
\centering
\includegraphics[width=9cm]{picture/858.png}
\caption{一线三等角}
\end{figure}
\begin{model}
    如图,$AB=AC,\angle ADE=\angle B$,则有$\triangle BAD \sim CDE,\triangle ADE \sim \triangle ACD.$
\end{model}
\begin{figure}[H]
\centering
\includegraphics[width=4cm]{picture/831.png}
\caption{一线三等角}
\end{figure}
\begin{problem}
    如上图,若$AB=AC=10,BC=16$,求解如下问题:\\
    (1)求$AE$长度的取值范围;\\
    (2)若$\triangle ABD \backcong \triangle DCE$,求$AD$的长;\\
    (3)当$\triangle DCE$为直角三角形时,求$BD$的长。
\end{problem}
\begin{problem}
    如图,在$\triangle ABC$中,$D$在线段$AB$上运动,$F$在线段$AC$上,$AB=AC=5,BC=6,\angle DEF=\angle B,DE\perp AB$,
    是否可能$S_{\triangle FCE}=4S_{\triangle EBD}$?若可能求出$BD$的长,若不可能,说明理由。
\end{problem}
\begin{figure}[H]
\centering
\includegraphics[width=4cm]{picture/832.png}
\caption{练习图}
\end{figure}
\begin{problem}
    如图,$\triangle ABC,\triangle ADE$均为等边三角形,$AB=9,BD=3$. \\
    (1)求$CF$的长;\\
    (2)求$AD$的长;\\
    (3)求$DF$的长。\\
\end{problem}
\begin{figure}[H]
\centering
\includegraphics[width=4cm]{picture/839.png}
\caption{练习图}
\end{figure}
\begin{problem}
    如上图,$\triangle ABC$为边长为1的等边三角形,$D$是线段$BC$上一个动点(不含端点),$F$是线段$AC$上的点,且$\angle ADF=60^\circ$,求$CF$的最大值。
\end{problem}
\clearpage
\section{旋转相似}
\begin{model}
    如图,$\triangle ABC\sim \triangle ADE \Leftrightarrow \triangle ABD \sim \triangle ACE$. \\
    $\triangle ABC\sim \triangle ADE$可以看作将$\triangle ABC$绕着点$A$旋转一定角度$(\angle BAD=\angle CAE)$,再进行放缩$\frac{AD}{AB}=\frac{AE}{AC}$倍得到$\triangle ADE$,$D$为$B$的对应点,$E$为$C$的对应点。\\
    $\triangle ABD\sim \triangle ACE$可以看作将$\triangle ABD$绕着点$A$旋转一定角度$(\angle BAC=\angle DAE)$,再进行放缩$\frac{AC}{AB}=\frac{AE}{AD}$倍得到$\triangle ACE$,$C$为$B$的对应点,$E$为$D$的对应点。\\
    $\triangle ABC\sim \triangle ADE \Rightarrow \frac{AB}{AD}=\frac{AC}{AE} \Rightarrow \frac{AB}{AC}=\frac{AD}{AE},\angle BAD=\angle CAE \Rightarrow \triangle ABD \sim \triangle ACE.$ \\
    $\triangle ABD\sim \triangle ACE \Rightarrow \frac{AB}{AC}=\frac{AD}{AE} \Rightarrow \frac{AB}{AD}=\frac{AC}{AE},\angle BAC=\angle DAE \Rightarrow \triangle ABC \sim \triangle ADE.$
\end{model}
\begin{figure}[H]
\centering
\includegraphics[width=4cm]{picture/855.png}
\caption{练习图}
\end{figure}
\begin{problem}
    如图,大正方形$ABCD$和小正方形$AEFG$共顶点,求$BE:CF:DG$的值。
\end{problem}
\begin{figure}[H]
\centering
\includegraphics[width=3cm]{picture/856.png}
\caption{练习图}
\end{figure}
\begin{problem}
    如图,$\triangle ABC$中,$\angle BAC=90^\circ,AD\perp BC,EF\perp AB,EG\perp AC$. \\
    (1)求证:$FD\perp DG$; \\
    (2)若$AB=AC$,求证:$\triangle FDG$为等腰直角三角形。
\end{problem}
\begin{figure}[H]
\centering
\includegraphics[width=4cm]{picture/845.png}
\caption{练习图}
\end{figure}
\begin{problem}
    如图,$Rt\triangle ABC$中,$\angle ACB=90^\circ,AB=5,BC=3$,点$D$是斜边$AB$上任意一点,联结$DC$,过点$C$作$CE\perp CD$,
    垂足为点$C$,联结$DE$,使得$\angle EDC=\angle A$,联结$BE$. \\
    (1)求证:$AC·BE=BC·AD$; \\
    (2)设$AD=x$,四边形$BDCE$面积为$S$,求$S$与$x$之间的函数关系式及$x$的取值范围。
\end{problem}
\begin{figure}[H]
\centering
\includegraphics[width=4cm]{picture/861.png}
\caption{练习图}
\end{figure}
\begin{problem}
    如图,在$\triangle ABC$中,$AB=AC,AE=BE,AD=CD,DF\perp AC$. \\
    (1)求证:$AD^2=DG·BD$ \\
    (2)求证:$\triangle BCE \sim \triangle GCD$ \\
    (3)求证:$\triangle BCG \sim \triangle ECD$
\end{problem}
\begin{figure}[H]
\centering
\includegraphics[width=4cm]{picture/865.png}
\caption{练习图}
\end{figure}
\clearpage
\section{射影定理}
\begin{model}
    如图,$\triangle CAB,\angle ACB=90^\circ,CD\perp AB$交$AB$于点$D$,有如下结论成立:\\
    (1)$\triangle ACD \sim \triangle ABC \sim \triangle CBD$ \\
    (2)$AC^2 = AD·AB(\triangle ACD \sim \triangle ABC)$ \\
    (3)$BC^2 = BD·BA(\triangle ABC \sim \triangle CBD)$ \\
    (4)$CD^2 = AD·BD(\triangle ACD \sim \triangle CBD)$ \\
    (5)$\frac{AC^2}{BC^2}=\frac{AD}{BD}$ \\
    (6)$CD·AB = AC·BC$
\end{model}
\begin{figure}[H]
\centering
\includegraphics[width=4cm]{picture/812.png}
\caption{射影定理}
\end{figure}
\begin{problem}
    如图,$P$是正方形$ABCD$的边$BC$上一点,$BP=3PC$,$M$是$CD$的中点,$MN\perp AP$于点$N$,求证:$MN^2=AN·PN.$
\end{problem}
\begin{figure}[H]
\centering
\includegraphics[width=4cm]{picture/813.png}
\caption{练习图}
\end{figure}
\clearpage
\section{实数与向量相乘与向量的线性运算}
\begin{knowledge}
    设$k$是一个实数,$\vec{a}$是向量,则$k$与$\vec{a}$相乘所得的积是一个向量,记作$k\vec{a}$. \\
    (1)若$k \ne 0$且$\vec{a}\ne \vec{0}$,则$k\vec{a}$的长度$|k\vec{a}|=|k||\vec{a}|$.
    当$k>0$时$k\vec{a}$与$\vec{a}$同方向;当$k<0$时$k\vec{a}$与$\vec{a}$反方向。 \\
    (2)若$k = 0$或$\vec{a}=\vec{0}$,则$k\vec{a}=\vec{0}$.
\end{knowledge}
\begin{knowledge}
    设$m,n$为实数,实数相乘满足如下运算律:\\
    (1)$m(n\vec{a})=(mn)\vec{a}$ \\
    (2)$(m+n)\vec{a}=m\vec{a}+n\vec{a}$ \\
    (3)$m(\vec{a}+\vec{b})=m\vec{a}+m\vec{b}$
\end{knowledge}
\begin{knowledge}
    若向量$\vec{b}$与非零向量$\vec{a}$平行,则存在唯一的实数$m$,使得$\vec{b}=m\vec{a}$.
\end{knowledge}
\begin{knowledge}
    长度为1的向量叫做单位向量,设$\vec{e}$为单位向量,则$|\vec{e}|=1$.单位向量有无数个,不同的单位向量方向不同。\\
    对于任意非零向量$\vec{a}$,与它同方向的单位向量记作$\vec{a_0}$,则$\vec{a}=|\vec{a}|\vec{a_0},\vec{a_0}=\frac{1}{|\vec{a}|}\vec{a}$.
\end{knowledge}
\begin{knowledge}
    向量加法、减法、实数与向量相乘以及它们的混合运算叫做向量的线性运算。\\
    一般来说,如果$\vec{a},\vec{b}$是两个不平行的向量,$\vec{c}$是平面内的一个向量,则$\vec{c}$可以用$\vec{c}=x\vec{a}+y\vec{b}$来表示,其中$x,y$为实数。
\end{knowledge}
\begin{knowledge}
    如果$\vec{a},\vec{b}$是两个不平行的向量,$\vec{c}=m\vec{a}+n\vec{b}$,其中$m,n$为实数,
    则$\vec{c}$是$m\vec{a},n\vec{b}$的合成,也可以说$\vec{c}$分解为$m\vec{a},n\vec{b}$两个向量,
    此时$m\vec{a},n\vec{b}$是$\vec{c}$分别在$\vec{a},\vec{b}$方向上的分向量,$m\vec{a}+n\vec{b}$是$\vec{c}$关于$\vec{a},\vec{b}$的分解式。
\end{knowledge}
\begin{knowledge}
    在$\triangle ABC$中,若$AD$为$\triangle ABC$中线,$G$为重心,则有如下结论成立:\\
    (1)$\overrightarrow{AD}=\frac{1}{2}\overrightarrow{AB}+\frac{1}{2}\overrightarrow{AC}$ \\
    (2)$\overrightarrow{AG}=\frac{1}{3}\overrightarrow{AB}+\frac{1}{3}\overrightarrow{AC}$ \\
    (3)$\overrightarrow{GA}+\overrightarrow{GB}+\overrightarrow{GC}=\vec{0}$
\end{knowledge}
\clearpage
\chapter{锐角的三角比}
\section{锐角的三角比及其性质}
\begin{knowledge}
    如图,$\triangle ACB,\angle C=90^\circ$. \\
    正弦:$\sin A=\frac{a}{c}$ \\
    余弦:$\cos A=\frac{b}{c}$ \\
    正切:$\tan A=\frac{a}{b}$ \\
    余切:$\cot A=\frac{b}{a}$ \\
    锐角三角比有如下性质:\\
    (1)$0 < \sin A < 1, 0 < \cos A < 1, \tan A > 0, \cot A > 0$ \\
    (2)当$0^\circ < A < 90^\circ$时,正弦值、正切值随角度增大而增大;余弦值、余切值随角度增大而减小。\\
    (3)$\sin^2 A+\cos^2 A=1$ \\
    (4)$\tan A=\frac{\sin A}{\cos A}$ \\
    (5)$\tan A·\cot A=1$ \\
    (6)$\sin A=\cos(90^\circ-A)$ \\
    (7)$\cos A=\sin(90^\circ-A)$ \\
    (8)$\tan A=\cot(90^\circ-A)$ \\
    (9)$\cot A=\tan(90^\circ-A)$ \\
\end{knowledge}
\begin{figure}[H]
\centering
\includegraphics[width=4cm]{picture/901.png}
\caption{锐角三角比}
\end{figure}
\clearpage
\section{特殊角的三角比}
\begin{knowledge}
    特殊三角比如下表所示,其中$15^\circ,22.5^\circ,75^\circ$只作了解:
\end{knowledge}
\begin{table}[H]
\centering
\caption{特殊角的三角比}
\begin{tabular}{l|l|l|l|l|l|l|l|l}
\hline
\hline
三角比 & $0^\circ$ & $30^\circ$ & $45^\circ$ & $60^\circ$ & $90^\circ$ & $15^\circ$ & $22.5^\circ$ & $75^\circ$ \\
\hline
$\sin A$ & 0 & $\frac{1}{2}$ & $\frac{\sqrt 2}{2}$ & $\frac{\sqrt3}{2}$ & 1 & $\frac{\sqrt 6-\sqrt 2}{4}$ & ... & $\frac{\sqrt 6+\sqrt 2}{4}$\\
\hline
$\cos A$ & 1 & $\frac{\sqrt3}{2}$ & $\frac{\sqrt 2}{2}$ & $\frac{1}{2}$ & 0 & $\frac{\sqrt 6+\sqrt 2}{4}$ & ... & $\frac{\sqrt 6-\sqrt 2}{4}$\\
\hline
$\tan A$ & 0 & $\frac{\sqrt3}{3}$ & 1 & $\sqrt3$ & $\times$ & $2-\sqrt 3 $ & $\sqrt2 - 1$ & $2+\sqrt 3$\\
\hline
$\cot A$ & $\times$ & $\sqrt3$ & 1 & $\frac{\sqrt3}{3}$ & 0 & $2+\sqrt 3 $ & $\sqrt2 + 1$ & $2-\sqrt 3$\\
\hline
\hline
\end{tabular}
\end{table}
\begin{knowledge}
    现给出计算15度角三角比的方法,22.5度角的三角比可类比此法求得。\\
    如图,$\angle A=15^\circ,\angle C=90^\circ$,在$AC$上取一点$D$,使得$AD=BD$,则$\angle BDC=30^\circ$,设$BC=a$,则$CD=\sqrt3 a,BD=2a,AD=2a,AC=2a+\sqrt3 a,AB=\sqrt{a^2+(2+\sqrt3)^2a^2}=\sqrt{(8+4\sqrt3)a^2}=(\sqrt6+\sqrt2)a$ \\
    $\sin A=\frac{BC}{AB}=\frac{a}{(\sqrt6+\sqrt2)a}=\frac{\sqrt6-\sqrt2}{4}$ \\
    $\cos A=\frac{AC}{AB}=\frac{(2+\sqrt3)a}{(\sqrt6+\sqrt2)a}=\frac{\sqrt6+\sqrt2}{4}$ \\
    $\tan A=\frac{BC}{AC}=\frac{a}{(2+\sqrt3)a}=2-\sqrt3 $ \\
    $\cot A=\frac{AC}{BC}=\frac{(2+\sqrt3)a}{a}=2+\sqrt3 $ \\
\end{knowledge}
\begin{figure}[H]
\centering
\includegraphics[width=4cm]{picture/906.png}
\caption{15度角的三角比计算}
\end{figure}
\clearpage
\section{解直角三角形}
\begin{knowledge}
    直角三角形中,除直角外,有5个元素:3条边和2个锐角,由除直角外已知元素求出所有未知元素的过程叫做解直角三角形。
    解直角三角形,除直角外必须具备两个元素,其中至少一条边。
\end{knowledge}
\begin{table}[H]
\centering
\caption{解直角三角形}
\begin{tabular}{l|l}
\hline
\hline
已知条件 & 解法 \\
\hline
斜边$c$和锐角$\angle A$ & $\angle B=90^\circ-\angle A,a=c·\sin A,b=c·\cos A$ \\
直角边$a$和锐角$\angle A$ & $\angle B=90^\circ-\angle A,b=\frac{a}{\tan A},c=\frac{a}{\sin A}$ \\
直角边$a$和锐角$\angle B$ & $\angle A=90^\circ-\angle B,b=a·\tan B,c=\frac{a}{\cos B}$ \\
直角边$a$和$b$ & $c=\sqrt{a^2+b^2},\angle A=\arctan \frac{a}{b},\angle B=90^\circ-\angle A$ \\
斜边$c$和直角边$a$ & $b=\sqrt{c^2-a^2},\angle A=\arcsin \frac{a}{c},\angle B=90^\circ-\angle A$ \\
\hline
\hline
\end{tabular}
\end{table}
\begin{figure}[H]
\centering
\includegraphics[width=3cm]{picture/901.png}
\caption{解直角三角形}
\end{figure}
\begin{knowledge}
    如图,视线与水平线所形成的角中,视线在水平线上方的叫做仰角,在水平线下方的叫做俯角。
\end{knowledge}
\begin{figure}[H]
\centering
\includegraphics[width=3cm]{picture/904.png}
\caption{俯角与仰角}
\end{figure}
\begin{knowledge}
    如图,坡面的垂直高度$h$和水平宽度$l$的比叫做坡度(或坡比),字母表示为$i=\frac{h}{l}$,
    坡面与水平面夹角为坡角,记作$\alpha$,有$\frac{h}{l}=\tan \alpha$. 坡度越大,坡面越陡。\\
\end{knowledge}
\begin{figure}[H]
\centering
\includegraphics[width=3cm]{picture/905.png}
\caption{坡角和坡度}
\end{figure}
\clearpage
\section{*直线斜率与倾斜角正切值}
\begin{knowledge}
    直线$y=kx+b$上有两点$A(x_1,y_1),B(x_2,y_2)$,$k$称作直线的斜率,$x$轴正轴与直线向上方向的夹角记作$\alpha(0^\circ \le \alpha < 180^\circ,\alpha \ne 90^\circ)$,称作直线的倾斜角,
    探究$\alpha$与$k$的关系:\\
    $k=\frac{y_1-y_2}{x_1-x_2}=\frac{\Delta y}{\Delta x}=\tan \alpha$ \\
    当$\alpha=90^\circ$时,直线垂直于$y$轴,$y_1=y_2$,此时$k=0,\tan \alpha=0$;\\
    当$0^\circ<\alpha<90^\circ$时,$k>0,\tan \alpha>0$; \\
    当$90^\circ<\alpha<180^\circ$时,$k<0,\tan \alpha<0$; \\
    注意:当$\alpha=90^\circ$时,直线垂直于$x$轴,$x_1=x_2$,此时$k$不存在,$\tan \alpha$也不存在。\\
    总结:倾斜角不是$90^\circ$的直线,其倾斜角的正切值叫做这条直线的斜率,垂直于$x$轴的直线斜率不存在。\\
    特别的,若直线$y=k_1x+b_1$与直线$y=k_2x+b_2$垂直(假设$k_1>0,k_2<0)$,倾斜角分别为$\alpha,\beta$,
    则有$\beta = 90^\circ + \alpha$.
    给出公式$\tan(90^\circ+\alpha)=\cot \alpha$,则有$k_1·k_2=\tan\alpha·\tan\beta=\tan\alpha·\tan(90^\circ+\alpha)=-\tan\alpha·\cot\alpha=-1$.
\end{knowledge}
\begin{figure}[H]
\centering
\includegraphics[width=4cm]{picture/907.png}
\caption{直线斜率与倾斜角正切值}
\end{figure}
\begin{knowledge}
    现推导点到直线距离公式:假设直线$y=kx+b$,直线的倾斜角为$\alpha$,有点$P(x_1,y_1)$,则点$P$到直线的距离$d=\frac{|kx_1-y_1+b|}{\sqrt{1+k^2}}$. \\
    证明:将直线平移,使其经过点$A$,则平移后的直线解析式为$y=k(x-x_1)+y_1$,设平移前的直线与$x$轴的交点坐标为$A(-\frac{b}{k},0)$,
    平移后的直线与$x$轴的交点坐标为$B(\frac{kx_1-y_1}{k},0)$,两条直线的距离则为$AB·\sin \alpha$或$AB·\sin(180^\circ-\alpha)$. \\
    当$0^\circ \le \alpha < 180 ^\circ$时,$\sin \alpha = \sin(180^\circ-\alpha)$,故$d=|\frac{kx_1-y_1+b}{k}|·\sin \alpha$. \\
    由$\tan \alpha = k $,推出$\sin \alpha = \frac{|k|}{\sqrt{1+k^2}}$,所以$d=\frac{|kx_1-y_1+b|}{\sqrt{1+k^2}}$.
\end{knowledge}
\clearpage
\section{*正弦定理和余弦定理}
\begin{knowledge}
    在$\triangle ABC$中,$a,b,c$分别为$\angle A,\angle B,\angle C$的对边,$R$为$\triangle ABC$外接圆的半径。 \\
    三角形面积公式:$S_{\triangle ABC}=\frac{1}{2}ab\sin C=\frac{1}{2}ac\sin B=\frac{1}{2}bc\sin A$ \\
    共角定理:$\frac{S_{\triangle ADE}}{S_{\triangle ABC}}=\frac{AD·AE}{AB·AC}$ \\
    正弦定理:$\frac{a}{\sin A}=\frac{b}{\sin B}=\frac{c}{\sin C}=2R$ \\
    余弦定理:\\
    $a^2=b^2+c^2-2bc\cos A$ \\
    $b^2=a^2+c^2-2ac\cos A$ \\
    $c^2=a^2+b^2-2ab\cos A$ \\
    三角比转换:\\
    $\sin(180^\circ - \alpha)=\sin \alpha$ \\
    $\cos(180^\circ - \alpha)=-\cos \alpha$ \\
    $\tan(180^\circ - \alpha)=-\tan \alpha$ \\
    设$\angle A$为最大角:\\
    若$\triangle ABC$为直角三角形,$a^2=b^2+c^2$; \\
    若$\triangle ABC$为锐角三角形,$a^2<b^2+c^2$; \\
    若$\triangle ABC$为钝角三角形,$a^2>b^2+c^2$. \\
\end{knowledge}
\begin{figure}[H]
\centering
\includegraphics[width=4cm]{picture/902.png}
\caption{共角定理}
\end{figure}
\clearpage
\section{*正切两角和差公式与45度角模型}
\begin{knowledge}
    高中会学习正切的两角和差公式:$\tan(\alpha\pm\beta)=\frac{\tan \alpha \pm \tan \beta}{1 \mp \tan \alpha·\tan \beta}$ \\
    现利用该公式总结如下结论:将一次函数$y=kx+b$绕某点旋转$45^\circ$: \\
    若为顺时针旋转,则旋转后的直线斜率为$\frac{k-1}{k+1}$; \\
    若为逆时针旋转,则旋转后的直线斜率为$\frac{1+k}{1-k}$.
\end{knowledge}
\begin{knowledge}
    如图,$\triangle ABC$三个顶点都在网格线上,假设每个小正方形边长为1,则$AC=\sqrt5,BC=\sqrt5,AC=\sqrt{10},\triangle ABC$为等腰直角三角形。\\
    $\angle ABC=45^\circ$由一个正切值为$\frac{1}{2}$和一个正切值为$\frac{1}{3}$的角组成。\\
    使用两角和正切公式求出$\angle ABC$的正切值如下:\\
    $\tan \angle ABC=\frac{\frac{1}{2}+\frac{1}{3}}{1-\frac{1}{2}·\frac{1}{3}}=1$ \\
\end{knowledge}
\begin{figure}[H]
\centering
\includegraphics[width=4cm]{picture/903.png}
\caption{45度角模型}
\end{figure}
\clearpage
\chapter{二次函数}
\section{二次函数的解析式}
\begin{knowledge}
    形如$y=ax^2+bx+c(a\ne 0)$的函数叫做二次函数。
\end{knowledge}
\begin{table}[H]
\centering
\caption{二次函数的解析式}
\begin{tabular}{l|l|p{7cm}}
\hline
\hline
形式 & 解析式 & 备注 \\
\hline
一般式 & $y=ax^2+bx+c(a\ne 0)$ & \\
\hline
顶点式 & $y=a(x-h)^2+k(a \ne 0)$ & 顶点为$(h,k)$,对称轴为直线$x=h$ \\
\hline
交点式(两根式) & $y=a(x-x_1)(x-x_2)(a \ne 0)$ & 与$x$轴交点为$(x_1,0),(x_2,0)$,交点式存在当且仅当图像与$x$轴有交点\\
\hline
对称式 & $y=a(x-x_1)(x-x_2)+k (a \ne 0)$ & 经过点$(x_1,k),(x_2,k)$ \\
\hline
\hline
\end{tabular}
\end{table}
\begin{knowledge}
    通常,这四种表达形式可以互相转化:\\
    一般式与顶点式的互化:
    $y=ax^2+bx+c \Rightarrow y=a(x+\frac{b}{2a})^2+\frac{4ac-b^2}{4a}$ \\
    $y=a(x-h)^2+k \Rightarrow y=ax^2-2ahx+ah^2+k$ \\
    两根式化为一般式:\\
    $y=a(x-x_1)(x-x_2) \Rightarrow y=ax^2-a(x_1+x_2)x+ax_1x_2$ \\
    对比各项系数,可以得到根与系数的关系(韦达定理):\\
    $x_1+x_2=-\frac{b}{a},x_1x_2=\frac{c}{a}$.
\end{knowledge}
\clearpage
\section{二次函数的图像与性质}
\begin{knowledge}
    二次函数$y=ax^2+bx+c(a\ne 0)$中各系数对图像的影响:
\end{knowledge}
\begin{table}[H]
\centering
\caption{二次函数各系数对图像的影响}
\begin{tabular}{l|l|p{8cm}}
\hline
\hline
系数 & 对图像的影响 & 结论 \\
\hline
$a$ & 开口大小和方向,对称轴 & $|a|$越大,开口越窄;$|a|$越小,开口越宽;$a>0$开口向上;$a<0$开口向下。 \\
$b$ & 对称轴 & 左同右异:$a,b$异号,对称轴在$y$轴右侧;$a,b$同号,对称轴在$y$轴左侧;$b=0\Leftrightarrow$对称轴为$y$轴。 \\
$c$ & 与$y$轴交点纵坐标 & 与$y$轴交于点$(0,c)$,$c=0\Leftrightarrow$ 图像经过原点。 \\
$\Delta=b^2-4ac$ & 与$x$轴交点 & $\Delta<0$,与$x$轴无交点;$\Delta=0$,与$x$轴一个交点;$\Delta>0$,与$x$轴两个交点。\\
\hline
\hline
\end{tabular}
\end{table}
\begin{knowledge}
    二次函数的对称轴与顶点 \\
    二次函数的对称轴:\\
    一般式:$y=ax^2+bx+c(a\ne 0)$ 对称轴为直线$x=-\frac{b}{2a}$ \\
    顶点式:$y=a(x-h)^2+k(a\ne 0)$ 对称轴为直线$x=h$ \\
    两点$(x_1,y_1),(x_2,y_2)(y_1=y_2)$:对称轴为直线$x=\frac{x_1+x_2}{2}$ \\
    已知一点$(x_1,y_1)$求其关于对称轴直线$x=h$的对称点:$(2h-x_1,y_1)$ \\
    二次函数的顶点: \\
    一般式:$y=ax^2+bx+c(a\ne 0)$ 顶点为$(-\frac{b}{2a},\frac{4ac-b^2}{4a})$ \\
    顶点式:$y=a(x-h)^2+k(a\ne 0)$ 顶点为$(h,k)$ \\
    交点式:$y=a(x-x_1)(x-x_2)$ 顶点为$(\frac{x_1+x_2}{2},-\frac{a(x_1-x_2)^2}{4})$
\end{knowledge}
\begin{knowledge}
    二次函数的图像与性质:\\
    $a>0$,图像开口向上,图像一定经过一二象限,顶点为最低点,函数有最小值。在对称轴左侧,$y$随着$x$的增大而减小;在对称轴右侧,$y$随着$x$的增大而增大。\\
    $a<0$,图像开口向下,图像一定经过三四象限,顶点为最高点,函数有最大值。在对称轴左侧,$y$随着$x$的增大而增大;在对称轴右侧,$y$随着$x$的增大而减小。
\end{knowledge}
\begin{knowledge}
    二次函数经过的象限与$a,b,c$的约束如下表所示:
\end{knowledge}
\begin{table}[H]
\centering
\caption{二次函数经过的象限}
\begin{tabular}{l|l}
\hline
\hline
经过的象限 & $a,b,c$的约束 \\
\hline
一、二象限 & $a>0,b^2-4ac \le 0$ \\
一、二、三象限 & $a>0,b>0,c\ge 0,b^2-4ac>0$ \\
一、二、四象限 & $a>0,b<0,c\ge 0,b^2-4ac>0$ \\
一、二、三、四象限 & $ac<0$ \\
三、四象限 & $a<0,b^2-4ac \le 0$ \\
一、三、四象限 & $a<0,b>0,c \le 0,b^2-4ac>0$ \\
二、三、四象限 & $a<0,b<0,c \le 0,b^2-4ac>0$ \\
\hline
\hline
\end{tabular}
\end{table}
\begin{figure}[H]
\centering
\includegraphics[width=12cm]{picture/1006.png}
\caption{决策树}
\end{figure}
\begin{knowledge}
    已知二次函数$y=ax^2+bx+c(a\ne 0)$上两点$(x_1,y_1),(x_2,y_2)(x_1 < x_2)$,对称轴为直线$x=h$,比较$y_1$和$y_2$的大小有口诀:$a>0$,离对称轴越远,函数值越大;$a<0$,离对称轴越远,函数值越小。\\
    具体分类讨论情况如下表所示:
\end{knowledge}
\begin{table}[H]
\centering
\caption{二次函数函数值比较}
\begin{tabular}{l|l}
\hline
\hline
$2h=x_1+x_2$ & $y_1=y_2$ \\
$a>0, x_1 \le h \le x_2, h-x_1>x_2-h$ & $y_1>y_2$ \\
$a>0, x_1 \le h \le x_2, h-x_1<x_2-h$ & $y_1<y_2$ \\
$a>0, h > x_2$ & $y_1>y_2$ \\
$a>0, h < x_1$ & $y_1<y_2$ \\
$a<0, x_1 \le h \le x_2, h-x_1>x_2-h$ & $y_1<y_2$ \\
$a<0, x_1 \le h \le x_2, h-x_1<x_2-h$ & $y_1>y_2$ \\
$a<0, h > x_2$ & $y_1<y_2$ \\
$a<0, h < x_1$ & $y_1>y_2$ \\
\hline
\hline
\end{tabular}
\end{table}
\clearpage
\section{二次函数的图像变换}
\begin{knowledge}
    二次函数图像的平移口诀:左加右减自变量,上加下减常数项。 \\
    任意两个二次函数,只要$a$相同,都能互相通过平移而转换。\\
    例如:$y=ax^2$通过平移得到$y=ax^2+bx+c$,只需要向左(右)平移$|\frac{b}{2a}|$个单位,再向上(下)平移$|\frac{4ac-b^2}{4a}|$个单位。\\
    二次函数沿某个方向平移$m(m>0)$个单位后的解析式如下表所示:
\end{knowledge}
\begin{table}[H]
\centering
\caption{二次函数图像的平移变换}
\begin{tabular}{l|l|l}
\hline
\hline
平移方向 & $y=ax^2+bx+c$ & $y=a(x-h)^2+k$ \\
\hline
向上平移 & $y=ax^2+bx+c+m$ & $y=a(x-h)^2+k+m$ \\
向下平移 & $y=ax^2+bx+c-m$ & $y=a(x-h)^2+k-m$  \\
向左平移 & $y=a(x+m)^2+b(x+m)+c$ & $y=a(x-h+m)^2+k$ \\
向右平移 & $y=a(x-m)^2+b(x-m)+c$ & $y=a(x-h-m)^2+k$ \\
\hline
\hline
\end{tabular}
\end{table}
\begin{knowledge}
    二次函数图像的对称变换如下表所示:
\end{knowledge}
\begin{table}[H]
\centering
\caption{二次函数图像的对称变换}
\begin{tabular}{l|l|l|l}
\hline
\hline
解析式 & 关于$x$轴对称 & 关于$y$轴对称 & 关于原点对称 \\
\hline
$y=ax^2+bx+c$ & $y=-ax^2-bx-c$ & $y=ax^2-bx+c$  & $y=-ax^2+bx-c$ \\
$y=a(x-h)^2+k$ & $y=-a(x-h)^2-k$ & $y=a(x+h)^2+k$ & $y=-a(x+h)^2-k$ \\
\hline
\hline
\end{tabular}
\end{table}
\clearpage
\section{二次函数最值}
\begin{knowledge}
    当$x$的取值范围为全体实数时,二次函数的最值在对称轴处取,即:\\
    $a>0$,$y=ax^2+bx+c(a\ne 0)$有最小值为$\frac{4ac-b^2}{4a}$,$x$取值为$-\frac{b}{2a}$,无最大值。\\
    $a<0$,$y=ax^2+bx+c(a\ne 0)$有最大值为$\frac{4ac-b^2}{4a}$,$x$取值为$-\frac{b}{2a}$,无最小值。 \\
    若给定$x$的取值范围为$m \le x \le n$,对称轴为直线$x=h$,二次函数$f(x)=ax^2+bx+c(a\ne 0)$最值情况如下表所示:
\end{knowledge}
\begin{table}[H]
\centering
\caption{给定区间的二次函数最值}
\begin{tabular}{l|p{5cm}|p{5cm}}
\hline
\hline
 & 最小值 & 最大值 \\
\hline
$a>0, m \le h \le n$ & $f(h)$ & $h-m>n-h$时为$f(m)$ $\qquad h-m\le n-h$时为$f(n)$ \\
$a>0, h > n$ & $f(n)$ & $f(m)$ \\
$a>0, h < m$ & $f(m)$ & $f(n)$ \\
$a<0, m \le h \le n$ & $h-m>n-h$时为$f(m)$ $\qquad h-m\le n-h$时为$f(n)$ & $f(h)$ \\
$a<0, h > n$ & $f(m)$ & $f(n)$ \\
$a<0, h < m$ & $f(n)$ & $f(m)$ \\
\hline
\hline
\end{tabular}
\end{table}
\begin{knowledge}
    对于不同的题目描述方法,转化为最值问题:\\
    图像与$x$轴无交点$\Rightarrow b^2-4ac<0$ \\
    二次函数的函数值恒为负/图像在$x$轴下方$\Rightarrow a<0, b^2-4ac < 0$ \\
    二次函数的函数值恒为正/图像在$x$轴上方$\Rightarrow a>0, b^2-4ac < 0$
\end{knowledge}
\begin{problem}
    $\sqrt{kx^2+4x+3}$的$x$取值范围为全体实数,求$k$的取值范围。
\end{problem}
\begin{problem}
    $\frac{1}{kx^2+4x+3}$的$x$取值范围为全体实数,求$k$的取值范围。
\end{problem}
\clearpage
\section{二次函数与方程不等式}
\begin{knowledge}
    对于$y=ax^2+bx+c(a\ne 0),\Delta = b^2-4ac$,用图像的观点看待二次方程与不等式:
\end{knowledge}
\begin{table}[H]
\centering
\caption{二次函数与方程不等式}
\begin{tabular}{l|p{5cm}|p{3cm}|p{2.8cm}}
\hline
\hline
函数、方程或不等式 & $\Delta > 0 $ & $\Delta = 0 $ & $\Delta < 0 $ \\
\hline
$y=ax^2+bx+c(a>0)$ & 图像与$x$轴交于两点$(x_1,0),(x_2,0)$ & 与$x$轴交于一点,$y$值恒非负 & 与$x$轴无交点,$y$值恒为正 \\
$y=ax^2+bx+c(a<0)$ & 图像与$x$轴交于两点$(x_1,0),(x_2,0)$ & 与$x$轴交于一点,$y$值恒非正 & 与$x$轴无交点,$y$值恒为负 \\
$ax^2+bx+c=0(a>0)$ & $x_1=\frac{-b-\sqrt\Delta}{2a},x_2=\frac{-b+\sqrt\Delta}{2a}$ & $x_1=x_2=-\frac{b}{2a}$ & 无实数解 \\
$ax^2+bx+c>0(a>0)$ & $x<x_1$或$x>x_2(x_1<x_2)$ & $x \ne -\frac{b}{2a}$ & 全体实数 \\
$ax^2+bx+c<0(a>0)$ & $x_1<x<x_2(x_1<x_2)$ & 无实数解 & 无实数解 \\
\hline
\hline
\end{tabular}
\end{table}
\clearpage
\section{*二次函数与直线型}
\begin{knowledge}
    判断其交点情况二次函数$y=ax^2+bx+c(a \ne 0)$与一次函数$y=mx+n(m \ne 0)$交点情况 \\
    联立$ax^2+bx+c=mx+n$,整理得$ax^2+(b-m)x+(c-n)=0,\Delta=(b-m)^2-4a(c-n)$. \\
    $\Delta > 0 \Rightarrow$ 有两个交点 \\
    $\Delta = 0 \Rightarrow$ 有一个交点(相切) \\
    $\Delta < 0 \Rightarrow$ 没有交点 \\
    二次函数$y=ax^2+bx+c(a \ne 0)$还与直线$y=\frac{4ac-b^2}{4ac}$相切,切点为顶点。
\end{knowledge}
\begin{knowledge}
    二次函数$y=ax^2+bx+c(a \ne 0)$与一次函数$y=mx+n(m \ne 0)$相切于点$(x_0,y_0)$时的直线斜率$m=2ax_0+b$ \\
    联立$ax^2+bx+c=mx+n$,整理得$ax^2+(b-m)x+(c-n)=0,\Delta=(b-m)^2-4a(c-n)$. \\
    因为$\Delta=(b-m)^2-4a(c-n)=0$,解得$x_0=\frac{m-b}{2a}$,故$m=2ax_0+b$ \\
\end{knowledge}
\clearpage
\section{*二次函数与角平分线}
\begin{model}
    如图,不平行于$x$轴的直线$EQF$与抛物线$y=ax^2$交于$E,F$两点,在$y$轴上找一点$P$,使得$y$轴平分$\angle EPF$. \\
    设$E(x_1,y_1),F(x_2,y_2),Q(0,y_3)$,则$P$点坐标为$(0,-y_3)$. \\
    证明:设$P$点坐标为$(0,y_4)$,设直线解析式为$y=kx+b(k\ne 0)$,与$y=ax^2$联立,由韦达定理得$x_1+x_2=\frac{k}{a},x_1x_2=-\frac{b}{a}$. \\
    $\tan \angle EPQ = -\frac{y_1-y_4}{x_1} = \tan \angle FPQ = \frac{y_2-y_4}{x_2} \Rightarrow x_1y_2+x_2y_1=y_4(x_1+x_2) $ \\
    $x_1(kx_2+b)+x_2(kx_1+b)=y_4(x_1+x_2) \Rightarrow 2kx_1x_2+b(x_1+x_2)=y_4(x_1+x_2)$ \\
    $-\frac{2kb}{a}+\frac{kb}{a}=\frac{k}{a}·y_4 \Rightarrow y_4=-b$
\end{model}
\begin{figure}[H]
\centering
\includegraphics[width=4cm]{picture/1001.png}
\caption{二次函数与角平分线}
\end{figure}
\begin{problem}
    如图1,抛物线$y=ax^2+bx+3$经过点$A(-3,0),B(-1,0)$两点。\\
    (1)求抛物线的解析式;\\
    (2)设抛物线的顶点为$M$,直线$y=-2x+9$与$y$轴交于点$C$,与直线$OM$交于点$D$,现将抛物线平移,保持顶点在直线$OD$上,
    若平移的抛物线与射线$CD$(含端点$C$)只有一个公共点,求它顶点横坐标的值或取值范围;\\
    (3)如图2,将抛物线平移,当顶点至原点时,过$Q(0,3)$作不平行于$x$轴的直线交抛物线于$E,F$两点,问在$y$轴的负半轴上是否存在一点$P$,
    使$\triangle PEF$的内心在$y$轴上?若存在,求出点$P$的坐标;若不存在,说明理由。
\end{problem}
\begin{figure}[H]
\centering
\includegraphics[width=8cm]{picture/1002.png}
\caption{2011武汉中考25题}
\end{figure}
\clearpage
\section{*抛物线的焦点与准线}
\begin{knowledge}
    平面内,到定点与定直线的距离相等的点的轨迹叫做抛物线。其中定点叫抛物线的焦点,定直线叫抛物线的准线。\\
    对于二次函数$y=ax^2(a\ne 0)$,其焦点坐标为$(0,\frac{1}{4a})$,其准线为直线$y=-\frac{1}{4a}$,图像上任意一点到焦点的距离和到准线的距离相等。\\
    证明:设二次函数上某点坐标为$(x,ax^2)$,其到焦点的距离为$\sqrt{x^2+(ax^2-\frac{1}{4a})^2}=\sqrt{(ax^2+\frac{1}{4a})^2}=|ax^2+\frac{1}{4a}|$,
    等于其到准线的距离。\\
    对于更一般的二次函数$y=ax^2+bx+c(a\ne 0)$,只需要对$y=ax^2(a\ne 0)$进行平移变换即可,
    可推导其焦点为$(-\frac{b}{2a},\frac{4ac-b^2+1}{4a})$,准线为直线$y=\frac{4ac-b^2-1}{4a}$. \\
    若为顶点式$y=a(x-h)^2+k(a \ne 0)$,其焦点为$(h,k+\frac{1}{4a})$,准线为直线$y=k-\frac{1}{4a}$. \\
\end{knowledge}
\begin{figure}[H]
\centering
\includegraphics[width=4cm]{picture/1003.png}
\caption{抛物线的焦点与准线}
\end{figure}
\begin{model}
    如图,直线$MN$经过点$F(0,\frac{1}{4a})$与二次函数$y=ax^2$交于$M,N$两点,分别过$M,N$作垂线$MM_1,NN_1$垂直于直线$y=-\frac{1}{4a}$,
    有如下结论成立:\\
    (1)$MM_1+NN_1=MN$ \\
    (2)以$MN$为直径的圆与$M_1N_1$相切 \\
    (3)$M_1F \perp N_1F$ \\
    (4)以$AB$为直径的圆与$PQ$相切 \\
    (5)$\frac{1}{MM_1}+\frac{1}{NN_1}=\frac{2}{FF_1}$
\end{model}
\begin{figure}[H]
\centering
\includegraphics[width=4cm]{picture/1004.png}
\caption{抛物线与直角梯形}
\end{figure}
\clearpage
%\section{二次函数50问}
%\begin{example}
%    如图,点$A,B,C$为二次函数$y=x^2+bx+c$与坐标轴的交点,$OA=OC=3$,顶点为$D$. \\
%    (1)求二次函数的解析式。 \\
%    (2)判断$\triangle ACD$的形状,并说明理由。 \\
%    (3)求四边形$ABCD$的面积。 \\
%    (4)在对称轴上找一点$P$,使得$\triangle BCP$周长最小,求出$P$点坐标及$\triangle BCP$的周长。 \\
%    (5)在直线$AC$下方的抛物线上有一点$N$,过点$N$作直线$l\px\px y$轴,交$AC$于点$M$,当点$N$坐标为何值时,线段$MN$的长度最大?并求出最大值。\\
%    (6)在直线$AC$下方的抛物线上,是否存在点$N$使得$\triangle ACN$面积最大?并求出最大面积。 \\
%    (7)在直线$AC$下方的抛物线上,是否存在点$N$使得四边形$ABCN$面积最大?并求出最大面积。 \\
%    (8)在$y$轴上存在点$E$,使得$\triangle ADE$为直角三角形,求出所有点$E$的坐标。 \\
%    (9)在$y$轴上存在点$F$,使得$\triangle ADF$为直角三角形,求出所有点$F$的坐标。 \\
%    (10)在抛物线上存在点$N$,使得$S_{\triangle ABN}=S_{\triangle ABC}$,求出所有点$N$的坐标。 \\
%    (11)在抛物线上存在点$H$,使得$S_{\triangle BCH}=S_{\triangle ABC}$,求出所有点$H$的坐标。 \\
%    (12)在抛物线上存在点$Q$,使得$S_{\triangle AOQ}=S_{\triangle COQ}$,求出所有点$Q$的坐标。 \\
%    (13)在抛物线上存在一点$E$,使得$BE$平分$\triangle ABC$的面积,求出点$E$的坐标。 \\
%    (14)在抛物线上存在一点$F$,作$FM\perp x$轴交$AC$于$H$,$AC$平分$\triangle AFM$的面积。 \\
%    (15)在抛物线对称轴上有一点$K$,抛物线上有一点$L$,若以$A,B,K,L$为顶点的四边形是平行四边形,求出所有可能的$K,L$两点坐标。\\
%    (16)作垂直于$x$轴的直线$x=-1$交直线$AC$于点$M$,交抛物线于点$N$,若以$A,M,N,E$为顶点的四边形是平行四边形,求点$E$的坐标。 \\
%    (17)在抛物线上存在点$P$,使得$\angle POC=\angle PCO$,求出所有的点$P$的坐标。 \\
%    (18)在线段$AC$上存在点$M$,使得$\triangle AOM \sim \triangle ABC$,求出所有的点$M$的坐标。 \\
%    (19)抛物线上存在点$E$,作$EH\perp x$轴于$H$,若$\triangle EAH$与$\triangle OBC$相似,求点$E$的坐标。 \\
%    (20)若点$P$以单位速度自$A$向$B$运动,点$Q$以同速自$O$向$C$运动。当某一点到达终点时,另一点也停止运动,设运动时间为t,$\triangle OPQ$面积为
%    S,求$S$与$t$之间的函数关系式,并求出$S$的最大值。\\
%    (21)动点$E$在$y$轴上,动点$F$在坐标平面内,且$A,D,E,F$构成菱形,求出点$E,F$的坐标。 \\
%    (22)动点$E$在$y$轴上,动点$F$在坐标平面内,且$A,D,E,F$构成矩形,求出点$E,F$的坐标。 \\
%    (23)点$P$为抛物线对称轴上一点,且使得$|PA-PC|$值最大,求出点$P$的坐标和$|PA-PC|$的最大值。 \\
%    (24)在直线$AC$下方的抛物线上存在一点$P$使得$P$到直线$AC$的距离最大,求出点$P$的坐标。 \\
%    (25)在直线$AC$下方的抛物线上存在一点$P$使得$P$到直线$AC$的距离为$\sqrt2$,求出点$P$的坐标。 \\
%    (26)在直线$AD$上,存在一点$P$使得$BP+CP$最小,求点$P$的坐标。 \\
%    (27)在直线$AC$上,存在一点$P$使得$BP+\frac{\sqrt3}{2}CP$最小,求点$P$的坐标和最小值。\\
%    (28)点$E$是线段$AC$上一动点,点$P$是线段$AB$上一动点,$PE\px \px BC$,若$P$点使得$\triangle PEC$面积最大,求出点$P$的坐标和面积的最大值。\\
%    (29)点$P$是直线$AC$下方抛物线上一个动点,过点$P$作$PE\perp x$轴交$AC$于点$E$,$PF\perp AC$于点$F$。求点$P$,使得$\triangle PEF$周长最大。\\
%    (30)判断点$O$关于直线$AC$的对称点是否在抛物线上。\\
%    (31)动点$E$在坐标轴上,若$\triangle ACE$为直角三角形,求点$E$所有可能的坐标。 \\
%    (32)动点$E$在坐标轴上,动点$P$在坐标平面内,$A,C,P,E$四点构成正方形顶点,求点$P$所有可能的坐标。\\
%    (33)对称轴交$x$轴于点$E$,以$E$为圆心,$\frac{3\sqrt{10}}{5}$为半径作圆,判断直线$BC$与圆的位置关系。\\
%    (34)动点$P$在直线$AC$上,动点$M$在抛物线上,且$PM\px \px AB$,$P,M,A,B$构成平行四边形,求点$P$的坐标。\\
%    (35)对称轴交$x$轴于点$E$,动点$M$在$x$轴上,动点$P$在抛物线上,且$PM\px\px ED$,$P,M,E,D$构成平行四边形,求点$P$的坐标。\\
%    (36)动点$M$在$x$轴上,动点$P$在抛物线上,且$PM \px\px AC$,$P,M,A,C$构成平行四边形,求$P$点坐标。\\
%    (37)已知点$F(-1,-\frac{15}{4})$,求证:$PD=PF$. \\
%    (38)已知点$F(-1,-\frac{15}{4})$,$\triangle PDF$是等边三角形,求点$P$的坐标。\\
%    (39)已知点$E(\frac{1}{2},2)$,求$PD+PE$的最小值。\\
%    (40)点$F$为平面内一定点,且$PD=PF$,求点$F$的坐标。\\
%    (41)
%    (42)
%    (43)
%    (44)
%    (45)
%    (46)
%    (47)
%    (48)
%    (49)
%    (50)
%\end{example}
\chapter{圆}
\section{圆的确定}
\begin{knowledge}
    圆是在同一平面内到定点距离等于定长的点的集合。定点即圆心,定长即半径长。\\
    同圆:圆心相同、半径相等的圆。\\
    等圆:半径相等的圆。\\
    同心圆:圆心相同、半径不相等的圆。
\end{knowledge}
\begin{knowledge}
    圆是轴对称、中心对称、旋转对称图形。有无数条经过圆心的对称轴,绕圆心旋转任意角度都能和自身重合。
\end{knowledge}
\begin{knowledge}
    圆的确定 \\
    (1)圆心和半径都确定时圆才能确定。\\
    (2)经过一个已知点能作无数个圆。 \\
    (3)经过两个已知点$A,B$能作无数个圆,这些圆的圆心在线段$AB$的垂直平分线上。\\
    (4)不在同一直线上的三个点确定一个圆,这三个点可以组成一个三角形,圆心是的三角形的三条垂直平分线的交点(三角形的外心),
    圆是三角形的外接圆,三角形是圆的内接三角形。\\
    (5)过$n(n\ge 4)$个点的圆,只可能作0个或1个圆,若能作一个圆,则圆心是任意三点确定的圆的圆心。
\end{knowledge}
\clearpage
\section{圆心角、弧、弦、弦心距的关系}
\begin{knowledge}
    圆心角、弦、弧、弦心距 \\
    圆心角:顶点在圆心的角。 \\
    弦:连接圆上任意两点的线段。经过圆心的弦叫做直径,直径是同一圆中最长的弦,等于半径的2倍。\\
    弧:圆上任意两点间的部分叫做圆弧,在同圆或等圆中,能够互相重合的弧叫做等弧。圆的任意一条直径的两个端点把圆分成两条弧,每条弧都叫做半圆。
    在一个圆中,大于半圆的弧叫做优弧,小于半圆的弧叫做劣弧。\\
    弦心距:圆心到弦长的距离。 \\
    如图,$\bigodot O$中,$OM\perp AB$,$\angle BOA$为圆心角,$AB$为弦,$\mathop{AB}\limits^\frown$为弧,$OM$为弦心距。
\end{knowledge}
\begin{figure}[H]
\centering
\includegraphics[width=4cm]{picture/1118.png}
\caption{圆心角、弦、弧、弦心距}
\end{figure}
\begin{problem}
    下面说法正确的有:\\
    (1)直径是圆中最长的弦 \\
    (2)长度相等的两条弧是等弧 \\
    (3)面积相等的两个圆是等圆 \\
    (4)半径相等的两个半圆是等弧 \\
    (5)弧分为优弧和劣弧 \\
    (6)过圆内一定点可以作无数条直径 \\
    (7)过圆内一定点可以作无数条弦 \\
    (8)经过圆心的线段是直径 \\
    (9)半径是弦
\end{problem}
\begin{knowledge}
    圆心角、弧、弦、弦心距的关系 \\
    在同圆或等圆中,相等的圆心角所对的弧相等,所对的弦相等,弦的弦心距相等。\\
    推论:在同圆或等圆中,如果两个圆心角、两条弧、两条弦、两条弦心距中有一组量相等,那么它们所对应的其余各组量分别相等。
    如图,在同圆或等圆中,若有如下条件: \\
    (1)$AB=A'B'$ \\
    (2)$\mathop{AB}\limits^{\frown}=\mathop{A'B'}\limits^{\frown}$ \\
    (3)$\angle AOB=\angle A'OB'$ \\
    (4)$OD=OD'$ \\
    任选一个条件,可以推出其他三个条件。
\end{knowledge}
\begin{figure}[H]
\centering
\includegraphics[width=8cm]{picture/1117.png}
\caption{圆心角、弧、弦、弦心距的关系}
\end{figure}
\clearpage
\section{垂径定理}
\begin{knowledge}
    垂径定理 \\
    垂直于弦的直径平分弦,并且平分弦所对的两条弧。\\
    如图:$CD\perp AB \Rightarrow AE=AB,\mathop{AC}\limits^{\frown}=\mathop{BC}\limits^{\frown}$. \\
    推论:平分弦(不是直径)的直径垂直于弦,并且平分弦所对的两条弧。\\
    如图:$AE=EB \Rightarrow CD\perp AB,\mathop{AC}\limits^{\frown}=\mathop{BC}\limits^{\frown}$.
\end{knowledge}
\begin{figure}[H]
\centering
\includegraphics[width=4cm]{picture/1113.png}
\caption{垂径定理}
\end{figure}
\begin{knowledge}
    半径$(r)$、弦长$(l)$与弦心距$(d)$的关系 \\
    $r^2=d^2+\frac{l^2}{4}$ \\
    如上图,$OA^2=OE^2+\frac{AB^2}{4}$
\end{knowledge}
\clearpage
\section{*圆周角定理与弦切角定理}
\begin{knowledge}
     顶点在圆上,并且两边都和圆相交的角叫圆周角。
\end{knowledge}
\begin{knowledge}
    圆周角定理 \\
    在同圆或等圆中,同弧或等弧所对的圆周角相等,都等于这条弧所对的圆心角的一半。\\
    推论1:在同圆或等圆中,如果两个圆周角相等,它们所对的弧一定相等。\\
    推论2:半圆(或直径)所对的圆周角是直角,$90^\circ$的圆周角所对的弦是直径。
\end{knowledge}
\begin{figure}[H]
\centering
\includegraphics[width=8cm]{picture/864.png}
\caption{圆周角定理}
\end{figure}
\begin{knowledge}
    如图,$\triangle ABC,OA=OB=OC$,求证:$\angle ACB=90^\circ$. \\
    证明:$OA=OC \Rightarrow \angle OAC=\angle OCA;OB=OC \Rightarrow \angle OBC=\angle OCB;$ \\
    $\angle OAC+\angle OCA+\angle OCB+\angle OBC=180^\circ \Rightarrow \angle OCA+\angle OCB=\angle ACB=90^\circ$.
\end{knowledge}
\begin{figure}[H]
\centering
\includegraphics[width=4cm]{picture/1106.png}
\caption{直径所对的圆周角是直角}
\end{figure}
\begin{knowledge}
    弦切角定理 \\
    弦切角(弦与切线的夹角)等于弦所夹的弧所对的圆周角。\\
    如图,$CE$为$\bigodot O$的切线,$\angle BCE = \angle BAC$.
\end{knowledge}
\begin{figure}[H]
\centering
\includegraphics[width=8cm]{picture/1107.png}
\caption{弦切角定理}
\end{figure}
\clearpage
\section{点和圆的位置关系}
\begin{knowledge}
    设$\bigodot O$半径为$r$,点到圆心$O$的距离为$d$,点和圆有如下位置关系:\\
    点在圆外$\Leftrightarrow d > r$ \\
    点在圆上$\Leftrightarrow d = r$ \\
    点在圆内$\Leftrightarrow d < r$
\end{knowledge}
\begin{figure}[H]
\centering
\includegraphics[width=4cm]{picture/1108.png}
\caption{点和圆的位置关系}
\end{figure}
\begin{problem}
    一个点到圆的最大距离为$m$,最小距离为$n(m>n)$,求圆的半径。
\end{problem}
\clearpage
\section{直线和圆的位置关系}
\begin{knowledge}
    设$\bigodot O$半径为$r$,直线$l$到圆心$O$的距离为$d$,直线和圆有如下位置关系:\\
    相离(直线与圆没有公共点)$\Leftrightarrow d > r$ \\
    相切(直线与圆有唯一公共点,直线叫做圆的切线,公共点叫做切点)$\Leftrightarrow d = r$ \\
    相交(直线与圆有两个公共点,直线叫做圆的割线)$\Leftrightarrow d < r$
\end{knowledge}
\begin{figure}[H]
\centering
\includegraphics[width=8cm]{picture/1109.png}
\caption{直线和圆的位置关系}
\end{figure}
\begin{knowledge}
    切线的判定 \\
    定理:经过半径的外端并且垂直于这条半径的直线是圆的切线 \\
    定义:和圆只有一个公共点的直线是圆的切线 \\
    距离:和圆心距离等于半径的直线是圆的切线
\end{knowledge}
\begin{knowledge}
    切线的性质 \\
    定理:圆的切线垂直于过切点的半径 \\
    推论1:经过圆心且垂直于切线的直线必经过切点 \\
    推论2:经过切点且垂直与切线的直线必经过圆心
\end{knowledge}
\begin{knowledge}
    切线长定理 \\
    在经过圆外一点的圆的切线上,这点和切点之间的线段长叫做这点到圆的切线长。\\
    从圆外一点引圆的两条切线,它们的切线长相等,圆心和这一点的连线平分两条切线的夹角。
\end{knowledge}
\begin{figure}[H]
\centering
\includegraphics[width=4cm]{picture/1110.png}
\caption{切线长定理}
\end{figure}
\clearpage
\section{圆和圆的位置关系}
\begin{knowledge}
    设$\bigodot O_1$半径为$r_1$,$\bigodot O_2$半径为$r_2$,圆心距$d$,圆和圆有如下位置关系: \\
    外离$\Leftrightarrow d > r_1+r_2$ \\
    外切$\Leftrightarrow d = r_1+r_2$ \\
    相交$\Leftrightarrow |r_1-r_2|<d < r_1+r_2$ \\
    内切$\Leftrightarrow d = |r_1-r_2|$ \\
    内含$\Leftrightarrow 0 \le d < |r_1-r_2|$
\end{knowledge}
\begin{figure}[H]
\centering
\includegraphics[width=8cm]{picture/1111.png}
\caption{圆和圆的位置关系}
\end{figure}
\begin{knowledge}
    如图,$\bigodot O_1$半径为$r_1$,$\bigodot O_2$半径为$r_2$,两圆圆心距为$d$. \\
    外公切线:两圆在公切线的同旁 \\
    内公切线:两圆在公切线的两旁 \\
    如左图,$AB,CD$为外公切线,长度为$\sqrt{d^2-(r_1-r_2)^2}.$ \\
    如右图,$AB,CD$为内公切线,长度为$\sqrt{d^2-(r_1+r_2)^2}.$
\end{knowledge}
\begin{figure}[H]
\centering
\includegraphics[width=8cm]{picture/1101.png}
\caption{公切线}
\end{figure}
\begin{table}[H]
\centering
\caption{公切线}
\begin{tabular}{l|l|l|l}
\hline
\hline
两圆位置关系 & 外公切线条数 & 内共切线条数 & 总条数 \\
\hline
外离 & 2 & 2 & 4 \\
外切 & 2 & 1 & 3 \\
相交 & 2 & 0 & 2 \\
内切 & 1 & 0 & 1 \\
外离 & 0 & 0 & 0 \\
\hline
\hline
\end{tabular}
\end{table}
\clearpage
\section{正多边形与圆}
\begin{knowledge}
    各边相等,各个内角相等的多边形叫做正多边形。所有的正多边形均有一个外接圆。\\
    (1)正多边形的中心:正多边形的外接圆的圆心 \\
    (2)正多边形的半径:正多边形的外接圆的半径 \\
    (3)正多边形的中心角:正多边形每一边所对的圆心角 \\
    (4)正多边形的边心距:中心到正多边形的一边的距离
\end{knowledge}
\begin{figure}[H]
\centering
\includegraphics[width=8cm]{picture/1112.png}
\caption{正多边形与圆}
\end{figure}
\begin{knowledge}
    正多边形的性质 \\
    (1)正$n$边形的半径和边心距把正$n$边形分成$2n$个全等的直角三角形 \\
    (2)正多边形有$n$条通过中心的对称轴 \\
    (3)奇数条边的正多边形的对称轴是顶点和对边中点的连线 \\
    (4)偶数条边的正多边形的对称轴是相对顶点的连线和对边中点的连线 \\
    (3)偶数条边的正多边形是中心对称图形
\end{knowledge}
\clearpage
\section{*三角形的外接圆与内切圆}
\begin{knowledge}
    经过三角形三个顶点的圆叫做三角形的外接圆,三角形叫做圆的内接三角形。\\
    (1)三角形的外接圆有且只有一个 \\
    (2)一个圆的内接三角形有无数个 \\
    (3)外接圆圆心(三角形外心)是三角形的三条垂直平分线的交点 \\
    (4)锐角三角形外接圆圆心在其内部 \\
    (5)直角三角形外接圆圆心在斜边中点 \\
    (6)钝角三角形外接圆圆心在其外部
\end{knowledge}
\begin{figure}[H]
\centering
\includegraphics[width=8cm]{picture/1114.png}
\caption{三角形外接圆}
\end{figure}
\begin{knowledge}
    和三角形三边都相切的圆叫做三角形的内切圆,设$\triangle ABC$周长为$C$,内切圆半径为$r$.\\
    (1)三角形的内切圆有且只有一个 \\
    (2)内切圆圆心(三角形内心)是三角形的三条角平分线的交点 \\
    (3)$S_{\triangle ABC}=\frac{1}{2}C·r$ \\
    证明(3):$S_{\triangle ABC}=S_{\triangle AOC}+S_{\triangle BOC}+S_{\triangle AOB}=\frac{1}{2}(AC+BC+AB)r$
\end{knowledge}
\begin{figure}[H]
\centering
\includegraphics[width=4cm]{picture/1115.png}
\caption{三角形内切圆}
\end{figure}
\begin{knowledge}
    直角三角形的外接圆和内切圆 \\
    如图,$\triangle ACB,\angle ACB=90^\circ,\bigodot O$是其内切圆,内切圆半径为$r$,外接圆半径为$R$.\\
    (1)$AE=AF\quad  CE=CD\quad BD=BF$ \\
    (2)四边形$CEOD$为正方形 \\
    (3)$r=\frac{AC+BC-AB}{2}$ \\
    (4)$R=\frac{AB}{2}$
\end{knowledge}
\begin{figure}[H]
\centering
\includegraphics[width=4cm]{picture/1116.png}
\caption{直角三角形的外接圆和内切圆}
\end{figure}
\begin{problem}
    如图,$\triangle ABC$中,$H$为三条高线的交点,$OM\perp BC$于$M$. \\
    (1)求证:$AH=2OM$; \\
    (2)若$\angle BAC=60^\circ$,求证:$AH=AO$.
\end{problem}
\begin{figure}[H]
\centering
\includegraphics[width=4cm]{picture/1119.png}
\caption{练习图}
\end{figure}
\clearpage
\section{*圆内接四边形与托勒密定理}
\begin{knowledge}
    若一个四边形四个顶点都在同一个圆上,则四边形叫做圆内接四边形,圆叫做四边形的外接圆。\\
    圆内接四边形的对角互补。\\
    圆内接四边形的一个外角等于它的内对角。
\end{knowledge}
\begin{knowledge}
    托勒密定理 \\
    圆内接四边形$ABCD$,两条对角线的乘积等于两组对边乘积之和,即$AB·CD+AD·BC=AC·BD$. \\
    证明:作$\angle BAE=\angle CAD$交$BD$于$E$,易证$\triangle ABE\sim \triangle ACD,\triangle ABC \sim \triangle AED$. \\
    $\triangle ABE \sim \triangle ACD \Rightarrow \frac{AB}{AC}=\frac{BE}{CD}=\frac{AE}{AD} \Rightarrow AB·CD=AC·BE$ \\
    $\triangle ABC \sim \triangle AED \Rightarrow \frac{AB}{AE}=\frac{BC}{ED}=\frac{AC}{AD} \Rightarrow AD·BC=AC·ED$ \\
    $AB·CD+AD·BC=AC(BE+ED)=AC·BD$ \\
\end{knowledge}
\begin{figure}[H]
\centering
\includegraphics[width=8cm]{picture/1102.png}
\caption{托勒密定理}
\end{figure}
\begin{problem}
    不使用托勒密定理,证明如下结论:\\
    (1)如左图,$\triangle ABC$为等边三角形,求证:$BD=AD+CD$. \\
    (2)如右图,$\triangle ABC$为等腰直角三角形,求证:$\sqrt{2}AD=BD+CD$.
\end{problem}
\begin{figure}[H]
\centering
\includegraphics[width=8cm]{picture/1103.png}
\caption{练习图}
\end{figure}
\clearpage
\section{*四点共圆}
\begin{knowledge}
    四点共圆的判定方法:\\
    (1)若四个点到一定点的距离相等,则这四个点共圆。\\
    (2)若一个四边形的一组对角的和等于180度,则这个四边形四个顶点共圆。\\
    (3)若一个四边形的一个外角等于它的内对角,则这个四边形四个顶点共圆。\\
    (4)若两个点在一条线段的同旁,并且和这条线段的两端连线所夹的角相等,那么这两点和这条线的两个端点共圆。
\end{knowledge}
\begin{figure}[H]
\centering
\includegraphics[width=6cm]{picture/862.png}
\caption{四点共圆}
\end{figure}
\clearpage
\section{*圆幂定理}
\begin{knowledge}
    圆幂定理:交点为$P$的两条相交直线与圆$O$相交于$A,B$与$C,D$,则有$PA·PB=PC·PD$,证明方法:$\triangle PAD \sim \triangle PCB$
    或$\triangle PAC \sim \triangle PDB$. \\
    (1)相交弦定理:圆内的两条弦$AB,CD$交于$P$.\\
    (2)割线定理:从圆外一点$P$引圆的两条割线$AB,CD$.\\
    (3)切割线定理:从圆外一点$P$引圆的一条切线$PA(B)$和一条割线$CD$. \\
    (4)切线长定理:从圆外一点$P$引圆的两条切线$PA(B),PC(D)$,此时$PA(B)=PC(D)$.
\end{knowledge}
\begin{figure}[H]
\centering
\includegraphics[width=12cm]{picture/863.png}
\caption{圆幂定理}
\end{figure}
\clearpage
\section{*阿基米德折弦定理}
\begin{model}
    如图,$AB,BC$是$\bigodot O$两条弦,$M$是$\mathop{ABC}\limits^{\frown}$的中点,$MD\perp BC$,垂足为$D$,求证:$AB+BD=CD$. \\
    证明:在$BC$上截取$CG=AB$,$\mathop{AM}\limits^{\frown}=\mathop{MC}\limits^{\frown} \Rightarrow AM=MC$,则$\triangle ABM \backcong \triangle CGM$,
    $MD\perp BC,BM=MG \Rightarrow \triangle BMD \backcong \triangle GMD \Rightarrow BD=DG \Rightarrow AB+BD=CD$.
\end{model}
\begin{figure}[H]
\centering
\includegraphics[width=4cm]{picture/1104.png}
\caption{阿基米德折弦定理}
\end{figure}
\begin{problem}
    如图,$\triangle ABC$内接于$\bigodot O,BC=2,AB=AC$,点$D$为$\mathop{AC}\limits^{\frown}$上一动点,且$\cos \angle ABC=\frac{\sqrt{10}}{10}$. \\
    (1)求$AB$的长度;\\
    (2)在点$D$的运动过程中,弦$AD$的延长线交$BC$延长线于点$E$,求$AD·AE$的值;\\
    (3)在点$D$的运动过程中,过$A$点作$AH\perp BD$,求证:$BH=CD+DH$.
\end{problem}
\begin{figure}[H]
\centering
\includegraphics[width=4cm]{picture/1105.png}
\caption{2018深圳中考}
\end{figure}
\clearpage
\chapter{统计初步}
\section{数据整理与表示}
\begin{knowledge}
    统计图 \\
    条形图:有利于比较数据间的差异 \\
    折线图:直观反映数据的变化趋势 \\
    扇形图:体现数据部分与整体的关系
\end{knowledge}
\begin{knowledge}
    统计学基本概念 \\
    总体:所有考察对象的全体 \\
    个体:总体中每一个考察对象 \\
    样本:从总体中抽取出的一部分个体 \\
    样本容量:样本中个体的数目
\end{knowledge}
\clearpage
\section{反映一组数据平均水平、波动程度、数据分布的量}
\begin{knowledge}
    反映一组数据平均水平的量 \\
    平均数:$\overline x = \frac{1}{n}(x_1+x_2+...+x_n)$ \\
    加权平均数:$\overline x=\frac{x_1f_1+x_2f_2+...+x_nf_n}{f_1+f_2+...+f_n}$
\end{knowledge}
\begin{knowledge}
    反映一组数据波动程度的量 \\
    众数:在一组数据中出现次数最多的数据 \\
    中位数:一组数据排序后排在最中间的一个数或最中间两个数的平均数 \\
    若有$n$个数,当$n$为奇数时,中位数为第$\frac{n+1}{2}$个数;当$n$为偶数时,中位数为第$\frac{n}{2},\frac{n}{2}+1$个数的平均数。\\
    一组数据的中位数一定唯一,众数可能有多个,也可能没有(所有数出现次数一样多)。\\
    平均数能反映所有数据的情况,但容易受极端数据的影响,中位数和众数不受极端数值的影响,但不能反映所有数据的情况。\\
    方差:$s^2=\frac{1}{n}[(x_1-\overline x)^2+(x_2-\overline x)^2+...+(x_n-\overline x)^2]$ \\
    标准差:$s=\sqrt{s^2}=\sqrt{\frac{1}{n}[(x_1-\overline x)^2+(x_2-\overline x)^2+...+(x_n-\overline x)^2]}$ \\
    数据的方差(或标准差)越小,说明这组数据越稳定,数据的波动越小,数据的离散程度越小,数据越集中。
\end{knowledge}
\begin{knowledge}
    若一组数据的平均数为$\overline x$,方差为$s^2$,标准差为$s$. \\
    (1)若所有数据都加上一个常数$b$,则平均数变为$\overline x+b$,方差和标准差不变。 \\
    (2)若所有数据都乘上一个常数$A$,则平均数变为$A\overline x$,方差变为$A^2s^2$,标准差变为$|A|s$.
\end{knowledge}
\begin{knowledge}
    反映一组数据分布的量 \\
    频数:每个对象出现的次数 \\
    频率:每个对象出现的次数与总次数的比值  \\
    组数:把全体样本分成的组的个数 \\
    组距:每个小组的两个端点的距离 \\
    极差:一组数据中最大数据与最小数据的差 \\
    绘制频数分布直方图时,若分组过少,则数据非常集中;若分组过多,则数据非常分散。
\end{knowledge}
\begin{knowledge}
    频数分布直方图与条形图的区别 \\
    (1)条形图用条形的高度表示频数的大小,直方图用长方形的面积表示频数。\\
    (2)条形图横轴的数据是孤立的,是具体的数据;直方图横轴的数据是连续的,是一个范围。 \\
    (3)条形图各个长方形之间有空隙,直方图各个长方形之间无空隙。
\end{knowledge}
\begin{example}
    某学校准备从63名初三学生中挑选出身高接近的40名同学参加比赛,收集到的63名同学身高如下(单位cm):\\
    158 158 160 168 159 159 151 158 159 \\
    168 158 154 158 154 169 158 158 158 \\
    159 167 170 153 160 160 159 159 160 \\
    149 163 163 162 172 161 153 156 162 \\
    162 163 157 162 162 161 157 157 164 \\
    155 156 165 166 156 154 166 164 165 \\
    156 157 153 165 159 157 155 164 156 \\
    选择怎样身高范围的学生比较合理呢?\\
    分析:以上数据,最小值为149cm,最大值为172cm,差值为23cm,采用等距分组,每3cm分一组,一共8组。\\
    身高范围:$149 \le x < 152, 152 \le x < 155,...,170 \le x < 173$,组距为3,组数为8. \\
    统计频数如下表:身高在155cm至164cm(不含164cm)的学生有41人,可以从这个身高范围中选取学生。
\end{example}
\begin{table}[H]
\centering
\caption{频数分布表}
\begin{tabular}{l|p{1.5cm}|p{1.5cm}|p{1.5cm}|p{1.5cm}|p{1.5cm}|p{1.5cm}|p{1.5cm}|p{1.5cm}}
\hline
分组 & $149\le x<152$ & $152\le x<155$ & $155\le x<158$ & $158\le x<161$ & $161\le x<164$ & $164\le x<167$ & $167\le x<170$ & $170\le x<173$\\
\hline
频数 & 2 & 6 & 12 & 19 & 10 & 8 & 4 & 2 \\
\hline
\end{tabular}
\end{table}
\begin{figure}[H]
\centering
\includegraphics[width=12cm]{picture/1200.jpg}
\caption{频数分布直方图}
\end{figure}
\begin{knowledge}
    频数分布折线图 \\
    取直方图各矩形上边的中点,在横轴上取两个频数为0的点,这两点分别与直方图左右两端的两个长方形的组中值(矩形宽的中点)相距一个组距,
    将这些点用线段依次联结起来,就可以得到频数分布折线图。
\end{knowledge}
\clearpage
% 正文后部分
\backmatter



\end{document}
